\chapter{Prerequisiti Matematici Essenziali}\label{ch:prerequisiti}
% Capitolo generale: basi di algebra lineare, probabilità/statistica e ottimizzazione.
% Stile discorsivo: definizioni, idee chiave, esempi numerici brevi e promemoria pratici.

\section{Orientamento e notazione}\label{sec:notazione}
Pensiamo a un vettore come a una freccia nello spazio (coordinate) e a una matrice come a un "trasformatore" di vettori.
\begin{itemize}
  \item Vettori colonna in grassetto: \(\mathbf{x}\in\mathbb{R}^d\). Matrici in maiuscolo: \(A\in\mathbb{R}^{m\times n}\). Trasposta: \(A^\top\).
  \item Prodotto scalare: \(\langle\mathbf{x},\mathbf{y}\rangle=\mathbf{x}^\top\mathbf{y}\). Norme: \(\|\mathbf{x}\|_2=\sqrt{\sum_i x_i^2}\), \(\|\mathbf{x}\|_1=\sum_i |x_i|\), \(\|\mathbf{x}\|_\infty=\max_i |x_i|\).
  \item Identità: \(I\). Vettore nullo: \(\mathbf{0}\).
\end{itemize}

\section{Vettori e matrici}\label{sec:vett-matr}
\subsection{Combinazioni lineari e prodotto matrice–vettore}
Una combinazione lineare di \(\mathbf{v}_1,\dots,\mathbf{v}_k\) è \(\sum_i \alpha_i\mathbf{v}_i\). Il prodotto \(A\mathbf{x}\) è una combinazione delle colonne di \(A\) con pesi i componenti di \(\mathbf{x}\).
\paragraph{Esempio.} Se \(A=\begin{bmatrix}1&2\\0&-1\end{bmatrix}\) e \(\mathbf{x}=\begin{bmatrix}3\\1\end{bmatrix}\), allora \(A\mathbf{x}=3\begin{bmatrix}1\\0\end{bmatrix}+1\begin{bmatrix}2\\-1\end{bmatrix}=\begin{bmatrix}5\\-1\end{bmatrix}.\)

\subsection{Prodotto matrice–matrice}
La riga \(i\) di \(AB\) si ottiene moltiplicando la riga \(i\) di \(A\) per ogni colonna di \(B\). Non è commutativo.
\paragraph{Esempio.} \(\begin{bmatrix}1&0\\2&1\end{bmatrix}\begin{bmatrix}3&1\\-1&2\end{bmatrix}=\begin{bmatrix}3&1\\5&4\end{bmatrix}\), ma invertendo l'ordine il risultato cambia.

\section{Distanze e similarità}\label{sec:norme}
\subsection{Norme classiche}
\begin{align}
\|\mathbf{x}\|_1&=\sum_i |x_i|, & \|\mathbf{x}\|_2&=\sqrt{\sum_i x_i^2}, & \|\mathbf{x}\|_\infty&=\max_i |x_i|.\label{eq:norme}
\end{align}
\paragraph{Esempio.} Per \(\mathbf{x}=(3,-4)\): \(\|\mathbf{x}\|_1=7\), \(\|\mathbf{x}\|_2=5\), \(\|\mathbf{x}\|_\infty=4\).
\subsection{Prodotto scalare e angolo}
\[\langle\mathbf{x},\mathbf{y}\rangle=\|\mathbf{x}\|_2\,\|\mathbf{y}\|_2\,\cos\theta\ \Rightarrow\ \cos\theta=\dfrac{\mathbf{x}^\top\mathbf{y}}{\|\mathbf{x}\|_2\,\|\mathbf{y}\|_2}.\]
\paragraph{Esempio (cosine).} \(\mathbf{x}=(1,0,1)\), \(\mathbf{y}=(1,1,0)\): \(\mathbf{x}^\top\mathbf{y}=1\), \(\|\mathbf{x}\|_2=\|\mathbf{y}\|_2=\sqrt{2}\) \(\Rightarrow\cos\theta=1/2\).
\subsection{Jaccard per insiemi}
Per insiemi \(A,B\): \(J(A,B)=\tfrac{|A\cap B|}{|A\cup B|}\). Utile con basket o set di parole.

\section{Sottospazi, basi e rango}\label{sec:span-rank}
Lo span di \(\{\mathbf{v}_1,\dots,\mathbf{v}_k\}\) è l'insieme delle combinazioni lineari possibili. Una base è un insieme indipendente che genera lo spazio.
\paragraph{Rango.} \(\mathrm{rank}(A)\) conta quante direzioni indipendenti contengono le colonne (o righe) di \(A\).
\paragraph{Esempio.} In \(\mathbb{R}^2\), \((1,0)\) e \((2,0)\) sono dipendenti (stessa direzione): rango 1. \((1,0)\) e \((0,1)\) sono indipendenti: rango 2.

\section{Proiezioni ortogonali e minimi quadrati}\label{sec:proiezioni}
\subsection{Proiezione su una direzione}
Se \(\mathbf{u}\) è unitario, \(\mathrm{proj}_{\mathbf{u}}(\mathbf{x})=(\mathbf{u}^\top\mathbf{x})\,\mathbf{u}\).
\paragraph{Esempio (retta \(y=x\)).} \(\mathbf{u}=\tfrac{1}{\sqrt{2}}(1,1)\), \(\mathbf{x}=(2,0)\). Allora \(\mathbf{u}^\top\mathbf{x}=\sqrt{2}\) e la proiezione è \(\sqrt{2}\,\mathbf{u}=(1,1)\).
\subsection{Minimi quadrati in due righe}
Dato \(A\in\mathbb{R}^{m\times n}\) (\(m\ge n\)) e \(\mathbf{b}\), risolvi \(\min_{\mathbf{x}}\|A\mathbf{x}-\mathbf{b}\|_2\). Le equazioni normali sono \(A^\top A\,\mathbf{x}=A^\top\mathbf{b}\).
\paragraph{Esempio.} \(A=\begin{bmatrix}1\\1\\1\end{bmatrix}\), \(\mathbf{b}=\begin{bmatrix}2\\3\\4\end{bmatrix}\). Qui \(x\) è lo scalare che approssima nel senso LS: \(A^\top A=3\), \(A^\top\mathbf{b}=9\) \(\Rightarrow x=3\).

\section{Autovalori e autovettori}\label{sec:eig}
Per \(A\in\mathbb{R}^{n\times n}\), \(A\mathbf{v}=\lambda\mathbf{v}\) significa che \(\mathbf{v}\) è una direzione lasciata invariata (a fattore \(\lambda\)). Se \(C=C^\top\) è simmetrica:
\begin{itemize}
  \item gli autovalori \(\lambda\) sono reali;
  \item autovettori di autovalori diversi sono ortogonali;
  \item esiste \(P\) ortogonale con \(C=P\,\Lambda\,P^\top\) (teorema spettrale).
\end{itemize}
\paragraph{Esempio.} \(C=\begin{bmatrix}2&1\\1&2\end{bmatrix}\): autovalori \(3,1\) con autovettori proporzionali a \((1,1)\) e \((1,-1)\).

\section{Probabilità e statistica}\label{sec:prob}
\subsection{Attesa e varianza}
Per variabile discreta \(X\): \(\mathbb{E}[X]=\sum_x x\,P(X=x)\), \(\mathrm{Var}(X)=\mathbb{E}[(X-\mu)^2]\). Linearità: \(\mathbb{E}[aX+bY]=a\,\mathbb{E}[X]+b\,\mathbb{E}[Y]\).
\paragraph{Esempio.} Dado equo: \(\mu=3{,}5\); \(\mathrm{Var}(X)=\tfrac{35}{12}\).
\subsection{Covarianza e correlazione}
\[\mathrm{Cov}(X,Y)=\mathbb{E}[(X-\mu_X)(Y-\mu_Y)],\qquad \rho=\tfrac{\mathrm{Cov}(X,Y)}{\sigma_X\sigma_Y}.\]
\paragraph{Matrice di covarianza (dati centrati).} Per \(X\in\mathbb{R}^{n\times d}\): \(C=\tfrac{1}{n}X^\top X\).
\paragraph{Esempio.} Dati centrati \((1,2),(2,3),(3,4)\) su due feature: \(C=\bigl[\begin{smallmatrix}1&1\\1&1\end{smallmatrix}\bigr]\).
\subsection{Quantili e IQR}
Il \(p\)-quantile è la soglia sotto cui cade una frazione \(p\) dei dati ordinati (mediana = 50\%). L'IQR è \(Q_3-Q_1\). Regola outlier: \([Q_1-1{,}5\,\mathrm{IQR},\,Q_3+1{,}5\,\mathrm{IQR}]\).
\paragraph{Esempio.} Dati ordinati \([3,5,7,8,9,10,13,15,20]\): \(Q_1\approx6\), mediana \(=9\), \(Q_3\approx14\), IQR \(=8\).

\section{Preprocessing numerico}\label{sec:scaling}
\paragraph{Standardizzazione (z-score).} \(z_{ij}=\dfrac{x_{ij}-\mu_j}{\sigma_j}\). Quando usarla: feature con scale diverse.
\paragraph{Min--max scaling.} \(y_{ij}=\dfrac{x_{ij}-\min_j}{\max_j-\min_j}\in[0,1]\). Nota: sensibile agli outlier.
\paragraph{Robust scaling.} \(r_{ij}=\dfrac{x_{ij}-\mathrm{mediana}_j}{\mathrm{IQR}_j}\). Pro: robusto ai valori estremi.
\paragraph{Codifiche categoriali.} One-hot: per categorie \(\{c_1,\dots,c_g\}\), \(c_i\mapsto\mathbf{e}_i\in\{0,1\}^g\).
\paragraph{Esempio.} Se media=50 e dev.stand.=10, il valore 65 diventa \(z=1{,}5\); con min--max su \([0,100]\) diventa \(0{,}65\).

\section{Ottimizzazione in pratica}\label{sec:opt}
\paragraph{Gradiente.} Per \(f:\mathbb{R}^d\to\mathbb{R}\), \(\nabla f(\mathbf{x})=[\partial f/\partial x_1,\dots,\partial f/\partial x_d]^\top\).
\paragraph{Discesa del gradiente.} \(\mathbf{x}^{(t+1)}=\mathbf{x}^{(t)}-\eta\,\nabla f(\mathbf{x}^{(t)})\).
\paragraph{Convessità (idea).} Se \(f\) è convessa, ogni minimo locale è globale.
\paragraph{Esempio 1D.} \(f(x)=(x-3)^2\). Con \(x^{(0)}=0\), \(\eta=0{,}5\): \(x^{(1)}=1{,}5\), \(x^{(2)}=2{,}25\), \(x^{(3)}=2{,}625\) \(\to\) converge a 3.

\section{Combinatoria utile}\label{sec:comb}
\paragraph{Coefficienti binomiali.} \(\binom{n}{k}=\dfrac{n!}{k!(n-k)!}\). Esempio: 3-itemset da 100 item = \(\binom{100}{3}=161700\).
\paragraph{Permutazioni.} \(P(n,k)=n\,(n-1)\cdots(n-k+1)\): sequenze ordinate senza ripetizione.

% --- Fine Capitolo 0 ---