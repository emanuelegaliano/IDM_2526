% !TEX encoding = UTF-8 Unicode
\documentclass[a4paper,11pt]{book}

% ===== Pacchetti essenziali =====
\usepackage[utf8]{inputenc}     % codifica sorgente (se usi pdfLaTeX)
\usepackage[T1]{fontenc}        % output font con accenti corretti
\usepackage[italian]{babel}     % sillabazione e testi automatici in italiano
\usepackage{lmodern}            % font Latin Modern
\usepackage{microtype}          % migliore giustificazione del testo

% ===== Matematica =====
\usepackage{amsmath,amssymb,amsthm}
\numberwithin{equation}{chapter}

% ===== Impaginazione =====
\usepackage{geometry}
% Margini comodi per tablet/stampa: non stretti, ma senza sprecare spazio
\geometry{a4paper, top=30mm, bottom=32mm, left=30mm, right=30mm}                                        

\usepackage{setspace}
% Interlinea: ben leggibile ma con buona densità di contenuto
\setstretch{1.15}

% Evita vedove/orfani (migliora la lettura su pagina singola)
\clubpenalty=10000
\widowpenalty=10000

% Profondità di numerazione / indice (come nello screenshot: 1.1.1)
\setcounter{secnumdepth}{3} % numerazione fino a \subsubsection
\setcounter{tocdepth}{2}    % TOC fino a \subsection (metti 3 se vuoi anche le subsubsection)

% ===== Intestazioni e piè di pagina =====
\usepackage{fancyhdr}
\pagestyle{fancy}
\fancyhf{} % pulisci
\fancyhead[LE,RO]{\thepage}
\fancyhead[LO]{\nouppercase{\rightmark}}
\fancyhead[RE]{\nouppercase{\leftmark}}
\setlength{\headheight}{15pt}
\addtolength{\topmargin}{-2pt}

% ===== Tabelle, grafica,    link =====
\usepackage{booktabs}
\usepackage{graphicx}
\usepackage[hidelinks]{hyperref}

\usepackage{csquotes}
\usepackage{biblatex}
\addbibresource{refs.bib}

% ===== Liste: leggibili, non compattate eccessivamente =====
\usepackage{enumitem}
% itemize / enumerate: spazi moderati per mantenere i dettagli visivi
\setlist[itemize]{topsep=4pt, partopsep=0pt, itemsep=3pt, parsep=2pt}
\setlist[enumerate]{topsep=4pt, partopsep=0pt, itemsep=3pt, parsep=2pt}
% description: rientro e allineamento gradevole; etichetta in grassetto
\setlist[description]{font=\normalfont\bfseries, labelsep=0.6em, leftmargin=2em}

% (Facoltativo) Se vuoi che gli elenchi siano più “a colonna” in certi punti:
% \begin{description}[widest=Etichetta-più-lunga,leftmargin=!,labelsep=0.6em]
%   \item[Etichetta] Testo...
% \end{description}

% ===== Metadati =====
\title{\Huge \bfseries Riassunti e Appunti di Introduzione al Data Mining}
\author{\Large Emanuele Galiano \\
Studente di Informatica \\
Università di Catania}
\date{Anno Accademico 2025/2026}

\begin{document}

\frontmatter
\maketitle
\tableofcontents

\mainmatter
\part{Concetti Preliminari}
\chapter{Prerequisiti Matematici Essenziali [WIP]}\label{ch:prerequisiti}
% Capitolo generale: basi di algebra lineare, probabilità/statistica e ottimizzazione.
% Stile discorsivo: definizioni, idee chiave, esempi numerici brevi e promemoria pratici.

\section{Orientamento e notazione}\label{sec:notazione}
Pensiamo a un vettore come a una freccia nello spazio (coordinate) e a una matrice come a un "trasformatore" di vettori.
\begin{itemize}
  \item Vettori colonna in grassetto: \(\mathbf{x}\in\mathbb{R}^d\). Matrici in maiuscolo: \(A\in\mathbb{R}^{m\times n}\). Trasposta: \(A^\top\).
  \item Prodotto scalare: \(\langle\mathbf{x},\mathbf{y}\rangle=\mathbf{x}^\top\mathbf{y}\). Norme: \(\|\mathbf{x}\|_2=\sqrt{\sum_i x_i^2}\), \(\|\mathbf{x}\|_1=\sum_i |x_i|\), \(\|\mathbf{x}\|_\infty=\max_i |x_i|\).
  \item Identità: \(I\). Vettore nullo: \(\mathbf{0}\).
\end{itemize}

\section{Vettori e matrici}\label{sec:vett-matr}
\subsection{Combinazioni lineari e prodotto matrice–vettore}
Una combinazione lineare di \(\mathbf{v}_1,\dots,\mathbf{v}_k\) è \(\sum_i \alpha_i\mathbf{v}_i\). Il prodotto \(A\mathbf{x}\) è una combinazione delle colonne di \(A\) con pesi i componenti di \(\mathbf{x}\).
\paragraph{Esempio.} Se \(A=\begin{bmatrix}1&2\\0&-1\end{bmatrix}\) e \(\mathbf{x}=\begin{bmatrix}3\\1\end{bmatrix}\), allora \(A\mathbf{x}=3\begin{bmatrix}1\\0\end{bmatrix}+1\begin{bmatrix}2\\-1\end{bmatrix}=\begin{bmatrix}5\\-1\end{bmatrix}.\)

\subsection{Prodotto matrice–matrice}
La riga \(i\) di \(AB\) si ottiene moltiplicando la riga \(i\) di \(A\) per ogni colonna di \(B\). Non è commutativo.
\paragraph{Esempio.} \(\begin{bmatrix}1&0\\2&1\end{bmatrix}\begin{bmatrix}3&1\\-1&2\end{bmatrix}=\begin{bmatrix}3&1\\5&4\end{bmatrix}\), ma invertendo l'ordine il risultato cambia.

\section{Distanze e similarità}\label{sec:norme}
\subsection{Norme classiche}
\begin{align}
\|\mathbf{x}\|_1&=\sum_i |x_i|, & \|\mathbf{x}\|_2&=\sqrt{\sum_i x_i^2}, & \|\mathbf{x}\|_\infty&=\max_i |x_i|.\label{eq:norme}
\end{align}
\paragraph{Esempio.} Per \(\mathbf{x}=(3,-4)\): \(\|\mathbf{x}\|_1=7\), \(\|\mathbf{x}\|_2=5\), \(\|\mathbf{x}\|_\infty=4\).
\subsection{Prodotto scalare e angolo}
\[\langle\mathbf{x},\mathbf{y}\rangle=\|\mathbf{x}\|_2\,\|\mathbf{y}\|_2\,\cos\theta\ \Rightarrow\ \cos\theta=\dfrac{\mathbf{x}^\top\mathbf{y}}{\|\mathbf{x}\|_2\,\|\mathbf{y}\|_2}.\]
\paragraph{Esempio (cosine).} \(\mathbf{x}=(1,0,1)\), \(\mathbf{y}=(1,1,0)\): \(\mathbf{x}^\top\mathbf{y}=1\), \(\|\mathbf{x}\|_2=\|\mathbf{y}\|_2=\sqrt{2}\) \(\Rightarrow\cos\theta=1/2\).
\subsection{Jaccard per insiemi}
Per insiemi \(A,B\): \(J(A,B)=\tfrac{|A\cap B|}{|A\cup B|}\). Utile con basket o set di parole.

\section{Sottospazi, basi e rango}\label{sec:span-rank}
Lo span di \(\{\mathbf{v}_1,\dots,\mathbf{v}_k\}\) è l'insieme delle combinazioni lineari possibili. Una base è un insieme indipendente che genera lo spazio.
\paragraph{Rango.} \(\mathrm{rank}(A)\) conta quante direzioni indipendenti contengono le colonne (o righe) di \(A\).
\paragraph{Esempio.} In \(\mathbb{R}^2\), \((1,0)\) e \((2,0)\) sono dipendenti (stessa direzione): rango 1. \((1,0)\) e \((0,1)\) sono indipendenti: rango 2.

\section{Proiezioni ortogonali e minimi quadrati}\label{sec:proiezioni}
\subsection{Proiezione su una direzione}
Se \(\mathbf{u}\) è unitario, \(\mathrm{proj}_{\mathbf{u}}(\mathbf{x})=(\mathbf{u}^\top\mathbf{x})\,\mathbf{u}\).
\paragraph{Esempio (retta \(y=x\)).} \(\mathbf{u}=\tfrac{1}{\sqrt{2}}(1,1)\), \(\mathbf{x}=(2,0)\). Allora \(\mathbf{u}^\top\mathbf{x}=\sqrt{2}\) e la proiezione è \(\sqrt{2}\,\mathbf{u}=(1,1)\).
\subsection{Minimi quadrati in due righe}
Dato \(A\in\mathbb{R}^{m\times n}\) (\(m\ge n\)) e \(\mathbf{b}\), risolvi \(\min_{\mathbf{x}}\|A\mathbf{x}-\mathbf{b}\|_2\). Le equazioni normali sono \(A^\top A\,\mathbf{x}=A^\top\mathbf{b}\).
\paragraph{Esempio.} \(A=\begin{bmatrix}1\\1\\1\end{bmatrix}\), \(\mathbf{b}=\begin{bmatrix}2\\3\\4\end{bmatrix}\). Qui \(x\) è lo scalare che approssima nel senso LS: \(A^\top A=3\), \(A^\top\mathbf{b}=9\) \(\Rightarrow x=3\).

\section{Autovalori e autovettori}\label{sec:eig}
Per \(A\in\mathbb{R}^{n\times n}\), \(A\mathbf{v}=\lambda\mathbf{v}\) significa che \(\mathbf{v}\) è una direzione lasciata invariata (a fattore \(\lambda\)). Se \(C=C^\top\) è simmetrica:
\begin{itemize}
  \item gli autovalori \(\lambda\) sono reali;
  \item autovettori di autovalori diversi sono ortogonali;
  \item esiste \(P\) ortogonale con \(C=P\,\Lambda\,P^\top\) (teorema spettrale).
\end{itemize}
\paragraph{Esempio.} \(C=\begin{bmatrix}2&1\\1&2\end{bmatrix}\): autovalori \(3,1\) con autovettori proporzionali a \((1,1)\) e \((1,-1)\).

\section{Probabilità e statistica}\label{sec:prob}
\subsection{Attesa e varianza}
Per variabile discreta \(X\): \(\mathbb{E}[X]=\sum_x x\,P(X=x)\), \(\mathrm{Var}(X)=\mathbb{E}[(X-\mu)^2]\). Linearità: \(\mathbb{E}[aX+bY]=a\,\mathbb{E}[X]+b\,\mathbb{E}[Y]\).
\paragraph{Esempio.} Dado equo: \(\mu=3{,}5\); \(\mathrm{Var}(X)=\tfrac{35}{12}\).
\subsection{Covarianza e correlazione}
\[\mathrm{Cov}(X,Y)=\mathbb{E}[(X-\mu_X)(Y-\mu_Y)],\qquad \rho=\tfrac{\mathrm{Cov}(X,Y)}{\sigma_X\sigma_Y}.\]
\paragraph{Matrice di covarianza (dati centrati).} Per \(X\in\mathbb{R}^{n\times d}\): \(C=\tfrac{1}{n}X^\top X\).
\paragraph{Esempio.} Dati centrati \((1,2),(2,3),(3,4)\) su due feature: \(C=\bigl[\begin{smallmatrix}1&1\\1&1\end{smallmatrix}\bigr]\).
\subsection{Quantili e IQR}
Il \(p\)-quantile è la soglia sotto cui cade una frazione \(p\) dei dati ordinati (mediana = 50\%). L'IQR è \(Q_3-Q_1\). Regola outlier: \([Q_1-1{,}5\,\mathrm{IQR},\,Q_3+1{,}5\,\mathrm{IQR}]\).
\paragraph{Esempio.} Dati ordinati \([3,5,7,8,9,10,13,15,20]\): \(Q_1\approx6\), mediana \(=9\), \(Q_3\approx14\), IQR \(=8\).

\subsection{Modello di Bayes e tipi di probabilità}\label{subsec:bayes}
Il \textbf{modello di Bayes} spiega come aggiornare in modo coerente la probabilità di
un’ipotesi quando osserviamo nuovi dati.

\paragraph{Tipi di probabilità.}
\begin{itemize}
  \item \textbf{Congiunta} $P(X,Y)$: probabilità che due eventi/variabili si verifichino insieme.
  \item \textbf{Marginale} $P(X)$: probabilità “totale” su $X$ ottenuta dalla congiunta,
        ad es.\ $P(X)=\sum_{y}P(X,Y)$ (discreto) oppure $P(X)=\int f_{X,Y}(x,y)\,dy$ (continuo).
  \item \textbf{Condizionata} $P(A\mid B)=\dfrac{P(A\cap B)}{P(B)}$ (per $P(B)>0$);
        in termini di variabili $P(X\mid Y)=\dfrac{P(X,Y)}{P(Y)}$.
  \item \textbf{Indipendenza} $X\perp Y$ se e solo se $P(X,Y)=P(X)P(Y)$ (equivale a $P(X\mid Y)=P(X)$).
\end{itemize}

\paragraph{Teorema di Bayes.}
Per ipotesi $h_1,\dots,h_K$ e un’osservazione $x$:
\[
P(h_k\mid x)=\frac{\underbrace{P(x\mid h_k)}_{\text{verosimiglianza}}\,
\underbrace{P(h_k)}_{\text{priori}}}
{\underbrace{P(x)}_{\text{evidenza}}},\qquad
P(x)=\sum_{j=1}^{K}P(x\mid h_j)P(h_j).
\]
Il \emph{posterior} $P(h_k\mid x)$ è la nostra opinione aggiornata su $h_k$; $P(h_k)$ è il \emph{prior};
$P(x\mid h_k)$ è la \emph{verosimiglianza} del dato sotto $h_k$; $P(x)$ è l’\emph{evidenza} (normalizzatore).
Per variabili continue, si sostituiscono le somme con integrali.

\section{Preprocessing numerico}\label{sec:scaling}
\paragraph{Standardizzazione (z-score).} \(z_{ij}=\dfrac{x_{ij}-\mu_j}{\sigma_j}\). Quando usarla: feature con scale diverse.
\paragraph{Min--max scaling.} \(y_{ij}=\dfrac{x_{ij}-\min_j}{\max_j-\min_j}\in[0,1]\). Nota: sensibile agli outlier.
\paragraph{Robust scaling.} \(r_{ij}=\dfrac{x_{ij}-\mathrm{mediana}_j}{\mathrm{IQR}_j}\). Pro: robusto ai valori estremi.
\paragraph{Codifiche categoriali.} One-hot: per categorie \(\{c_1,\dots,c_g\}\), \(c_i\mapsto\mathbf{e}_i\in\{0,1\}^g\).
\paragraph{Esempio.} Se media=50 e dev.stand.=10, il valore 65 diventa \(z=1{,}5\); con min--max su \([0,100]\) diventa \(0{,}65\).

\section{Combinatoria utile}\label{sec:comb}
\paragraph{Coefficienti binomiali.} \(\binom{n}{k}=\dfrac{n!}{k!(n-k)!}\). Esempio: 3-itemset da 100 item = \(\binom{100}{3}=161700\).
\paragraph{Permutazioni.} \(P(n,k)=n\,(n-1)\cdots(n-k+1)\): sequenze ordinate senza ripetizione.

\section{Entropia}\label{sec:entropy}

\subsection{Definizione}
Sia \(X\) una variabile aleatoria discreta che assume valori in un insieme finito \(\mathcal{X}\), con massa di probabilità \(p_X(x)=\Pr[X=x]\).
L'\textbf{entropia (di Shannon)} di \(X\) è
\[
H_b(X)\;=\;-\sum_{x\in\mathcal{X}} p_X(x)\,\log_b p_X(x),
\]
dove \(b\) è la base del logaritmo (tipicamente \(b=2\), quindi l'unità di misura è il \emph{bit}; se \(b=e\) l'unità è il \emph{nat}). Per convenzione \(0\log 0:=0\).

\paragraph{Proprietà essenziali.}
\[
0 \le H_b(X) \le \log_b |\mathcal{X}|,\qquad
H_b(X)=0 \ \Leftrightarrow\ X \text{ è deterministica},\qquad
H_b(X)=\log_b|\mathcal{X}| \ \Leftrightarrow\ X \text{ è uniforme}.
\]
In pratica, più è \emph{dispersa} la distribuzione di \(X\), maggiore è l'entropia.

\paragraph{Stima empirica.}
Dati \(n\) campioni \(x_1,\dots,x_n\) e le frequenze \(n_x=|\{i:\,x_i=x\}|\), la stima plug--in è
\[
\widehat{H}_b(X)\;=\;-\sum_{x\in\mathcal{X}} \frac{n_x}{n}\,\log_b\!\Big(\frac{n_x}{n}\Big).
\]
Nel caso di etichette di classe \(Y\), \(\widehat{H}_2(Y)\) misura l'\emph{impurità} del dataset ed è usata in molti criteri di splitting (es.\ \emph{information gain}).

\subsection{In parole più semplici}
L'entropia quantifica \textbf{quanta incertezza} c'è in un fenomeno casuale: è alta quando i risultati sono imprevedibili e tutti più o meno uguali in probabilità, è bassa quando un risultato è quasi certo.

\paragraph{Esempio (moneta).}
Se una moneta è equa, testa/croce valgono \(50\%\) ciascuno: \(H_2=1\) bit (serve, in media, 1 bit per descrivere l'esito).
Se la moneta è sbilanciata, ad esempio \(\Pr[\text{testa}]=0.99\), l'entropia scende a
\[
H_2 \approx -0.99\log_2 0.99 - 0.01\log_2 0.01 \approx 0.080\text{ bit},
\]
perché l'esito è quasi sempre testa: c'è poca incertezza.
\chapter{Grafi}
I grafi risolvono uno dei problemi più comuni in informatica: la rappresentazione e l'analisi delle relazioni tra oggetti. Molto spesso, infatti, i dati non sono semplicemente una lista di elementi ma possono essere visti come un insieme di connessioni tra degli elementi: basti pensare a una rete sociale, dove gli utenti sono collegati tra loro da amicizie, oppure a una mappa stradale, dove le città sono collegate da strade. In questi casi, i grafi forniscono un modo efficace per rappresentare e analizzare queste relazioni.

\section{Definizione formale}
Un graffo è una coppia ordinata \( G = (V, E) \), dove:
\begin{itemize}
    \item \( V \) è un insieme non vuoto di vertici (o nodi).
    \item \( E \) è un insieme di archi (o spigoli), che sono coppie ordinate o non ordinate di vertici.
\end{itemize}
Gli archi possono essere diretti (nel caso di grafi orientati) o non diretti (nel caso di grafi non orientati).

\begin{figure}[htbp]
    \centering
    \includegraphics[width=0.6\textwidth]{images/grafo_esemipo.png}
    \caption{Grafo non orientato con quattro vertici \(V=\{A,B,C,D\}\) e archi \(E=\{(A,B),(A,C),(A,D),(B,D),(C,D)\}\). In figura: A è collegato a B (in alto), a C (sinistra) e a D (diagonale); B è collegato a D (destra); C è collegato a D (in basso).}
    \label{fig:grafo_esempio}
\end{figure}

\subsection{Network science}
La scienza che studia i grafi, o reti in questo caso, è la \textbf{network science}. É una disciplina interdisciplinare che combina elementi di matematica, informatica, fisica e sociologia per analizzare e comprendere le strutture complesse delle reti.

Un esempio recente di Network science è il \emph{Covid}: Gli epidemiologi hanno utilizzato modelli basati su grafi per tracciare la diffusione del virus, identificare i nodi critici (come le persone più connesse) e prevedere l'impatto delle misure di contenimento.

\section{Grafi diretti e indiretti}
I grafi possono essere classificati in due categorie principali: grafi diretti (o orientati) e grafi indiretti (o non orientati):
\begin{description}
    \item[Grafo diretto.] In un grafo diretto, gli archi hanno una direzione specifica, rappresentata da una freccia. Questo significa che la relazione tra due vertici è unidirezionale. Ad esempio, in un grafo che rappresenta le relazioni di follower su un social media, se l'utente A segue l'utente B, c'è un arco diretto da A a B, ma non necessariamente da B ad A.
    \item[Grafo indiretto.] In un grafo indiretto, gli archi non hanno una direzione specifica. Questo significa che la relazione tra due vertici è bidirezionale, ovvero per ogni arco $(a, b) $ esiste anche l'arco $(b, a)$. Ad esempio, in un grafo che rappresenta le amicizie in un social media, se l'utente A è amico dell'utente B, c'è un arco non diretto tra A e B.
\end{description}

\begin{figure}[htbp]
    \centering
    \includegraphics[width=0.8\textwidth]{images/grafo_diretto_indiretto.png}
    \caption{Confronto tra grafo non orientato (a sinistra) e grafo orientato (a destra). Nel grafo orientato, gli archi hanno una direzione indicata da frecce, mentre nel grafo non orientato rappresentano relazioni bidirezionali.}
    \label{fig:grafo_diretto_indiretto}
\end{figure}

\paragraph{Come capire che tipologia usare.} La scelta tra grafo diretto e indiretto dipende dalla natura delle relazioni che si desidera rappresentare. Se le relazioni sono unidirezionali, come nel caso dei follower sui social media, è appropriato utilizzare un grafo diretto. Se le relazioni sono bidirezionali, come nel caso delle amicizie, è più adatto utilizzare un grafo indiretto.

\section{Grafi pesati e grafi etichettati}
Oltre alla distinzione tra grafi diretti e indiretti, i grafi possono essere ulteriormente classificati in grafi pesati e grafi etichettati:
\begin{description}
    \item[Grafo pesato.] In un grafo pesato, ogni arco ha un peso associato, che rappresenta il costo, la distanza o qualsiasi altra misura quantitativa tra i vertici collegati. Ad esempio, in un grafo che rappresenta una rete stradale, il peso degli archi può rappresentare la distanza tra le città.
    \item[Grafo etichettato.] In un grafo etichettato, i vertici e/o gli archi hanno etichette o nomi associati, che forniscono informazioni aggiuntive sui nodi o sulle relazioni. Ad esempio, in un grafo che rappresenta una rete sociale, i vertici possono essere etichettati con i nomi degli utenti.
\end{description}

\begin{figure}[htbp]
    \centering
    \includegraphics[width=0.8\textwidth]{images/grafo_pesato_etichettato.png}
    \caption{Esempio di grafo pesato (a sinistra) ed etichettato (a destra). Nel grafo pesato, i numeri sugli archi rappresentano i pesi associati a ciascun arco. Nel grafo etichettato, i vertici sono etichettati con nomi specifici.}
    \label{fig:grafo_pesato_etichettato}
\end{figure}

\section{Gradi dei vertici}
Il grado di un vertice in un grafo rappresenta il numero di archi che sono collegati a quel vertice. In un grafo indiretto, il grado di un vertice \( v \) è semplicemente il numero di archi incidenti su \( v \).

\noindent
Nel caso particolare di un \textbf{grafo diretto} si fa distinzione tra:
\begin{itemize}
    \item \textbf{Grado entrante} (in-degree): il numero di archi che arrivano al vertice \( v \).
    \item \textbf{Grado uscente} (out-degree): il numero di archi che partono dal vertice \( v \).
\end{itemize}

\noindent
Poi si parla di \textbf{grado totale} di un vertice \( v \) come la somma del grado entrante e del grado uscente.

Si può parlare anche di \textbf{grado medio} di un grafo, che rappresenta la media dei gradi di tutti i vertici nel grafo. In un grafo indiretto, il grado medio \( \bar{k} \) può essere calcolato utilizzando la formula:
\[
\bar{k} = \frac{|E|}{|V|}
\]
dove \( |E| \) è il numero totale di archi e \( |V| \) è il numero totale di vertici nel grafo.

\subsection{Distribuzione dei gradi}
La distribuzione dei gradi di un grafo descrive come i gradi dei vertici sono distribuiti all'interno del grafo.

Questa è una distribuzione di probabilità $P$ dove $p_k$ è la probabilità che un vertice scelto a caso abbia grado $k$. 

In una rete reale, $p_k$ si ottiene dividendo il numero $N_k$ di nodi con grado $k$ per il numero totale $N$ di nodi nella rete:
\[
p_k = \frac{N_k}{N}
\]

\section{Grafi bipartiti}
Un grafo bipartito è un tipo speciale di grafo in cui i vertici possono essere divisi in due insiemi disgiunti \( U \) e \( V \) tali che ogni arco collega un vertice in \( U \) a un vertice in \( V \). Non ci sono archi che collegano vertici all'interno dello stesso insieme.

\begin{figure}[htbp]
    \centering
    \includegraphics[width=0.8\textwidth]{images/grafo_bipartito_esempio.png}
    \caption{Esempio di grafo bipartito con partizioni \(U\) (nodi verdi) e \(V\) (nodi viola): gli archi collegano solo vertici appartenenti a insiemi diversi. In basso sono mostrati i grafi proiettati: \emph{projection \(U\)} (a sinistra), dove due vertici di \(U\) sono adiacenti se condividono almeno un vicino in \(V\), e \emph{projection \(V\)} (a destra), definita simmetricamente.}
    \label{fig:grafo_bipartito}
\end{figure}


Da un \emph{grafo bipartito} si possono derivare due \emph{grafi proiettati}, che sono grafi non bipartiti ottenuti collegando i vertici di uno degli insiemi \( U \) o \( V \) se condividono un vicino nell'altro insieme.

\paragraph{Generalizzazione.} I grafi bipartiti possono essere generalizzati a \emph{grafi multipartiti}, dove i vertici sono divisi in più di due insiemi disgiunti, e gli archi collegano solo vertici appartenenti a insiemi diversi.

\section{Grafo completo vs Grafo regolare}
Un grafo completo è un grafo in cui ogni coppia di vertici distinti è collegata da un arco. In altre parole, in un grafo completo con \( n \) vertici, ogni vertice ha un arco che lo collega a tutti gli altri \( n-1 \) vertici. Un grafo completo con \( n \) vertici è denotato come \( K_n \).

Un grafo regolare è un grafo in cui ogni vertice ha lo stesso grado. In altre parole, in un grafo regolare con \( n \) vertici, ogni vertice ha esattamente \( k \) archi collegati ad esso, dove \( k \) è un numero fisso. Un grafo regolare con \( n \) vertici e grado \( k \) è denotato come \( R(n, k) \).

\begin{figure}[htbp]
    \centering
    \includegraphics[width=\textwidth]{images/grafo_regolare_completo.png}
    \caption{A sinistra: esempi di grafi completi \(K_n\) per \(n=2,\dots,7\), in cui ogni coppia di vertici è adiacente. A destra: esempi di grafi \(k\)-regolari (da \(k=1\) a \(k=4\)), in cui ogni vertice ha lo stesso grado \(k\).}
    \label{fig:grafo_regolare}
\end{figure}

\section{Cammini tra due nodi}
Si definisce \textbf{cammino} tra due nodi \( u \) e \( v \) in un grafo come una sequenza di vertici e archi che collega \( u \) a \( v \). Un cammino può essere rappresentato come una sequenza di vertici \( (u = v_0, v_1, v_2, \ldots, v_k = v) \) tale che ogni coppia di vertici consecutivi \( (v_i, v_{i+1}) \) è collegata da un arco nel grafo.

\begin{figure}[htbp]
    \centering
    \includegraphics[width=0.6\textwidth]{images/cammino.png}
    \caption{Esempio di cammino in un grafo: gli archi verdi definiscono la struttura, mentre in arancione è evidenziato un \emph{cammino} che attraversa i vertici \(1 \to 2 \to 5 \to 7\).}
    \label{fig:cammino_grafo}
\end{figure}

\subsection{Cammino minimo} Si definisce \textbf{cammino minimo} tra due nodi \( u \) e \( v \) come il cammino che collega \( u \) a \( v \) con il minor numero di archi possibile. In un grafo pesato, il cammino minimo può essere definito come il cammino che minimizza la somma dei pesi degli archi attraversati. Nella figura \ref{fig:cammino_grafo}, il cammino minimo tra i nodi \(1\) e \(7\) è \(1 \to 2 \to 5 \to 7\), che attraversa tre archi, quindi $d(1, 7) = 3$.

\subsection{Diametro} Il \textbf{diametro} di un grafo è la massima distanza tra tutte le coppie di vertici nel grafo. In altre parole, è la lunghezza del cammino minimo più lungo tra qualsiasi coppia di vertici nel grafo. Il diametro fornisce una misura della "grandezza" del grafo in termini di distanza tra i suoi vertici. Nel caso della figura \ref{fig:cammino_grafo}, il diametro del grafo è \(4\), che corrisponde alla distanza massima tra le coppie di nodi \( (1, 6) \) e \( (3, 6) \).

\subsection{Ciclo}
Un particolare tipo di cammino è il \textbf{ciclo}, che inizia e termina nello stesso vertice senza ripetere alcun altro vertice lungo il percorso. Un ciclo può essere rappresentato come una sequenza di vertici \( (v_0, v_1, v_2, \ldots, v_k, v_0) \) tale che ogni coppia di vertici consecutivi \( (v_i, v_{i+1}) \) è collegata da un arco nel grafo, e \( v_0 = v_k \). Nell'immagine \ref{fig:cammino_grafo}, un esempio di ciclo è \(2 \to 5 \to 4 \to 2\).

\paragraph{Cappio.} Un particolare tipo di ciclo è il \textbf{cappio}, che è un arco che collega un vertice a se stesso. In altre parole, un cappio è un ciclo di lunghezza 1.

\section{Connettività}
Due nodi $i, j$ di un grafo si dicono \textbf{connessi} se esiste almeno un cammino che li collega. Un grafo si dice \textbf{connesso} se ogni coppia di nodi del grafo è connessa, al contrario si dice \textbf{disconnesso}.

Un grafo disconnesso $G$ risulta formato dall'unione di più sottografi connessi, detti \textbf{componenti connesse} di $G$.

\subsection{Connettività forte e debole}
In un grafo diretto, si distinguono due tipi di connettività:
\begin{description}
    \item[Connettività forte] Due nodi \( u \) e \( v \) sono fortemente connessi se esiste un cammino diretto da \( u \) a \( v \) e un cammino diretto da \( v \) a \( u \). Un grafo diretto è fortemente connesso se ogni coppia di nodi è fortemente connessa.
    \item[Connettività debole] Due nodi \( u \) e \( v \) sono debolmente connessi se esiste un cammino diretto da \( u \) a \( v \) o un cammino diretto da \( v \) a \( u \). Un grafo diretto è debolmente connesso se ogni coppia di nodi è debolmente connessa.
\end{description}

\section{Coefficiente di Clustering}
Il \textbf{coefficiente di clustering} $C_n$ di un nodo $n$ è una misura di quanto gli adiacenti di $n$ siano \textbf{connessi tra loro}. Misura la \emph{densit\`a locale} della rete attorno a $n$: pi\`u il vicinato di $n$ \`e densamente connesso, pi\`u alto \`e il coefficiente.

Formalmente, se $k_n$ \`e il grado di $n$ (numero di vicini) e $L_n$ \`e il numero di archi effettivamente presenti tra i $k_n$ vicini, il coefficiente di clustering locale si definisce come:
\[
C_n \,=\, \frac{2\,L_n}{k_n\,(k_n-1)}
\]
La formula normalizza il numero di archi esistenti rispetto al numero massimo possibile di archi tra i $k_n$ vicini, cos\`i che $C_n\in[0,1]$.

Esempi visuali: il primo caso mostra un vicinato completamente connesso ($C_i=1$), il secondo un vicinato parzialmente connesso ($C_i=1/2$) e il terzo un vicinato senza archi tra vicini ($C_i=0$).

\begin{figure}[htbp]
    \centering
    \includegraphics[width=\textwidth]{images/clustering_coeff.png}
    \caption{Esempio del coefficiente di clustering locale per un nodo centrale (viola) con quattro vicini (arancioni). A sinistra il vicinato è completamente connesso, quindi \(C_i=1\); al centro solo metà delle possibili connessioni tra i vicini è presente (\(C_i=\tfrac{1}{2}\)); a destra non ci sono archi tra i vicini e il coefficiente è nullo (\(C_i=0\)).}
    \label{fig:clustering-coeff}
\end{figure}

\paragraph{Clustering medio.}
Il \textbf{clustering medio} $\langle C \rangle$ di un grafo si ottiene calcolando la media aritmetica dei coefficienti di clustering locali di tutti i nodi del grafo:
\[
\langle C \rangle \,=\, \frac{1}{N} \sum_{n=1}^{N} C_n
\]
dove $N$ è il numero totale di nodi nel grafo. Il clustering medio fornisce una misura globale della tendenza dei nodi a formare gruppi o comunità all'interno del grafo.

\paragraph{Coefficiente di clustering globale.} Un'altra misura del clustering in un grafo è il \emph{coefficiente di clustering globale} $\phi$, che si basa sul conteggio dei triangoli e delle triplette nel grafo. Un triangolo è una tripla di nodi tutti connessi tra loro, mentre una tripletta è una sequenza di tre nodi collegati da due archi. 

Un triangolo però, si può esprimere in 3 modi diversi in base all'orientamento: ad esempio, i nodi \(A\), \(B\) e \(C\) formano un triangolo, ma si possono contare le triplette \( (A, B, C) \), \( (B, C, A) \) e \( (C, A, B) \). 

\section{Misure di centralità}
La \textbf{centralità} di un nodo in un grafo è una misura dell'importanza di un nodo nella rete. Esistono diverse misure di centralità a seconda del criterio di misura.

\subsection{Centralità di grado}
La \textbf{centralità di grado} \( C_G(v) \) di un nodo \( v \) è definita come il numero di archi incidenti su \( v \). In un grafo indiretto, la centralità di grado è semplicemente il grado del nodo:
\[C_G(v) = \text{grado}(v) \]
In un grafo diretto , si può distinguere tra centralità di grado entrante e centrale di grado uscente:
\[C_G^{in}(v) = \text{grado entrante}(v) \]
\[C_G^{out}(v) = \text{grado uscente}(v) \] 

\subsection{Centralità di vicinanza}
La \textbf{centralità di vicinanza} \( C_C(v) \) di un nodo \( v \) è definita come il numero di cammini minimi che passano per \( v \) tra tutte le coppie di nodi nel grafo:
\[C_C(v) = \sum_{s \neq v \neq t} \frac{\delta_{ij}(v)}{\delta_{ij}} \]
dove \( \delta_{ij} \) è il numero di cammini minimi tra i nodi \( i \) e \( j \), e \( \delta_{ij}(v) \) è il numero di tali cammini che passano per \( v \). Un nodo con alta centralità di vicinanza agisce come un ponte critico nella rete, facilitando la comunicazione tra altri nodi.

\begin{figure}[htbp]
    \centering
    \includegraphics[width=0.4\textwidth]{images/betweenness_centrality.png}
    \caption{Esempio di calcolo della centralità di vicinanza per i nodi in un grafo. I numeri accanto ai nodi indicano le distanze minime da ciascun nodo agli altri nodi nel grafo.}
    \label{fig:centralita_vicinanza}
\end{figure}

\paragraph{Esempio (nodo \(3\)).}
Consideriamo ora il calcolo della \textbf{centralità di vicinanza} per il nodo \(3\). Questa misura quantifica quante volte il nodo \(3\) si trova sui cammini minimi tra coppie di altri nodi del grafo.

Nella tabella seguente riportiamo tutte le coppie di nodi \((i,j)\) che non includono il nodo \(3\), insieme ai valori di \(\delta_{ij}(3)\), \(\delta_{ij}\) e del loro rapporto.

\[
\begin{array}{c|c|c|c}
\text{Coppia } (i,j) & \delta_{ij}(3) & \delta_{ij} & \frac{\delta_{ij}(3)}{\delta_{ij}} \\ \hline
(1,2) & 0 & 1 & 0 \\ 
(1,4) & 1 & 1 & 1 \\
(1,5) & 1 & 2 & 0.5 \\
(1,6) & 1 & 2 & 0.5 \\
(1,7) & 1 & 2 & 0.5 \\ \hline
(2,4) & 1 & 1 & 1 \\
(2,5) & 0 & 1 & 0 \\
(2,6) & 0 & 1 & 0 \\
(2,7) & 0 & 1 & 0 \\ \hline
(4,5) & 1 & 1 & 1 \\
(4,6) & 1 & 1 & 1 \\
(4,7) & 1 & 1 & 1 \\ \hline
(5,6) & 0 & 1 & 0 \\
(5,7) & 0 & 1 & 0 \\
(6,7) & 0 & 1 & 0 \\ \hline
\text{Totale} & - & - & 6.5
\end{array}
\]

La somma dei rapporti fornisce il valore complessivo della centralità di intermediazione per il nodo \(3\):
\[
C_B(3) = 6.5
\]

\subsection{Centralità di prossimità}
La \textbf{centralità di prossimità} \( C_P(v) \) di un nodo \( v \) è definita come l'inverso della somma delle distanze minime da \( v \) a tutti gli altri nodi nel grafo:
\[C_P(v) = \frac{1}{\sum_{u \neq v} d(v, u)} \]
dove \( d(v, u) \) è la distanza minima tra i nodi \( v \) e \( u \). Un nodo con alta centralità di prossimità è in grado di raggiungere rapidamente tutti gli altri nodi nel grafo.

Il grafo in figura \ref{fig:centralita_vicinanza} può essere riscritto con i valori di centralità di prossimità, come mostrato in figura \ref{fig:centralita_prossimita}.

\begin{figure}[htbp]
    \centering
    \includegraphics[width=0.4\textwidth]{images/centralita_prossimita.png}
    \caption{Esempio di calcolo della centralità di prossimità per i nodi in un grafo. I numeri accanto ai nodi indicano i valori di centralità di prossimità calcolati in base alle distanze minime da ciascun nodo agli altri nodi nel grafo.}
    \label{fig:centralita_prossimita}
\end{figure}

\subsection{Centralità di PageRank}
La \textbf{centralità di PageRank} \( C_{PR}(v) \) di un nodo \( v \) è una misura dell'importanza di un nodo basata sulla struttura del grafo e sulle connessioni tra i nodi. PageRank è stato originariamente sviluppato da Google per valutare l'importanza delle pagine web, ma può essere applicato a qualsiasi grafo.

La formula di PageRank è data da:
\[C_{PR}(v) = \frac{1 - d}{N} + d \sum_{u \in \text{In}(v)} \frac{C_{PR}(u)}{\text{OutDegree}(u)} \]
dove:
\begin{itemize}
    \item \( d \) è il fattore di damping, solitamente impostato a 0.85.
    \item \( N \) è il numero totale di nodi nel grafo.
    \item \( \text{In}(v) \) è l'insieme dei nodi che puntano a \( v \).
    \item \( \text{OutDegree}(u) \) è il grado uscente del nodo \( u \).
\end{itemize}

\paragraph{Esempio.}
Consideriamo una \textbf{rete diretta} costituita da quattro nodi \( A, B, C, D \).  
Ogni nodo parte con uno \emph{score iniziale} uguale per tutti, pari a \(0.25\), poiché lo score complessivo della rete è \(1\).  
La figura~\ref{fig:pagerank_initial} mostra la configurazione iniziale del grafo, in cui ogni nodo ha lo stesso valore di PageRank.

\begin{figure}[htbp]
    \centering
    \includegraphics[width=0.45\textwidth]{images/pagerank_initial.png}
    \caption{Configurazione iniziale della rete diretta con quattro nodi \(A, B, C, D\), ciascuno con uno score iniziale di \(0.25\).}
    \label{fig:pagerank_initial}
\end{figure}

In ogni iterazione, ciascun nodo cede il proprio score in \textbf{parti uguali} tra i nodi verso i quali ha archi uscenti.  
Nel primo round le distribuzioni sono le seguenti:
\begin{itemize}
    \item \(A\) cede metà score a \(B\) (0.125) e metà a \(C\) (0.125);
    \item \(B\) cede tutto il proprio score a \(C\) (0.25);
    \item \(C\) cede tutto il proprio score a \(D\) (0.25);
    \item \(D\) cede metà score a \(A\) (0.125) e metà a \(B\) (0.125).
\end{itemize}

Dopo il primo round si ottengono i seguenti valori di PageRank: \( A = 0.125, \quad B = 0.25, \quad C = 0.375, \quad D = 0.25 \)

Iterando questo processo di redistribuzione, i valori dei nodi si aggiornano progressivamente fino a raggiungere una situazione di \textbf{equilibrio}, in cui i punteggi non variano più in modo significativo:

\begin{figure}[htbp]
    \centering
    \includegraphics[width=0.7\textwidth]{images/pagerank_rounds.png}
    \caption{Evoluzione dei valori di PageRank per i nodi \(A, B, C, D\) nei round successivi.}
    \label{fig:pagerank_rounds}
\end{figure}

Al termine delle iterazioni, i valori di equilibrio rappresentano la \textbf{centralità di PageRank} dei nodi, indicando la loro importanza relativa all'interno della rete.

\section*{Riferimenti}
I riferimenti di questo capitolo possono essere trovati al capitolo 1 e 2 del libro \cite{Barabasi2016NetworkScience}.

\part{Tecniche di Data Mining}
\chapter{Introduzione al Data Mining}\label{ch:introduzione}

\section{Definizione e finalità}\label{sec:definizione}
Il \emph{Data Mining} consiste nello scoprire \emph{pattern} (modelli e regolarità) interessanti e possibilmente inattesi all'interno di un insieme di dati. Le conoscenze estratte possono essere impiegate per supportare decisioni, formulare previsioni o fungere da base per ulteriori attività (ad es.\ profilazione di utenti).
%
\begin{itemize}
  \item \textbf{Data cleaning (pre-processing)}: individuare e rimuovere artefatti e dati fittizi o rumorosi, armonizzare formati, gestire valori mancanti.
  \item \textbf{Visualizzazione}: comunicare in modo efficace i risultati del processo di data mining.
\end{itemize}

\section{Caratteristiche dei pattern}\label{sec:caratteristiche-pattern}
I pattern da estrarre dovrebbero essere:
\begin{itemize}
  \item \textbf{Validi}: veri (con alta probabilità) anche su dati nuovi non visti.
  \item \textbf{Utili}: capaci di suggerire o guidare azioni concrete.
  \item \textbf{Inattesi}: non banali, non ovvi.
  \item \textbf{Comprensibili}: interpretabili da esseri umani.
\end{itemize}

\section{Metodi di data mining}\label{sec:metodi}
Gli algoritmi di data mining si distinguono in:
\begin{itemize}
  \item \textbf{Descrittivi}: mirano a rappresentare e \emph{descrivere} la struttura dei dati (es.\ clustering, regole di associazione, analisi di similarità).
  \item \textbf{Predittivi}: usano alcune variabili per \emph{predire} valori sconosciuti o futuri (es.\ classificazione, regressione, sistemi di raccomandazione).
\end{itemize}

\section{Perché fare data mining}\label{sec:motivazioni}
Negli ultimi anni la quantità di dati disponibili è esplosa. Le principali sorgenti includono:
\begin{itemize}
  \item \textbf{Business}: web, e-commerce, transazioni, mercati finanziari, log applicativi.
  \item \textbf{Multimedia}: video, audio, testo, immagini.
  \item \textbf{Scienza}: telerilevamento, medicina, bioinformatica.
  \item \textbf{Società}: news, e-mail, social network, forum.
\end{itemize}

\subsection{Big Data}\label{subsec:bigdata}
I dati moderni sono spesso:
\begin{itemize}
  \item \textbf{Grandi} (\emph{volume});
  \item \textbf{Ad elevata dimensionalità} (\emph{varietà} di attributi);
  \item \textbf{Complessi} (relazionali, temporali, eterogenei).
\end{itemize}
La sola disponibilità di molti dati non si traduce automaticamente in conoscenza: servono metodi e strumenti per analizzarli in modo efficace.

\subsection{Dai dati alla conoscenza e alle comunità coinvolte}\label{subsec:comunita}
L'enorme quantità di dati non diventa automaticamente conoscenza: occorre trasformarla con tecniche appropriate. Il data mining è al crocevia di più comunità scientifiche (apprendimento automatico, pattern recognition, visualizzazione, algoritmi, \ldots).

\section{Limiti e insidie del data mining}\label{sec:limiti}
Un'idea intuitiva (ma pericolosa) è: ``raccogliamo quanti più dati possibile e troveremo pattern affidabili''. In realtà, al crescere della dimensione aumenta anche la probabilità di osservare regolarità \emph{spurie}, cioè non realmente significative. Consideriamo un caso di studio.

\subsection{Caso di studio: Total Information Awareness (TIA)}\label{subsec:tia}
Dopo gli attentati dell'11 settembre 2001, il Dipartimento della Difesa degli Stati Uniti propose il programma \emph{Total Information Awareness} (TIA), volto a raccogliere in modo massivo informazioni su persone (ricevute di pagamento, spostamenti, ecc.) per prevenire attacchi terroristici. Al di là delle criticità etiche e di privacy, un rischio metodologico è la generazione di moltissimi \emph{falsi positivi}: attività apparentemente anomale ma statisticamente spiegabili.

\subsection{Esempio: co-presenza in hotel come criterio di sospetto}\label{subsec:esempio-hotel}
Vogliamo identificare potenziali coppie di malfattori assumendo che essi si riuniscano periodicamente nello stesso hotel. Sui dati osservati cerchiamo tutte le coppie di persone che, in \emph{due} giorni distinti, risultano nello \emph{stesso} hotel.
%
\paragraph{Dati di partenza.}
\begin{itemize}
  \item Numero di persone tracciate: $N = 10^9$.
  \item Orizzonte temporale: $D = 1000$ giorni.
  \item Numero di hotel: $H = 10^5$.
  \item Capacità per hotel e per giorno: $C = 100$ persone.
\end{itemize}
\paragraph{Ipotesi nulla (random).} Ogni persona, in ciascun giorno, sceglie in modo casuale e indipendente se (e dove) soggiornare; in particolare, per un dato hotel in un dato giorno la probabilità che una persona vi alloggi è
\[
  P(\text{persona in un hotel specifico in un giorno}) \;=\; \frac{C}{N} \;=\; \frac{100}{10^9} \;=\; 10^{-7}.
\]
\paragraph{Calcoli numerici.}
\begin{enumerate}
  \item \textbf{Stesso hotel in un giorno fissato.} Per due persone specifiche $p,q$, la probabilità di trovarle nello stesso hotel in un \emph{giorno specifico} è
  \[
    P_1 \;=\; H \cdot \left(\frac{C}{N}\right)^2 \;=\; 10^5 \cdot (10^{-7})^2 \;=\; 10^{-9}.
  \]
  \item \textbf{Due giorni distinti non specificati.} Le coppie di giorni distinti sono $\binom{D}{2} = \frac{D(D-1)}{2} \approx \frac{1000\cdot 999}{2} \approx 5\cdot 10^5$. Supponendo indipendenza tra i giorni, la probabilità che $p,q$ si trovino nello stesso hotel in \emph{entrambi} i due giorni è
  \[
    P_2 \;=\; \binom{D}{2}\cdot P_1^2 \;=\; \left(5\cdot 10^5\right)\cdot (10^{-9})^2 \;=\; 5\cdot 10^{-13}.
  \]
  \item \textbf{Numero atteso di coppie sospette.} Le coppie distinte di persone sono $\binom{N}{2}\approx \frac{10^9(10^9-1)}{2}\approx 5\cdot 10^{17}$. Il numero atteso di coppie candidate è dunque
  \[
    \mathbb{E}[\#\text{coppie}] \;=\; P_2 \cdot \binom{N}{2} \;\approx\; \left(5\cdot 10^{-13}\right)\cdot \left(5\cdot 10^{17}\right) \;=\; 2{,}5\cdot 10^{5}.
  \]
\end{enumerate}

\paragraph{Considerazioni.} Verificare manualmente $\sim 250{,}000$ coppie è impraticabile, specie a fronte di un numero reale di coppie colpevoli verosimilmente molto più basso. L'esempio mostra come, su dati enormi, criteri apparentemente sensati possano generare moltissimi falsi positivi \emph{anche in assenza di segnale}.

\section{Principio di Bonferroni e test multipli}\label{sec:bonferroni}
\textbf{Principio di Bonferroni.} Se il numero atteso di occorrenze dell'evento cercato (sotto l'ipotesi di casualità dei dati) è significativamente \emph{maggiore} del numero di istanze che ci si aspetta di trovare nella realtà, allora qualsiasi ``pattern'' osservato è più verosimilmente un \emph{artefatto} che non un'evidenza.
%
\paragraph{Interpretazione operativa.} Quando formuliamo molte ipotesi/ricerche sui dati (\emph{multiple testing}), è necessario \emph{correggere} il livello di significatività per tenere sotto controllo i falsi positivi. Una correzione conservativa è la \emph{correzione di Bonferroni}: se eseguiamo $m$ test, imponiamo che il $p$--value di ciascun test sia $< \alpha/m$ per mantenere il Family-Wise Error Rate al di sotto di $\alpha$.
%
\paragraph{Quando applicarlo.} In scenari esplorativi su grandi basi dati (come nel caso sopra), prima di agire sui ``pattern'' trovati occorre verificare che il loro numero sia compatibile con quanto ci si aspetterebbe per puro caso. In caso contrario, i pattern vanno trattati con sospetto e sottoposti a verifica indipendente (es.\ dati di conferma, A/B test, validazione su hold-out).

% --- Fine capitolo 1 ---
\chapter{Preprocessing}
Il preprocessing dei dati è una fase cruciale nel flusso di lavoro di qualsiasi progetto che sfrutta un dataset. Questa fase include una serie di operazioni volte a migliorare la qualità dei dati, rendendoli più adatti per l'analisi e la modellazione. In questo capitolo, esploreremo le principali tecniche di preprocessing utilizzate nel contesto del dataset in esame.

Generalmente una fase di preprocessing dei dati dipende dalle caratteristiche specifiche del dataset e dagli obiettivi dell'analisi. Tuttavia, alcune operazioni comuni includono:
\begin{description}
  \item[Estrazione di feature] - Questa parte del preprocessing coinvolge l'identificazione e l'estrazione delle caratteristiche più rilevanti dai dati grezzi. Queste feature possono essere utilizzate come input per modelli di machine learning o altre analisi statistiche. 
  \item[Portabilità dei dati] - Assicurarsi che i dati siano in un formato compatibile con gli strumenti e le librerie utilizzate per l'analisi. Questo può includere la conversione di formati di file, la normalizzazione delle strutture dei dati e l'adeguamento delle codifiche.
  \item[Cleaning dei dati] - Questa fase include la gestione dei valori mancanti, la rimozione di outlier e la correzione di errori nei dati. Un dataset pulito è essenziale per garantire risultati affidabili nelle fasi successive dell'analisi.
  \item[Riduzione dei dati] - In alcuni casi, può essere utile ridurre la dimensionalità del dataset o selezionare un sottoinsieme di dati per migliorare l'efficienza computazionale e ridurre il rumore nei dati. 
\end{description}

\section{Estrazione di feature}
L'estrazione di feature consiste nel creare un \textbf{set di feature} più adatte al problema da risolvere rispetto ai dati grezzi. Ipotizziamo di avere un dataset di immagini, le feature grezze generalmente sono i pixel dell'immagine, ma sono poco utili per un'analisi più approfondita. In questo caso, potremmo estrarre feature come bordi, texture o forme presenti nell'immagine, che possono essere più informative per un modello di machine learning.

\subsection{Tecniche di estrazione di feature}
Esistono diverse tecniche per l'estrazione di feature, tra cui:
\begin{itemize}
  \item \textbf{Feature basate su statistiche} - Calcolo di statistiche descrittive come media, varianza, skewness e kurtosis.
  \item \textbf{Feature basate su trasformazioni} - Utilizzo di trasformazioni matematiche come la Trasformata di Fourier o la Trasformata Wavelet per estrarre informazioni frequenziali.
  \item \textbf{Feature basate su modelli} - Applicazione di modelli predefiniti per estrarre feature, come l'uso di reti neurali convoluzionali per l'estrazione di feature da immagini.
\end{itemize}

\section{Portabilità dei dati}
In questo caso il problema diventa quello di convertire i dati in un formato che sia facilmente utilizzabile dagli strumenti di analisi. Ad esempio, se i dati sono in un formato proprietario, potrebbe essere necessario convertirli in un formato standard come CSV o JSON. Inoltre i dati sono generalmente salvati in modo \textbf{eterogeneo} (valori numerici oppure categorici) e quindi è necessario uniformarli per poterli utilizzare in modo efficace.

\subsection{Da dati numerici a categorici}
La conversione da dati \emph{numerici} a \emph{categorici} è detta \textbf{discretizzazione}. Questa tecnica consiste nel suddividere l'intervallo dei valori numerici in un numero finito di intervalli, assegnando a ciascun intervallo una categoria specifica. Ad esempio, i valori di età possono essere suddivisi in categorie come "giovane", "adulto" e "anziano".

\paragraph{Equi-width ranges.}
In questo caso, l'intervallo dei valori numerici viene suddiviso in intervalli di larghezza uguale. Ad esempio, se i valori variano da $0$ a $100$ e si desidera creare 5 categorie, ogni intervallo avrà una larghezza di $20$ $(0-19, 20-39, 40-59, 60-79, 80-100)$.

\paragraph{Equi-log ranges.}
Questa tecnica invece, si costruisce ogni intervallo $[a, b]$ in modo tale che la scala logaritmica sia uniforme, ovvero $\log_x a - \log_x b$ è costante e ha sempre lo stesso valore per ogni intervallo. Questa tecnica è particolarmente utile quando i dati coprono un ampio intervallo di valori e si desidera preservare le proporzioni relative tra i valori. Ad esempio , se i valori variano da $1$ a $1000$ e si desidera creare 3 categorie, gli intervalli potrebbero essere $[1, 10]$, $[11, 100]$ e $[101, 1000]$ (per un logaritmo in base $10$).

\paragraph{Equi-depth ranges.}
In questo caso, gli intervalli vengono creati in modo tale che ogni intervallo contenga lo stesso numero di istanze. Ad esempio, se si hanno $100$ istanze e si desidera creare $5$ categorie, ogni intervallo conterrà $20$ istanze. Questa tecnica è utile quando si desidera bilanciare la distribuzione delle categorie.

\subsection{Da dati categorici a numerici}
La conversione da dati \emph{categorici} a \emph{numerici} è detta \textbf{codifica}. Questa tecnica consiste nell'assegnare un valore numerico a ciascuna categoria. Ad esempio, le categorie "rosso", "verde" e "blu" possono essere codificate come $1$, $2$ e $3$ rispettivamente. Questo si fa perché alcuni modelli di machine learning o algoritmi di analisi richiedono input numerici. 

\paragraph{One-hot encoding.}
Questo schema di codifica consiste nel creare una nuova variabile binaria per ogni categoria. Quindi si ottiene un vettore di lunghezza pari al numero di categorie, in cui solo la posizione corrispondente alla categoria attuale è impostata a $1$, mentre tutte le altre sono impostate a $0$. Ad esempio, per le categorie "rosso", "verde" e "blu", la codifica one-hot sarebbe:
\begin{itemize}
  \item Rosso: [1, 0, 0]
  \item Verde: [0, 1, 0]
  \item Blu: [0, 0, 1]
\end{itemize} 

\subsection{Da testo a dati numerici}
La conversione da \emph{testo} a \emph{dati numerici} è detta \textbf{vettorizzazione}. Questa tecnica consiste nel rappresentare il testo come un vettore di numeri, in modo che possa essere utilizzato come input per modelli di machine learning o altre analisi statistiche. Ad esempio, una frase come "Il gatto è sul tappeto" può essere rappresentata come un vettore di numeri che rappresentano la frequenza delle parole nella frase.

\section{Cleaning dei dati}
Il cleaning dei dati è una fase essenziale del preprocessing che mira a migliorare la qualità dei dati rimuovendo errori, gestendo valori mancanti e trattando outlier. Un dataset pulito è fondamentale per garantire risultati affidabili nelle fasi successive dell'analisi.

\subsection{Gestione dei valori mancanti}
I valori mancanti possono verificarsi per vari motivi, come errori di raccolta dati o problemi di trasmissione. Ci sono diverse strategie per gestire i valori mancanti:
\begin{description}
  \item[Rimozione delle istanze] - Eliminare le righe o le colonne che contengono valori mancanti. Questa strategia è semplice ma può portare alla perdita di informazioni importanti, tuttavia funziona bene quando la quantità di dati mancanti è minima.
  \item[Stima dei valori mancanti] - Utilizzare tecniche di imputazione\footnote{Per imputazione si intende la sostituzione dei valori mancanti con valori stimati basati su altre informazioni presenti nel dataset.} per stimare i valori mancanti basandosi sui dati disponibili. Alcuni metodi comuni includono la media, la mediana o la moda per variabili numeriche, e il valore più frequente per variabili categoriche. Metodi più avanzati includono l'uso di modelli di regressione o algoritmi di machine learning per prevedere i valori mancanti.
\end{description}

\subsection{Gestione dei valori errati}
I valori errati possono derivare da errori di inserimento dati, misurazioni imprecise o problemi di trasmissione. Per gestire i valori errati, è possibile:
\begin{description}
  \item[Rilevamento di inconsistenze] - Identificare valori che non rientrano nell'intervallo previsto o che sono logicamente incoerenti con altri dati. Ad esempio, un'età negativa o una data di nascita futura.
  \item[Correzione dei valori errati] - Sostituire i valori errati con valori stimati o medi basati su altre informazioni nel dataset. In alcuni casi, potrebbe essere necessario consultare esperti del dominio per determinare il valore corretto. Per esempio, se abbiamo dati di ristoranti negli Stati Uniti e troviamo un ristorante nella città "Roma", è probabile che si tratti di un errore.
\end{description}

Quando si parla di valori errati si parla anche di \textbf{outlier}, ovvero valori che si discostano significativamente dalla maggior parte dei dati. Gli outlier possono essere il risultato di errori di misurazione o possono rappresentare fenomeni rari ma validi.Per esempio, in un dataset di altezze umane, un valore di $250$ cm potrebbe essere considerato un outlier. Esistono diversi metodi per eliminare gli outlier, come l'uso di tecniche statistiche.

\paragraph{Quantili.}
Un quantile è un valore che divide un insieme di dati ordinati in intervalli con una certa percentuale di dati in ciascun intervallo. Ad esempio, il primo quartile (Q1) è il valore che separa il $25\%$ inferiore dei dati dal resto del dataset, mentre il terzo quartile (Q3) separa il $75\%$ inferiore dal $25\%$ superiore. Un valore $x$ potrebbe essere definito come \emph{outlier} se non appartenente all'intervallo:
\[
[Q1 - 1.5 \cdot IQR, Q3 + 1.5 \cdot IQR]
\]
dove $IQR$ (Interquartile Range) è la differenza tra il terzo e il primo quartile ($IQR = Q3 - Q1$).

Per esempio , se in un dataset il primo quartile è $10$ e il terzo quartile è $20$, l'intervallo per identificare gli outlier sarebbe:
\[[10 - 1.5 \cdot (20 - 10), 20 + 1.5 \cdot (20 - 10)] = [-5, 35]\]
Quindi, qualsiasi valore al di fuori di questo intervallo sarebbe considerato un outlier.

\subsection{Scala dei dati}
La scala dei dati si riferisce all'intervallo di valori che una variabile può assumere. In alcuni casi, le variabili possono avere scale molto diverse, il che può influenzare negativamente le prestazioni di alcuni algoritmi. Si può pensare ad esempio a due variabili: altezza (in centimetri) e peso (in chilogrammi). L'altezza può variare da $150$ a $200$ cm, mentre il peso può variare da $50$ a $150$ kg. Se si utilizzano queste variabili senza alcuna normalizzazione, l'algoritmo potrebbe dare più importanza alla variabile con la scala più ampia (in questo caso, l'altezza).

\paragraph{Standardizzazione.}
La standardizzazione è una tecnica che trasforma i dati in modo che abbiano una media di $0$ e una deviazione standard di $1$. La formula per standardizzare un valore $x$ è:
\[
z = \frac{x - \mu}{\sigma}
\]
dove $\mu$ è la media dei dati e $\sigma$ è la deviazione standard:
\[
\mu = \frac{1}{N} \sum_{i=1}^{N} x_i, \quad \sigma = \sqrt{\frac{1}{N} \sum_{i=1}^{N} (x_i - \mu)^2}
\]

\paragraph{Min-Max scaling.}
Il Min-Max scaling è una tecnica che trasforma i dati in modo che rientrino in un intervallo $[0, 1]$. Viene considerato un metodo meno robusto della standardizzazione, perché è sensibile agli outlier. Dato un valore $x$, la formula per il Min-Max scaling è:
\[
x' = \frac{x - x_{min}}{x_{max} - x_{min}}
\]
dove $x_{min}$ e $x_{max}$ sono rispettivamente il valore minimo e massimo dei dati.

\section{Riduzione dei dati}
La riduzione dei dati consiste nel \emph{rappresentare} i dati in maniera più compatta, in modo da facilitare l'uso di algoritmi di analisi e modellazione. 

\subsection{Sampling dei dati}
Un modo semplice per ridurre la quantità di dati è il \textbf{sampling}, ovvero la selezione di un sottoinsieme rappresentativo del dataset originale. Esistono diverse tecniche di sampling, tra cui:
\begin{description}
  \item[Biased] - Selezione di istanze in base a criteri specifici, come la frequenza di una classe o la rilevanza per un particolare obiettivo di analisi.
  \item[Random] - Selezione casuale di istanze dal dataset originale, garantendo che ogni istanza abbia la stessa probabilità di essere selezionata.
  \item[Stratificato] - Suddivisione del dataset in sottogruppi (strati) basati su una o più caratteristiche, e successiva selezione casuale di istanze da ciascuno strato per garantire una rappresentazione equilibrata delle diverse categorie nel sottoinsieme.
\end{description}

\subsection{Selezione di feature}
La selezione di feature consiste nel scartare dai dati attributi che sono irrilevanti per l'analisi. La rilevanza delle feature \textbf{dipende} dal dominio del problema. Ad esempio, in un dataset di immagini, le feature relative al colore potrebbero essere irrilevanti per un'analisi che si concentra sulla forma degli oggetti presenti nell'immagine. Le tecniche possono essere \textbf{supervised}, ovvero basate su etichette di classe, oppure \textbf{unsupervised}, ovvero basate solo sulle caratteristiche intrinseche dei dati.

\subsection{Riduzione della dimensionalità}
I dati reali spesso contengono molte feature, alcune delle quali possono essere ridondanti o irrilevanti, spesso difficili da notare oppure \emph{implicite}. La riduzione della dimensionalità mira a ridurre il numero di feature mantenendo quante più informazioni possibili.

Una possibile tecnica di riduzione è quella di individuare una rotazione degli assi, ovvero una nuova base, in cui i dati possano essere rappresentati in modo più compatto.Ricordando dall'algebra lineare, una rotazione degli assi è un cambiamento di sistema di coordinate che preserva le distanze e gli angoli tra i punti. In altre parole, i dati vengono proiettati su un nuovo insieme di assi che sono combinazioni lineari delle feature originali.

\subsection{PCA: Principal Component Analysis}
La PCA è una tecnica di riduzione della dimensionalità che identifica le direzioni principali (componenti principali). Per capire quali sono, le componenti principali, dobbiamo partire dalla matrice di \textbf{covarianza} dei dati: dati due vettori $X, Y$ la covarianza misura la \emph{varianza} di $X$ rispetto a $Y$, ovvero quanto variano insieme. Se a più alti valori di $X$ corrispondono più alti valori di $Y$, la covarianza sarà positiva, mentre se a più alti valori di $X$ corrispondono più bassi valori di $Y$, la covarianza sarà negativa. La covarianza tra due variabili $X$ e $Y$ è calcolata come:
\begin{align*}
C(X, Y) 
    &= \frac{1}{N} \sum_{k=1}^{N} (X_k - \mu_X)(Y_k - \mu_Y) \\
&\Rightarrow\;
c_{ij} = \frac{x_i \cdot x_j}{N} - \mu_i \mu_j
\end{align*}
dove $\mu_X$ e $\mu_Y$ sono le medie di $X$ e $Y$, rispettivamente, e $N$ è il numero di osservazioni.

\noindent
La matrice di covarianza $C$ gode di diverse proprietà:
\begin{itemize}
  \item La covarianza di un vettore con se stesso equivale alla \textbf{varianza} del vettore: $C(X, X) = \sigma_X^2$.
  \item La matricce di covarianza è \textbf{simmetrica}, ovvero $C = C^T \Rightarrow C(X, Y) = C(Y, X)$.
  \item Il segno della covarianza indica la direzione della relazione tra le variabili: una covarianza positiva indica che le variabili tendono a variare nella stessa direzione, mentre una covarianza negativa indica che variano in direzioni opposte.
\end{itemize}

Le componenti principali sono le direzioni lungo le quali i dati variano maggiormente. Queste direzioni sono rappresentate dagli \textbf{autovettori} della matrice di covarianza, mentre la quantità di varianza spiegata da ciascuna componente principale è rappresentata dagli \textbf{autovalori} corrispondenti. 

Si consideri un dataset rappresentato da una matrice $D$ con $M$ osservazioni (righe) e $N$ feature (colonne). A partire da $D$ si costruisce la matrice di covarianza, da cui si ricavano le componenti principali, ovvero nuovi attributi ottenuti come combinazioni lineari degli attributi originari. I nuov attributi presentano due proprietà fondamentali:
\begin{itemize}
  \item Sono tra di loro \textbf{non correlate}, ovvero la covarianza tra due componenti principali è zero.
  \item Concentrano la maggior parte della \textbf{varianza} dei dati originali in poche dimensioni.
\end{itemize}

In presenza di forte correlazione tra attriuti, poche componenti principali riescono a spiegare gran parte della varianza totale del dataset. Questo permette di ridurre la dimensionalità del dataset mantenendo la maggior parte delle informazioni rilevanti.

\noindent
I passi principali della PCA sono:
\begin{description}
  \item[Standardizzazione dei dati.] I dati vengono standardizzati per avere media zero e deviazione standard uno. Ciò è necessario perché la PCA è sensibile alla varianza iniziale dei dati nelle varie dimensioni, infatti dimensioni con varianze molto diverse possono dominare la direzione delle componenti principali.
  \item[Calcolo delle componenti principali.] Grazie alla PCA si può decomporre la matrice di covarianza nel prodotto di 3 matrici:
  \[
  C = P \Lambda P^T
  \]
  dove $P$ è la matrice degli autovettori (le componenti principali) e $\Lambda$ è la matrice diagonale degli autovalori (la varianza spiegata da ciascuna componente principale).
  \item[Selezione delle componenti principali.] Le componenti principali determinate sono uguali al numero di feature originali. La prima componente principale, però, \textbf{cattura} la più alta varianza nei dati e da un punto di vista geometrico corrisponde nel trovare la retta dove le proiezioni dei punti hanno la massima varianza. La seconda componente principale è ortogonale alla prima e cattura la seconda più alta varianza, e così via. Si selezionano le prime $k$ componenti principali che spiegano una percentuale significativa della varianza totale (ad esempio, il $95\%$).
  \item[Trasformazione dei dati.] Infine, utilizzando il dataset $D: M \times N$ la matrice $P: N \times k$ ottenuta concatenando per colonna i primi $k$ autovettori selezionati, si ottiene il dataset ridotto $D': M \times k$ tramite la moltiplicazione:
  \[
  D' = D \cdot P
  \]
\end{description}

\subsection{SVD: Singular Value Decomposition}
La Singular Value Decomposition (SVD) è un'altra tecnica di riduzione della dimensionalità che decompone una matrice $M$ di dimensione $n \times d$ nel prodotto:
\[
M = U \Sigma V^T
\]
dove:
\begin{itemize}
  \item $U$ è una matrice ortogonale di dimensione $n \times n$ le cui colonne sono chiamate \emph{left singular vectors}.
  \item $\Sigma$ è una matrice diagonale di dimensione $n \times d$ i cui elementi diagonali sono chiamati \emph{singular values}, ordinati in ordine decrescente.
  \item $V$ è una matrice ortogonale di dimensione $d \times d$ le cui colonne sono chiamate \emph{right singular vectors}.
\end{itemize}

\paragraph{Interpretazione geometrica.}
La SVD può essere interpretata come una rotazione e una scalatura dello spazio dei dati. I \emph{right singular vectors} (colonne di $V$) rappresentano le direzioni principali nello spazio delle feature, mentre i \emph{left singular vectors} (colonne di $U$) rappresentano le direzioni principali nello spazio delle osservazioni. I \emph{singular values} nella matrice $\Sigma$ indicano l'importanza di ciascuna direzione.

Consideriamo il caso in cui $M$ sia una matrice $2 \times 2$ e agisca quindi sul piano $\mathbb{R}^2$. Partiamo da un cerchio unitario con i due vettori canonici. La SVD 
\[
M = U \Sigma V^T
\]
può essere vista come una successione di tre operazioni semplici. La prima trasformazione, data da $V^T$, ruota il cerchio e riallinea gli assi secondo le direzioni individuate dai \emph{right singular vectors}. La matrice diagonale $\Sigma$ applica poi una scalatura lungo tali direzioni, trasformando il cerchio in un'ellisse i cui semiassi hanno lunghezze pari ai valori singolari non nulli. Infine, la matrice $U$ effettua un'ulteriore rotazione, orientando l'ellisse nelle direzioni dei \emph{left singular vectors}. In sintesi, nella SVD la matrice $M$ agisce come una combinazione di rotazioni e scalature che mappa la sfera unitaria in un'ellissoide.

\paragraph{Varianti ridotte della SVD.}
Esistono varianti della SVD che permettono di ridurre il tempo computazionale e lo spazio di memoria necessari per calcolare la decomposizione, specialmente quando la matrice $M$ è di grandi dimensioni o sparsa. La variante principale si chiama \textbf{Full-SVD} ed è la versione completa della SVD descritta sopra, ne esistono tuttavia altre versioni:
\begin{description}
  \item[Think SVD] - In questo approccio si rimuovono le colonne di $U$ e le righe di $\Sigma$ in eccesso rispetto alle colonne della matrice $V$, in modo da assicurare una decomposizione più compatta. La matrice $U$ diventa quindi di dimensione $n \times r$, dove $r$ è il rango della matrice $M$, mentre $\Sigma$ diventa una matrice diagonale di dimensione $r \times r$.
  \item[Compact SVD] - Questa versione rimuove le righe di $\Sigma$ che contengono valori singolari nulli e di conseguenza anche le colonne di $U$ e le righe di $V^T$ in eccesso rispetto al nuovo numero di righe di $\Sigma$. La matrice $U$ diventa quindi di dimensione $n \times k$, dove $k$ è il numero di valori singolari non nulli, mentre $\Sigma$ diventa una matrice diagonale di dimensione $k \times k$.
  \item[Truncated SVD] - In questa variante si selezionano solo i primi $k$ valori singolari più grandi e le corrispondenti colonne di $U$ e righe di $V^T$. Questo approccio è particolarmente utile quando si desidera ridurre la dimensionalità dei dati mantenendo solo le componenti più significative. La matrice $U$ diventa quindi di dimensione $n \times k$, mentre $\Sigma$ diventa una matrice diagonale di dimensione $k \times k$.  
\end{description}

\begin{figure}[htbp]
  \centering
  \includegraphics[width=\textwidth]{images/SVD.png}
  \caption{Confronto tra le principali varianti della Decomposizione ai Valori Singolari (SVD): Full SVD, Compact SVD, Thin SVD e Truncated SVD, con rappresentazione delle dimensioni delle matrici coinvolte.}
  \label{fig:SVD}
\end{figure}

\paragraph{SVD vs PCA.}
La SVD e la PCA sono entrambe tecniche di riduzione della dimensionalità, ma presentano alcune differenze chiave:
\begin{itemize}
  \item SVD è più generale della PCA, poiché produce due set di autovettori anziché uno solo. La PCA può essere vista come un caso speciale della SVD applicata alla matrice di covarianza.
  \item SVD corrisponde alla PCA nel caso in cui i dati sono centrati attorno allo zero, ovvero quando la media dei valori di ogni attributo è zero.
  \item La PCA cattura quanto più varianza possibile nei dati, la SVD cattura quanta più distanza euclidea al quadrato rispetto all'origine possibile.
\end{itemize}

\subsection{LSA: Latent Semantic Analysis}
La Latent Semantic Analysis (LSA) è una tecnica di riduzione della dimensionalità utilizzata principalmente nell'elaborazione del linguaggio naturale e nel recupero delle informazioni. LSA mira a identificare le relazioni semantiche tra parole e documenti, riducendo la dimensionalità dello spazio delle caratteristiche.

La matrice di partenza è una matrice $n \times d$ di $n$ documenti e $d$ termini, contenente le frequenze normalizzate delle parole in ciascun documento. La LSA utilizza la SVD per decomporre questa matrice nei suoi componenti principali, identificando le direzioni principali nello spazio delle caratteristiche che catturano le relazioni semantiche tra parole e documenti.

\subsection{Riduzione di dimensionalità con trasformazione dei dati}
Oltre alle tecniche basate su decomposizioni matriciali, esistono metodi di riduzione della dimensionalità che si basano sulla trasformazione non lineare dei dati. Questi metodi cercano di mappare i dati originali in uno spazio a bassa dimensionalità preservando le relazioni strutturali tra i punti dati.

\paragraph{Esempio: serie temporali.}
Un esempio di riduzione della dimensionalità basata sulla trasformazione dei dati è l'analisi delle serie temporali. Le serie temporali sono sequenze di dati raccolti nel tempo, e spesso presentano una struttura complessa che può essere difficile da analizzare direttamente. Tecniche come la trasformata di Fourier o la trasformata wavelet possono essere utilizzate per rappresentare le serie temporali in uno spazio a bassa dimensionalità, catturando le caratteristiche principali delle variazioni temporali.
\chapter{Insiemi Frequenti e Regole d'Associazione}\label{ch:frequent-itemsets}
% Basato sulle slide del corso ("Mining insiemi frequenti -- Parte 1") e approfondimenti dal libro (Leskovec et al., cap. 6).

\section{Market-basket model e definizioni}\label{sec:mbm}
Nel \emph{market-basket model} ogni transazione (\emph{basket}) è un insieme di oggetti (\emph{item}). L'obiettivo è individuare \textbf{itemset frequenti}, cioè insiemi di item che compaiono assieme in molte transazioni, e derivarne \textbf{regole d'associazione} utili per descrivere co-occorrenze interessanti.

\paragraph{Supporto.} Sia $\mathcal{D}$ l'insieme dei basket (con $|\mathcal{D}|=N$) e sia $I=\{i_1,\dots,i_k\}$ un itemset. Il \textbf{supporto} assoluto di $I$ è
\[
\mathrm{supp}(I) = |\{ T\in\mathcal{D}\,:\, I\subseteq T\}|,\qquad \mathrm{supp}_\mathrm{rel}(I)=\frac{\mathrm{supp}(I)}{N}.
\]
Dato un valore soglia $\sigma$ (\emph{min-sup}), $I$ è detto \textbf{frequente} se $\mathrm{supp}(I)\ge \sigma$.

\paragraph{Soglia di supporto: trade-off.} Una soglia troppo alta può eliminare pattern interessanti; una troppo bassa produce un'esplosione di candidati difficili da analizzare e validare.

\section{Regole d'associazione}\label{sec:assoc}
Una \textbf{regola d'associazione} è un'implicazione $X\to j$, con $X$ itemset e $j$ un singolo item con $j\notin X$. Si estende naturalmente a $X\to Y$ con $X\cap Y=\varnothing$.

\subsection{Qualità di una regola}\label{subsec:qualita-regole}
\paragraph{Confidenza.} Con $\mathrm{supp}(\cdot)$ definito sopra, la confidenza di $X\to j$ è
\[
\mathrm{conf}(X\to j) = \frac{\mathrm{supp}(X\cup\{j\})}{\mathrm{supp}(X)} \;=\; P(j\mid X).\label{eq:confidence}
\]
\paragraph{Coverage.} $\mathrm{supp}(X)$ è detto anche \emph{coverage}: misura quanto spesso è applicabile la regola.
\paragraph{Interesse (o \emph{interest}).} Quantifica l'influenza di $X$ su $j$ come scostamento dalla prevalenza marginale di $j$:
\[
\mathrm{int}(X\to j) = \mathrm{conf}(X\to j) - \frac{\mathrm{supp}(\{j\})}{N}.
\]
\paragraph{Lift.} Confronta la co-occorrenza osservata con quella attesa in caso di indipendenza:
\[
\mathrm{lift}(X\to j) = \frac{N\cdot\mathrm{supp}(X\cup\{j\})}{\mathrm{supp}(X)\,\mathrm{supp}(\{j\})} \;=\; \frac{\mathrm{conf}(X\to j)}{\mathrm{supp}(\{j\})/N}.
\]
Valori $>1$ indicano associazione positiva, $<1$ negativa.

\paragraph{Nota.} Supporto e confidenza alti non implicano necessariamente interesse: regole ovvie (es. {pasta, pomodoro} $\to$ {pasta}) possono essere poco informative.

\subsection{Mini-esempio (dataset giocattolo)}\label{subsec:mini-esempio}
Sia $N=8$ e consideriamo item $\{b,c,j,m,p\}$. Supponiamo che $\mathrm{supp}(\{b\})=6$, $\mathrm{supp}(\{c\})=5$, $\mathrm{supp}(\{j\})=4$, $\mathrm{supp}(\{m\})=5$, $\mathrm{supp}(\{p\})=2$ e, tra le coppie, $\mathrm{supp}(\{b,c\})=4$, $\mathrm{supp}(\{c,j\})=3$, $\mathrm{supp}(\{c,m\})=2$, $\mathrm{supp}(\{m,p\})=2$, ecc. Per la regola $\{c,m\}\to b$ si ha
\[
\mathrm{conf}=\tfrac{\mathrm{supp}(\{b,c,m\})}{\mathrm{supp}(\{c,m\})}=\tfrac{2}{2}=1.0,\quad
\mathrm{lift}=\frac{8\cdot 2}{2\cdot 6}=1.33\,.
\]

\section{Insiemi frequenti chiusi e massimali}\label{sec:closed-maximal}
Sia $I$ frequente.
\begin{itemize}
  \item $I$ è \textbf{chiuso} se nessun suo super-insieme ha lo \emph{stesso} supporto di $I$.
  \item $I$ è \textbf{massimale} se nessun suo super-insieme è frequente.
\end{itemize}
Gli insiemi massimali sono (per definizione) chiusi; gli insiemi chiusi sono un sottoinsieme degli insiemi frequenti e consentono una rappresentazione più compatta senza perdere il supporto degli insiemi chiusi stessi.

\begin{figure}[htbp]
  \centering
  \includegraphics[width=0.8\textwidth]{images/insiemi_frequenti_massimale.png}
  \caption[Itemset lattice (minsup 2)]%
  {Grafo degli itemset con minsup $=2$. Un itemset è \emph{frequente} se il suo supporto è \texorpdfstring{$\ge 2$}{>= 2};
  è \emph{chiuso} se nessun superinsieme ha lo stesso supporto; è \emph{massimale} se nessun superinsieme è frequente.
  Nell'esempio: alcuni chiusi (es. CE) e chiusi–massimali (CE, DE); conteggi indicati: \#closed = 9, \#maximal = 4.}
  \label{fig:ifm}
\end{figure}


\section{Proprietà anti-monotona e Principio di Apriori}\label{sec:apriori-principle}
Per itemset $S\subseteq I$ vale l'\textbf{anti-monotonicità del supporto}:
\[
\mathrm{supp}(I)\le \mathrm{supp}(S).\label{eq:anti-mon}
\]
Da cui il \textbf{Principio di Apriori}: se un itemset $I$ è frequente, ogni suo sottoinsieme è frequente; equivalentemente, se $I$ non è frequente, nessun suo super-insieme può esserlo. Questa proprietà consente un \emph{pruning} efficace dello spazio dei candidati.

\section{Algoritmo Apriori}\label{sec:apriori}
Ricerca bottom-up per cardinalità crescente.
\begin{enumerate}
  \item Calcola l'insieme $L_1$ degli item singoli frequenti.
  \item Per $k=1,2,\dots$:
  \begin{enumerate}
    \item \textbf{Join}: genera $C_{k+1}$ (candidati di taglia $k{+}1$) con self-join di $L_k$.
    \item \textbf{Prune}: elimina da $C_{k+1}$ gli itemset che contengono sottoinsiemi di taglia $k$ non frequenti (per \S\ref{sec:apriori-principle}).
    \item \textbf{Conteggio}: calcola $\mathrm{supp}(\cdot)$ dei candidati scorrendo il DB e costruisci $L_{k+1}=\{c\in C_{k+1}: \mathrm{supp}(c)\ge\sigma\}$.
  \end{enumerate}
  \item Arresta quando $C_{k+1}=\varnothing$.
\end{enumerate}

\subsection{Apriori: esempio (minsup = 2)}\label{subsec:apriori-esempio}
Consideriamo $N=8$ basket e gli item $\{b,c,j,m,p\}$. I supporti degli item singoli sono:
\begin{center}
\begin{tabular}{@{}lccccc@{}}
\toprule
Item & $b$ & $c$ & $j$ & $m$ & $p$ \\
\midrule
$\mathrm{supp}(\cdot)$ & 6 & 5 & 4 & 5 & 2 \\
\bottomrule
\end{tabular}
\end{center}
Con minsup $=2$, tutti e cinque gli item sono frequenti, quindi $L_1=\{b,c,j,m,p\}$. \emph{(Dati come nelle slide)}.

\paragraph{Passo $k=1\to2$: generazione $C_2$ e pruning.}
$C_2$ si ottiene con self-join di $L_1$ e contiene tutte le coppie possibili:
\[
\{b,c\},\{b,j\},\{b,m\},\{b,p\},\{c,j\},\{c,m\},\{c,p\},\{j,m\},\{j,p\},\{m,p\}.
\]
Dai conteggi nel DB (come in tabella delle slide) si ottengono i supporti:
\begin{center}
\begin{tabular}{@{}lcccccccccc@{}}
\toprule
Itemset & $\{b,c\}$ & $\{b,j\}$ & $\{b,m\}$ & $\{b,p\}$ & $\{c,j\}$ & $\{c,m\}$ & $\{c,p\}$ & $\{j,m\}$ & $\{j,p\}$ & $\{m,p\}$ \\
\midrule
$\mathrm{supp}(\cdot)$ & 4 & 2 & 4 & 1 & 3 & 2 & 0 & 2 & 1 & 2 \\
\bottomrule
\end{tabular}
\end{center}
Applicando minsup, otteniamo $L_2=\{\{b,c\},\{b,j\},\{b,m\},\{c,j\},\{c,m\},\{j,m\},\{m,p\}\}$. 

\paragraph{Passo $k=2\to3$: generazione $C_3$ da $L_2$ (self-join) e pruning.}
Si combinano coppie con i primi $k-1$ item uguali (ordine lessicografico) e si rimuovono i candidati che hanno qualche sottoinsieme di taglia 2 non in $L_2$ (Principio di Apriori, \S\ref{sec:apriori-principle}). I candidati che restano sono:
\[
C_3=\{\{b,c,j\},\{b,c,m\},\{b,j,m\},\{c,j,m\}\}.
\]
Conteggiando i supporti sul DB (slide):
\begin{center}
\begin{tabular}{@{}lcccc@{}}
\toprule
Itemset & $\{b,c,j\}$ & $\{b,c,m\}$ & $\{b,j,m\}$ & $\{c,j,m\}$ \\
\midrule
$\mathrm{supp}(\cdot)$ & 2 & 2 & 1 & 1 \\
\bottomrule
\end{tabular}
\end{center}
Quindi $L_3=\{\{b,c,j\},\{b,c,m\}\}$.

\paragraph{Passo $k=3\to4$: generazione $C_4$ e arresto.}
L’unico candidato unibile è $\{b,c,j,m\}$, ma il suo supporto vale $1<2$, dunque non è frequente e $L_4=\varnothing$. L’algoritmo termina.

\paragraph{Riassunto dell’esempio.}
\[
L_1=\{b,c,j,m,p\},\quad
L_2=\{\{b,c\},\{b,j\},\{b,m\},\{c,j\},\{c,m\},\{j,m\},\{m,p\}\},
\]
\[
L_3=\{\{b,c,j\},\{b,c,m\}\},\quad
L_4=\varnothing.
\]
L’anti-monotonicità del supporto permette il \emph{pruning} efficace a ogni livello, riducendo drasticamente i candidati da contare. 


\subsection{Generazione dei candidati}\label{subsec:candidate-gen}
Se gli item sono ordinati, due insiemi $A=(a_1,\dots,a_{k-1},x)$ e $B=(a_1,\dots,a_{k-1},y)$ in $L_k$ con $x<y$ producono il candidato $(a_1,\dots,a_{k-1},x,y)$. Il passo di \emph{prune} scarta i candidati che hanno almeno un sottoinsieme di taglia $k$ non presente in $L_k$.

\paragraph{Esempio (schema).} Da $L_2=\{\{b,c\},\{b,j\},\{b,m\},\{c,j\},\{c,m\},\{j,m\},\{m,p\}\}$ si generano candidati di taglia 3 come $\{b,c,j\}$, $\{b,c,m\}$, $\{b,j,m\}$, $\{c,j,m\}$, ecc., poi si eliminano quelli che contengono coppie non frequenti.

\section{Ottimizzazioni di Apriori}\label{sec:apriori-opt}
\subsection{Hashing in bucket: PCY}\label{subsec:pcy}
Alla prima passata si contano i singoli item e, parallelamente, si proiettano tutte le coppie in bucket tramite una funzione hash. I bucket con supporto sotto soglia vengono marcati come non frequenti: alla seconda passata, una coppia $(i,j)$ è candidata solo se \emph{entrambi} gli item sono frequenti e il bucket hash di $(i,j)$ è frequente. Ciò riduce notevolmente $|C_2|$.

\subsection{Partizionamento del DB: SON}\label{subsec:son}
Divide il dataset in partizioni; su ciascuna partizione si esegue Apriori con min-sup scalato (proporzionale alla frazione di transazioni della partizione). L'unione degli insiemi frequenti locali fornisce i candidati globali, che vengono poi verificati su tutto il DB. L'algoritmo è adatto a calcolo distribuito.

\subsection{Campionamento e frontiera negativa: Toivonen}\label{subsec:toivonen}
Si esegue Apriori su un campione casuale $S$ con soglia più bassa ($\sigma'$); si ottiene un insieme di itemset frequenti in $S$ e la \emph{frontiera negativa}: insiemi non frequenti in $S$ i cui \emph{immediati} sottoinsiemi sono frequenti. Se nessun elemento della frontiera negativa risulta frequente sull'intero DB, i frequenti di $S$ sono la risposta; altrimenti si ripete con un nuovo campione (per evitare falsi negativi), regolando $\sigma'$.

\section{Perch\'e andare oltre Apriori}\label{sec:oltre-apriori}
Apriori richiede (i) generare esplicitamente i candidati $C_k$ a ogni livello e (ii) pi\`u passate sul database per calcolare i supporti. Con soglie basse o molti pattern, il numero di candidati esplode e le scansioni diventano costose. \textbf{FP-Growth} evita entrambi: rappresenta il DB in modo compatto (\emph{FP-tree}) e \emph{fa crescere} i pattern frequenti senza generare $C_k$.

\section{FP-Growth: idea di base}\label{sec:fpgrowth}
\begin{enumerate}
  \item \textbf{Costruzione FP-tree} (\emph{Frequent Pattern tree}): scansiona il DB per ottenere i supporti degli item, scarta quelli con supporto $<\sigma$, ordina gli item per supporto decrescente e inserisci le transazioni nell’albero condividendo i prefissi comuni. Mantieni una \emph{header table} con link ai nodi per item.
  \item \textbf{Pattern-growth}: per ogni item $x$ (dall’ultimo al primo nell’ordine per supporto) estrai la \emph{pattern base condizionale} di $x$ dai cammini che portano a $x$, costruisci l’\emph{FP-tree condizionale} e ripeti ricorsivamente. I pattern trovati si concatenano con $x$.
\end{enumerate}
Servono in genere \textbf{due passate} sul DB (una per i conteggi degli item, una per costruire l’albero); poi si lavora su strutture in memoria.

\subsection{Costruzione dell'FP-tree}\label{subsec:costruzione-fptree}
\begin{enumerate}
  \item \textbf{Prima passata}: calcola $\mathrm{supp}(i)$ per ogni item; elimina gli item con $\mathrm{supp}(i)<\sigma$.
  \item \textbf{Ordina} gli item per supporto decrescente (tie-break fisso) e \textbf{riordina} ogni transazione seguendo lo stesso ordine.
  \item \textbf{Inserisci} ciascuna transazione nell’albero a partire dalla radice: percorri/crea i nodi lungo il prefisso ordinato, incrementando i contatori dei nodi e aggiornando i \emph{node link} nella header table.
\end{enumerate}
\emph{Propriet\`a}: l’FP-tree conserva l’informazione necessaria a ricostruire i supporti dei pattern frequenti ed \`e molto compatto se molte transazioni condividono prefissi.

\subsection{Esempio di FP-Growth}\label{subsec:fpg-example}
Soglia $\sigma=3$. Dalla prima passata otteniamo gli item frequenti (con supporto) in ordine decrescente:
\[
f:4,\quad c:4,\quad a:3,\quad b:3,\quad m:3,\quad p:3.
\]
Ogni transazione viene \textbf{riordinata} secondo l’ordine $f\!\succ c\!\succ a\!\succ b\!\succ m\!\succ p$ ed \textbf{inserita} nell’FP-tree, aggregando i prefissi per incrementare i contatori.

\paragraph{Header table iniziale.}
\begin{center}
\begin{tabular}{@{}lcccccc@{}}
\toprule
Item & $f$ & $c$ & $a$ & $b$ & $m$ & $p$ \\
\midrule
$\mathrm{supp}(\cdot)$ & 4 & 4 & 3 & 3 & 3 & 3 \\
\bottomrule
\end{tabular}
\end{center}

\begin{figure}[htbp]
  \centering
  \includegraphics[width=0.92\textwidth]{images/fp-growth-complete.png}
  \caption{Costruzione dell’FP-tree: a sinistra le transazioni con gli item frequenti ordinati; a destra l’albero ottenuto condividendo i prefissi e incrementando i contatori dei nodi. L’ultima transazione (TID 500) segue il percorso f $\rightarrow$ c $\rightarrow$ a $\rightarrow$ m $\rightarrow$ p e aggiorna i relativi nodi.}
  \label{fig:fp-growth-complete}
\end{figure}

\paragraph{Visita per pattern-growth.}
Si processano gli item \emph{dal meno frequente al pi\`u frequente} nell’ordine della header table (a parit\`a, dall’ultimo al primo):
\[
p \rightarrow m \rightarrow b \rightarrow a \rightarrow c \rightarrow f.
\]

\paragraph{Come si espande un item $x$ (pattern-growth).}
\begin{enumerate}
  \item \textbf{Pattern base condizionale di $x$.} Segui i \emph{node link} di $x$ e,
        per ogni nodo $x$, prendi il cammino dalla radice al \emph{genitore} di $x$
        (escludi $x$). Assegna a quel cammino un \emph{peso} uguale al contatore del nodo $x$.
        % Multinsieme di cammini pesati: quali prefissi compaiono insieme a $x$ e quanto spesso.
  \item \textbf{FP-tree condizionale di $x$.} Dai cammini pesati:
        (i) somma i pesi per ogni item e \emph{rimuovi} quelli con supporto $<\sigma$;
        (ii) ordina gli item per supporto decrescente; 
        (iii) inserisci i cammini (con pesi) costruendo l’albero $T_x$.
  \item \textbf{Ricorsione e output.} I pattern frequenti che \emph{contengono} $x$
        sono $\{x\}$ unito a ciascun pattern frequente trovato in $T_x$.
        \emph{Caso speciale (cammino unico)}: se $T_x$ è una sola path,
        tutte le combinazioni dei suoi nodi sono frequenti; il supporto è il \emph{minimo} dei contatori lungo la combinazione.
\end{enumerate}

\begin{figure}[htbp]
  \centering
  \includegraphics[width=0.9\textwidth]{images/fp-growth-links.png}
  \caption{Header table e node-link per l’item $p$: i puntini tratteggiati collegano le occorrenze di $p$ nell’FP-tree. Seguendo i node-link si raccolgono i cammini verso la radice (senza $p$) con i rispettivi contatori: questa è la pattern base condizionale di $p$, da cui si costruisce l’FP-tree condizionale $T_p$.}
  \label{fig:fp-growth-links}
\end{figure}

\paragraph{Esempio 1: item $p$.}
Supponiamo che, seguendo i \emph{node link} di $p$, si incontrino i cammini verso radice:
\[
\langle f,c,a,m\rangle:2 \quad \text{e} \quad \langle c,b\rangle:1.
\]
La \textbf{base condizionale} di $p$ \`e quindi $\{\langle f,c,a,m\rangle \text{ con peso } 2,\ \langle c,b\rangle \text{ con peso } 1\}$. Con $\sigma=3$ nessun sotto-pattern che include $p$ raggiunge la soglia (pesi massimi 2 e 1), dunque \emph{nessun} pattern frequente contiene $p$.

\paragraph{Esempio 2: item $m$.}
Cammini verso $m$ (esempio coerente con le slide):
\[
\langle f,c,a\rangle:2,\quad \langle f,c\rangle:1.
\]
La base condizionale di $m$ \`e $\{\langle f,c,a\rangle:2,\ \langle f,c\rangle:1\}$. Frequenze condizionali:
\[
\mathrm{supp}_{\mathrm{cond}}(f)=3,\ \mathrm{supp}_{\mathrm{cond}}(c)=3,\ \mathrm{supp}_{\mathrm{cond}}(a)=2.
\]
Con $\sigma=3$ risultano frequenti i pattern $\{m,f\}$, $\{m,c\}$ e, proseguendo, $\{m,f,c\}$ con supporto $3$ (intersezione dei cammini). 

\paragraph{Esempio 3: item $b$.}
Cammini verso $b$:
\[
\langle f,c,a\rangle:2,\quad \langle c\rangle:1.
\]
Base condizionale di $b$: $\{\langle f,c,a\rangle:2,\ \langle c\rangle:1\}$. Frequenze condizionali:
\[
\mathrm{supp}_{\mathrm{cond}}(c)=3,\ \mathrm{supp}_{\mathrm{cond}}(f)=2,\ \mathrm{supp}_{\mathrm{cond}}(a)=2.
\]
Con $\sigma=3$ si ottiene $\{b,c\}$ frequente; combinazioni con $f$ o $a$ non superano la soglia.

\section{Confronto: FP-Growth vs Apriori}\label{subsec:confronto-fp-apriori}
\begin{table}[htbp]
\centering
\begin{tabular}{@{}p{0.28\textwidth}p{0.33\textwidth}p{0.33\textwidth}@{}}
\toprule
\textbf{Aspetto} & \textbf{Apriori} & \textbf{FP-Growth} \\
\midrule
Generazione candidati &
Sì: crea $C_k$ a ogni livello (rischio di esplosione combinatoria) &
No: crescita diretta dei pattern dall’FP-tree \\
Accessi al DB &
Molte passate (una per ogni $k$) &
Tipicamente 2 passate, poi si lavora in memoria \\
Strutture dati principali &
Liste di candidati e conteggi &
FP-tree + header table (node-link) \\
Quando preferirlo &
DB piccoli/sparsi, soglie alte, ambienti distribuiti molto semplici &
DB densi, soglie basse, molti prefissi condivisi (compressione efficace) \\
Note pratiche &
Pruning con principio di Apriori, implementazione semplice &
Evita i candidati; molto veloce se l’FP-tree è compatto \\
\bottomrule
\end{tabular}
\caption{Confronto sintetico tra Apriori e FP-Growth.}
\label{tab:apriori-vs-fpgrowth}
\end{table}
\chapter{Clustering}
Il clustering è una tecnica di apprendimento \textbf{non supervisionato} che mira a \textbf{raggruppare} un insieme di oggetti in modo che gli oggetti all'interno dello stesso gruppo (o cluster) siano più simili tra loro rispetto a quelli di altri gruppi. Questa tecnica è utile per scoprire strutture nascoste nei dati e per ridurre la dimensionalità.

\section{Spazi metrici e distanze}
Per applicare il clustering, è necessario definire una \textbf{metrica} che misura la \textbf{distanza} tra gli oggetti. Gli oggetti da raggruppare sono, infatti, punti appartenenti a un certo spazio metrico $S$ dove è possibile definire una funzione di distanza $D: S \times S \to \mathbb{R}^+$ che soddisfi le seguenti proprietà:
\begin{enumerate}
    \item \textbf{Non negatività}: $D(x, y) \geq 0$ per ogni $x, y \in S$.
    \item \textbf{Simmetria}: $D(x, y) = D(y, x)$ per ogni $x, y \in S$.
    \item \textbf{Disuguaglianza triangolare}: $D(x, z) \leq D(x, y) + D(y, z)$ per ogni $x, y, z \in S$.
\end{enumerate}

\subsection{Spazio euclideo}
Un esempio di spazio metrico è lo spazio euclideo $\mathbb{R}^n$ con la metrica euclidea, definita come:
\[
D_2(x, y) = \sqrt{\sum_{i=1}^n (x_i - y_i)^2} 
\]
dove $x = (x_1, x_2, \ldots, x_n)$ e $y = (y_1, y_2, \ldots, y_n)$ sono due punti in $\mathbb{R}^n$ (si usa $D_2$ per indicare che utilizza la norma $L_2$).


Un problema della distanza euclidea è che utilizza la norma $L_2$, che può non essere adatta per tutti i tipi di dati, specialmente quando le variabili hanno scale diverse o quando i dati contengono outlier. In questi casi, si possono considerare altre metriche come la distanza di Manhattan (con norma $L_1$):
\[
D_1(x, y) = \sum_{i=1}^n |x_i - y_i|
\]

Altre distanze in spazi euclidei includono:
\begin{align*}
&D_r = \left( \sum_{i=1}^n |x_i - y_i|^r \right)^{1/r} &\text{(norma $L_r$)} \\
&D_\infty = \max_{i} |x_i - y_i| &\text{(norma $L_\infty$)} \\
&D_{cos} = 1 - \frac{x \cdot y}{\|x\| \|y\|} &\text{(distanza coseno)}
\end{align*}

\noindent
dove in particolare la distanza coseno è utile per misurare la similarità tra vettori in spazi ad alta dimensione, come nel caso di documenti rappresentati da vettori di frequenze di parole.

\paragraph{Centroide.}
Negli spazi euclidei viene utilizzato il concetto di \textbf{centroide} per rappresentare un cluster. Il centroide di un insieme di punti $X = \{x_1, x_2, \ldots, x_k\}$ è definito come:
\[
c(X) = \frac{1}{k} \sum_{i=1}^k x_i
\]
ovvero $c(X)$ è il punto medio di tutti i punti nel cluster. Il centroide minimizza la somma delle distanze quadrate dai punti del cluster, rendendolo una rappresentazione efficace della posizione centrale del cluster.

\subsection{Spazi non euclidei}
Negli spazi non euclidei il concetto di centroide può non essere definito o utile. In questi casi, si possono utilizzare altre rappresentazioni per i cluster, come il \textbf{medoide}, che è un punto reale del dataset che minimizza la somma delle distanze ai punti del cluster:
\[
m(X) = \arg\min_{x \in X} \sum_{y \in X} D(x, y)
\]
Il medoide è particolarmente utile in spazi dove non ha senso calcolare un centroide, come in spazi discreti o quando i dati sono categoriali. Esempi di distanze non euclidee sono:
\begin{description}
    \item[Distanza di Edit.] - La distanza di edit misura quanto è necessario modificare una stringa per trasformarla in un'altra, considerando operazioni come inserimenti, cancellazioni e sostituzioni di caratteri. Ad esempio partendo da una stringa $A$ e trasformandola in una stringa $B$, la distanza di edit può essere calcolata come il numero minimo di operazioni necessarie per ottenere $B$ da $A$.
    \item[Distanza di Hamming.] - La distanza di Hamming conta il numero di posizioni in cui due stringhe di uguale lunghezza differiscono, ed è utile per dati binari o categoriali. Ad esempio, per le stringhe $A = 10101$ e $B = 10011$, la distanza di Hamming è 2, poiché differiscono nelle posizioni 2 e 4.
    \item[Distanza di Jaccard.] - La distanza di Jaccard misura la dissimilarità tra due insiemi, definita come il rapporto tra l'intersezione e l'unione degli insiemi:
    \[D_J(A, B) = 1 - \frac{|A \cap B|}{|A \cup B|}\]
    Questa distanza è particolarmente utile per dati categoriali o binari, come nel caso di documenti rappresentati da insiemi di parole. Ad esempio, per gli insiemi $A = \{1, 2, 3\}$ e $B = \{2, 3, 4\}$, la distanza di Jaccard è:
    \[D_J(A, B) = 1 - \frac{|\{2, 3\}|}{|\{1, 2, 3, 4\}|} = 1 - \frac{2}{4} = 0.5\]
\end{description}

\section{Algoritmi di clustering}
Gli algoritmi di clustering possono essere classificati in base al tipo di metrica utilizzata e alla loro strategia di raggruppamento. 

\subsection{Tipi di clustering}
Una tassonomia comune include:
\begin{description}
    \item[Clustering gerarchico o agglomerativo.] - Costruisce una gerarchia di cluster, che può essere rappresentata come un dendrogramma. Gli algoritmi più comuni sono l'algoritmo di Agglomerative Nesting (AGNES) e il Divisive Analysis (DIANA). Questi algoritmi iniziano con ogni punto come un cluster separato e successivamente uniscono i cluster più vicini fino a ottenere un unico cluster o fino a raggiungere un numero desiderato di cluster.
    \item[Clustering partizionale.] - Questi algoritmi cercano di partizionare i dati in un numero fisso di cluster, come il K-means e il K-medoids. Il K-means cerca di minimizzare la somma delle distanze quadrate tra i punti e i loro centroidi, mentre il K-medoids utilizza i medoidi come rappresentanti dei cluster.
    \item[Clustering per densità.] - Identifica cluster come aree di alta densità separate da aree di bassa densità. L'algoritmo DBSCAN (Density-Based Spatial Clustering of Applications with Noise) è un esempio di questo tipo di clustering, che richiede due parametri: il raggio di ricerca (eps) e il numero minimo di punti in un cluster (minPts). DBSCAN è particolarmente utile per identificare cluster di forma arbitraria e per gestire il rumore nei dati.
\end{description} 

\subsection{Bontà di un algoritmo}
La bontà di un algoritmo di clustering può essere valutata attraverso diversi fattori:
\begin{itemize}
    \item Scalabilità: l'algoritmo deve essere in grado di gestire grandi dataset senza un aumento esponenziale del tempo di esecuzione.
    \item Robustezza: l'algoritmo deve essere in grado di gestire rumore e outlier nei dati senza influenzare significativamente i risultati del clustering.
    \item Interpretabilità: i risultati del clustering devono essere facilmente interpretabili e utili per l'analisi dei dati.
    \item Stabilità: l'algoritmo dovrebbe produrre risultati simili su esecuzioni multiple con lo stesso dataset, a meno che non vengano introdotte variazioni significative nei dati.
    \item Flessibilità: l'algoritmo dovrebbe essere in grado di gestire diversi tipi di dati, come dati numerici, categoriali o testuali, e dovrebbe essere in grado di utilizzare diverse metriche di distanza a seconda delle esigenze del problema.
    \item Insensibilità: l'algoritmo non deve essere troppo sensibili all'aggiunta di nuovi dati, in modo che i cluster rimangano stabili anche con l'introduzione di nuovi punti.
\end{itemize}

\subsection{Curse of dimensionality}
Uno dei problemi comuni che il clustering deve affrontare è la \textbf{curse of dimensionality}, che si riferisce al fenomeno in cui l'aumento del numero di dimensioni (caratteristiche) rende difficile la misurazione delle distanze tra i punti. In spazi ad alta dimensione, i punti tendono a diventare equidistanti, rendendo difficile distinguere tra cluster. Per convincersi di questo problema, si può usare la distanza euclidea tra due punti: per un certo $n$ molto grande e per punti casuali $x$ e $y$, la distanza tende a essere molto simile per tutti i punti:
\begin{itemize}
    \item $D_2(x, y)$ ha un'alta probabilità di essere almeno pari a $1$, perché la somma di molti termini positivi tende a essere grande creando un \textbf{limite inferiore}.
    \item Invece è improbabile che $D_2(x, y)$ sia molto grande, perché la probabilità che tutti i termini della somma siano grandi è bassa, creando un \textbf{limite superiore}.
\end{itemize}

Da questo, possiamo dire che in alta dimensione, tutte le distanze tendono a concentrarsi intorno a un valore medio:
\[
\underbrace{1}_{\text{limite inferiore}} \ \lesssim\ D(\mathbf{x},\mathbf{y})\ \lesssim\ \underbrace{\sqrt{n}}_{\text{limite superiore}}
\]

\subsection{Ortogonalità dei vettori}
Un altro aspetto della curse of dimensionality è l'ortogonalità dei vettori in spazi ad alta dimensione. In uno spazio euclideo di dimensione elevata, la maggior parte dei vettori tende ad essere quasi ortogonale tra loro. Questo significa che l'angolo tra due vettori casuali tende a essere vicino a 90 gradi, rendendo difficile trovare direzioni significative nei dati. Questo fenomeno può complicare ulteriormente il processo di clustering, poiché i cluster potrebbero non essere ben definiti in termini di direzioni nei dati.

Per convincersi di questo, si può considerare la distanza del coseno tra due vettori casuali in uno spazio di dimensione elevata. La distanza del coseno tende a essere vicina a 1, indicando che i vettori sono quasi ortogonali:
\[
D_{cos}(x, y) = 1 - \frac{x \cdot y}{\|x\| \|y\|} \approx 1
\]

Questo perché il denominatore è formato da quantità positive, mentre il numeratore (il prodotto scalare) tende a essere piccolo in confronto, poiché le componenti dei vettori si annullano a vicenda in alta dimensione. Si può dimostrare che, al crescere di $n$, il denominatore cresca linearmente mentre il numeratore assuma un valore atteso 0 con una deviazione standard pari a $\sqrt{n}$. Pertanto, il rapporto tende a 0 e la distanza del coseno tende a 1.

\section{Clustering Gerarchico}
Il clustering gerarchico è una tecnica di clustering che costruisce una gerarchia di cluster, rappresentata come un \textbf{dendrogramma}: un albero che mostra le relazioni tra i cluster a diversi livelli di granularità. Esistono due approcci principali al clustering gerarchico: l'approccio agglomerativo e l'approccio divisivo.

\subsection{Distanze tra cluster}
Per considerare due cluster vicini si devono utilizzare delle metriche:
\begin{itemize}
    \item In uno spazio euclideo si considera il \emph{centroide} di ogni cluster e si calcola la distanza tra i centroidi.
    \item In spazi non euclidei si considera il \emph{medoide} di ogni cluster e si calcola la distanza tra i medoidi.
\end{itemize}

Uno dei problemi di utilizzare il centroide o il medoide è che non sempre rappresentano bene la forma del cluster, specialmente se i cluster hanno forme complesse o non sono convessi. Per questo motivo, si possono utilizzare altre strategie per misurare la distanza tra cluster:

\begin{figure}[htbp]
\centering
\begin{minipage}[t]{0.65\textwidth}
\vspace{0pt}
\begin{itemize}[leftmargin=1.2em]
  \item \textbf{Single-link}: $\min\{D(x,y): x\in C_i,\ y\in C_j\}$ (tende a catene). Questo metodo può portare a cluster allungati e poco compatti, poiché si basa solo sulla distanza minima tra i punti dei cluster, permettendo a punti distanti di essere raggruppati insieme se esiste una catena di punti vicini.
  \item \textbf{Complete-link}: $\max\{D(x,y): x\in C_i,\ y\in C_j\}$ (favorisce cluster compatti). Questo metodo tende a creare cluster più compatti e sferici, poiché considera la distanza massima tra i punti dei cluster, evitando che punti lontani vengano raggruppati insieme.
  \item \textbf{Average-link}: media delle distanze su tutte le coppie $x\in C_i,\,y\in C_j$ (compromesso). Questo metodo bilancia le caratteristiche di single-link e complete-link, considerando la distanza media tra tutti i punti dei cluster, risultando in cluster più equilibrati.
\end{itemize}
\end{minipage}\hfill
\begin{minipage}[t]{0.27\textwidth}
\vspace{0pt}
\centering
\includegraphics[width=\linewidth]{images/cluster_distances.png}
\caption{Esempi grafici delle diverse nozioni di distanza tra cluster.}
\label{fig:cluster_distances}
\end{minipage}
\end{figure}

\subsection{Dendrogramma}
Un dendrogramma è una rappresentazione grafica della gerarchia dei cluster ottenuta tramite il clustering gerarchico. Ogni foglia del dendrogramma rappresenta un punto dati, mentre i nodi interni rappresentano i cluster formati unendo i punti o i cluster più vicini. L'altezza di un nodo nel dendrogramma indica la distanza tra i cluster uniti in quel punto.

\begin{figure}
  \centering
  \includegraphics[width=0.75\textwidth]{images/dendograms.png}
  \caption{A sinistra: punti nel piano con centroidi (triangoli) e cerchi che schematizzano la coesione dei gruppi; i colori indicano i cluster. A destra: dendrogramma agglomerativo che mostra l'ordine di fusione e l'altezza (distanza di linkage). Un taglio orizzontale del dendrogramma determina il numero di cluster.}
  \label{fig:dendograms}
\end{figure}

Il dendrogramma consente di visualizzare come i cluster si formano a diversi livelli di distanza, permettendo di scegliere il numero di cluster desiderato effettuando un taglio orizzontale al livello appropriato. Ad esempio, tagliando il dendrogramma a un'altezza specifica, si possono ottenere un certo numero di cluster, come mostrato nella Figura \ref{fig:dendograms}.

Può essere tuttavia utilizzato come \textbf{criterio di arresto} per gli algoritmi agglomerativi: si può decidere di fermarsi quando la distanza tra i cluster da unire supera una certa soglia, evitando così di dover specificare a priori il numero di cluster desiderato.

\subsection{Clustering divisivo}
Uno dei problemi che il clustering agglomerativo può incontrare è la \textbf{scelta del numero di cluster}: spesso non è noto a priori quanti cluster esistano nei dati, e scegliere un numero errato può portare a risultati di clustering subottimali. 

Il clustering \emph{divisivo} affronta questo problema iniziando con tutti i punti in un unico cluster e suddividendolo iterativamente in cluster più piccoli (fino a raggiungere un criterio di arresto). Questo approccio consente di esplorare la struttura dei dati in modo più flessibile, poiché non richiede la specifica del numero di cluster iniziale.

\subsection{Complessità computazionale}
Al primo passo si valuta la distanza per ogni coppia di cluster e si sceglie la migliore: costo $\Theta(n^2)$.
Dopo ogni fusione i cluster diminuiscono di uno, quindi i passi successivi costano, nell'ordine,
$(n-1)^2,(n-2)^2,\dots,2^2$.
\[
T_{\text{naive}}
=\sum_{k=2}^{n} k^{2}
=\frac{n(n+1)(2n+1)}{6}-1
=\Theta(n^{3}).
\]

Utilizzando però delle \textbf{code di priorità} per memorizzare le distanze tra i cluster e ottenere il minimimo in $O(1)$, si può ridurre la complessità a $\Theta(n^2 \log n)$:
\[
T_{\text{optimized}}
=\sum_{k=2}^{n} \left( O(\log k) + O(k) \right)
=O(n^2 \log n).
\]

\section{Clustering partizionale: K-means}
Il clustering partizionale è un'alternativa più efficiente rispetto al clustering gerarchico, specialmente per grandi dataset. Questi algoritmi cercano di partizionare i dati in un numero fisso di cluster, ottimizzando una funzione obiettivo che misura la qualità del clustering.

Il clustering partizionale più noto è il K-means, che mira a minimizzare la somma delle distanze quadrate tra i punti e i loro centroidi. L'algoritmo assume di conoscere a priori il numero di cluster $k$ e procede iterativamente attraverso i seguenti passi:
\begin{enumerate}
    \item Si scelgono inizialmente $k$ punti che abbiano probabilità alta di essere ben distribuiti (ad esempio, selezionandoli casualmente dal dataset) come centroidi iniziali.
    \item Si assegna ogni punto al cluster il cui centroide è più vicino, utilizzando la distanza euclidea.
    \item Si ricalcolano i centroidi di ogni cluster come la media dei punti assegnati a quel cluster.
    \item Si ripetono i passi 2 e 3 fino a quando i centroidi non cambiano più significativamente o fino a raggiungere un numero massimo di iterazioni.
\end{enumerate}

\subsection{Scelta greedy dei centroidi iniziali}
Un modo semplice per scegliere i centroidi iniziali è selezionarli casualmente dal dataset. Tuttavia, questa scelta può portare a risultati subottimali se i centroidi iniziali non sono ben distribuiti. Un modo ottimale è utilizzare una scelta \emph{greedy}\footnote{Un algoritmo greedy è un algoritmo che prende decisioni localmente ottimali in ogni passo con l'aspettativa di trovare una soluzione globale ottimale.} che massimizzi la distanza tra i centroidi iniziali:
\begin{enumerate}
    \item Si sceglie il primo centroide casualmente dal dataset $c_1$ e lo si aggiunge all'insieme $S$ dei punti scelti.
    \item Si aggiunge a $S$ il punto $x \in X \setminus S$ ($X$ è il dataset originale) che massimizza la distanza minima da tutti i punti già scelti:
    \[
    c_{i} = \arg\max_{x \in X \setminus S} \min_{y \in S} D(x, y)
    \]
    \item Si ripete il passo 2 fino a quando non si sono scelti $k$ centroidi.
\end{enumerate}

\subsection{Funzione obiettivo}
Per arrestare l'algoritmo K-means, si può utilizzare una funzione obiettivo che misura la qualità del clustering. Questa funzione può essere definita a paartire dalla somma delle distanze quadrate tra i punti e i loro centroidi:
\[
E = \sum_{i=1}^k \sum_{x \in C_i} D_2(x, c_i)^2
\]
dove $C_i$ è il cluster $i$ e $c_i$ è il centroide del cluster $i$. L'algoritmo K-means mira a minimizzare questa funzione obiettivo, e si può arrestare quando la variazione di $E$ tra due iterazioni consecutive è inferiore a una soglia predefinita.

\subsection{Scelta del numero di cluster}
Uno dei problemi del clustering partizionali è la scelta del numero di cluster $k$. Generalmente non si conoscono a priori il numero di cluster nei dati, e scegliere un valore errato può portare a risultati di clustering subottimali.

Si può quindi considerare il valore di $k$ come un \textbf{iperparametro}\footnote{Un parametro il cui valore è fissato prima dell'addestramento e non viene appreso dal modello.} da ottimizzare. 

\paragraph{Metodo elbow.}
Un metodo comune per scegliere il numero di cluster è il \textbf{metodo elbow}, che consiste nel calcolare la funzione obiettivo $E$ per diversi valori di $k$ e tracciare un grafico di $E$ in funzione di $k$. Si cerca quindi un punto nel grafico dove la diminuzione di $E$ inizia a rallentare, formando una "gomito" (elbow). Questo punto indica un buon compromesso tra la qualità del clustering e la complessità del modello.

\begin{figure}[htbp]
  \centering
  \includegraphics[width=0.8\textwidth]{images/elbow-k.png}
  \caption{Metodo (\emph{elbow}). Si traccia la distanza media dal centroide (o WCSS/n) al variare di $k$; il valore “ottimo” è nel punto di flesso, dove l'aumento di $k$ porta benefici marginali trascurabili.}
  \label{fig:elbow}
\end{figure}

Dalla figura \ref{fig:elbow} notiamo che la funzione:
\[
W(k) = \frac{1}{n} \sum_{i=1}^k \sum_{x \in C_i} D_2(x, c_i)^2
\]
decresce al crescere di $k$, ma il tasso di decrescita diminuisce. Si sceglie quindi il valore di $k$ in corrispondenza del punto di flesso della curva, dove l'aggiunta di ulteriori cluster non porta a una riduzione significativa della somma delle distanze quadrate. Questo può essere anche individuato matematicamente supponendo che tra due valori $x$ e $y$ del parametro $k$ ci sia una differenza \emph{non trascurabile} nella distanza media dai centroidi:
\begin{enumerate}
    \item Si prende il valore medio $z = (x + y) / 2$ e si effettua il clustering per $k = z$.
    \item Se il valore della distanza media dai centroidi per $k = z$ è vicino a quello per $k = x$, allora si sceglie $k = x$, altrimenti si sceglie $k = y$.
    \item Si ripetono i passi 1 e 2 fino a quando non si trova il valore ottimale di $k$.
\end{enumerate}

\paragraph{Metodo silhouette.}
Un altro metodo per scegliere il numero di cluster è il \textbf{metodo silhouette}, che valuta la qualità del clustering calcolando un indice di silhouette per ogni punto. L'indice di silhouette misura \emph{quanto} un punto \emph{è ben assegnato al suo cluster} rispetto agli altri cluster. Per un punto $x_i$ appartenente al cluster $C_i$, si definiscono:
\begin{itemize}
    \item $a(i)$: la distanza media tra $x_i$ e tutti gli altri punti nel suo cluster $C_i$ (coesione):
    \[
    a(i) = \frac{1}{|C_i| - 1} \sum_{x_j \in C_i, j \neq i} D(x_i, x_j)
    \]
    \item $b(i)$: la distanza media tra $x_i$ e tutti i punti nel cluster più vicino a $C_i$ (separazione):
    \[
    b(i) = \min_{j \neq i} \frac{1}{|C_j|} \sum_{x_k \in C_j} D(x_i, x_k)
    \]
\end{itemize}

L'indice di silhouette $s(i)$ per il punto $x_i$ è quindi definito come:
\[
s(i) = \frac{b(i) - a(i)}{\max(a(i), b(i))}
\]

Questo valore è compreso in $[-1, 1]$, dove un valore vicino a 1 indica che il punto è ben assegnato al suo cluster, un valore vicino a 0 indica che il punto è al confine tra due cluster, e un valore negativo indica che il punto potrebbe essere stato assegnato al cluster sbagliato. 

Possiamo analizzare anche le funzioni $a(i), b(i)$:
\begin{itemize}
    \item Se $a(i) \ll b(i)$, allora $s(i) \approx 1$: il punto è ben assegnato al suo cluster. Questo perché la distanza media all'interno del cluster è molto più piccola rispetto alla distanza media al cluster più vicino.
    \item Se $a(i) \approx b(i)$, allora $s(i) \approx 0$: il punto è al confine tra due cluster. In questo caso, la distanza media all'interno del cluster è simile a quella al cluster più vicino, indicando che il punto non è chiaramente associato a un cluster specifico.
    \item Se $a(i) \gg b(i)$, allora $s(i) \approx -1$: il punto potrebbe essere stato assegnato al cluster sbagliato. Questo accade quando la distanza media all'interno del cluster è maggiore rispetto alla distanza al cluster più vicino, suggerendo che il punto sarebbe meglio posizionato in un altro cluster.
\end{itemize}

Grazie all'indice di silhouette, possiamo definire un problema di ottimizzazione per scegliere il numero di cluster $k$:
\[
k* = \arg\max_{k} \frac{1}{n} \sum_{i=1}^n s(i)
\]
Ovvero, scegliamo il numero di cluster che massimizza la media degli indici di silhouette su tutti i punti del dataset. Questo approccio consente di valutare la qualità del clustering in modo più dettagliato rispetto al metodo elbow, poiché tiene conto della coesione e della separazione dei cluster.

\subsection{Complessità computazionale}
Dato $t$ il numero di iterazioni dell'algoritmo K-means, $k$ il numero di cluster ed $n$ il numero di elementi del dataset, la complessità computazionale dell'algoritmo K-means è:
\[
T = O(t \cdot k \cdot n)
\]
Questo perché in ogni iterazione si devono assegnare tutti i $n$ punti ai $k$ cluster (costo $O(k \cdot n)$) e poi ricalcolare i centroidi (costo $O(n)$, trascurabile rispetto al costo di assegnazione).

\subsection{K-means su Big data}
Per clusterizzare grosse quantità di dati in spazi con elevato numero di dimensioni, che non possonoo risiedere in memoria principale, si utilizzano opportune varianti del $k$-means. Queste varianti, BFR e CURE, utilizzano una serie di statistiche e di valori per rappresentare in modo compatto i cluster e ottimizzare l'uso della RAM (gestendo anche gli outlier).

\section{Clustering per densità: DBSCAN}
Il clustering per densità definisce i cluster come \emph{regioni dello spazio a densità elevata} separate da regioni a densità bassa. A differenza dei metodi partizionali, non richiede di fissare a priori il numero di cluster $k$ e riesce a individuare cluster di \emph{forma arbitraria} (non necessariamente convessi). Inoltre, tratta in modo naturale il \emph{rumore} marcandolo come outlier.

\subsection{DBSCAN}
DBSCAN (Density-Based Spatial Clustering of Applications with Noise) è un algoritmo di clustering per densità che identifica cluster come aree di alta densità separate da aree di bassa densità. In particolare una \emph{regione densa} è una regione contenente un numero di punti \emph{sufficientemente elevato} in un intorno dello spazio \emph{sufficientemente limitato}. DBSCAN funziona ad \textbf{intorni}\footnote{Un intorno di un punto $p$ è l'insieme di punti che si trovano entro una certa distanza $\varepsilon$ da $p$.}, che richiede due parametri:
\begin{itemize}
    \item $\varepsilon$ (eps): il raggio dell'intorno.
    \item $MinPts$: il numero minimo di punti richiesti per formare una regione densa.
\end{itemize}

\noindent
Da qui si possono definire tre tipi di punti, per un certo intorno $N_\varepsilon(Q)$\footnote{Intorno N di raggio $\varepsilon$ con centro Q}:
\begin{enumerate}
    \item \textbf{Punto $\mathbf{P}$ direttamente raggiungibile per densità da $\mathbf{Q}$}: se $P \in N_\varepsilon(Q)$ e $|N_\varepsilon(Q)| \geq MinPts$. Questo geometricamente, significa che $P$ si trova all'interno dell'intorno di $Q$ e che l'intorno di $Q$ contiene almeno $MinPts$ punti, indicando che $Q$ è in una regione densa.
    \item \textbf{Punto $\mathbf{P}$ raggiungibile per densità da $\mathbf{Q}$}: se esiste una catena di punti $P_1, P_2, \ldots, P_n$ tali che $P_1 = Q$, $P_n = P$ e ogni punto $P_{i+1}$ è direttamente raggiungibile per densità da $P_i$. Geometricamente, significa che si può raggiungere $P$ partendo da $Q$ attraversando una serie di punti, ciascuno dei quali si trova in una regione densa.
    \item \textbf{Punto $\mathbf{P}$ densamente connesso a $\mathbf{Q}$}: se esiste un punto $O$ tale che sia $P$ che $Q$ sono raggiungibili per densità da $O$. Geometricamente, significa che sia $P$ che $Q$ possono essere raggiunti partendo da un punto comune $O$ attraverso regioni dense.
\end{enumerate}

\begin{figure}[htbp]
\centering
\begin{subfigure}{0.32\textwidth}
  \centering
  \includegraphics[width=\linewidth]{images/dbscan_directly_reachable.png}
  \caption{Directly density-reachable}
\end{subfigure}\hfill
\begin{subfigure}{0.32\textwidth}
  \centering
  \includegraphics[width=\linewidth]{images/dbscan_reachable.png}
  \caption{Density-reachable}
\end{subfigure}\hfill
\begin{subfigure}{0.32\textwidth}
  \centering
  \includegraphics[width=\linewidth]{images/dbscan_connected.png}
  \caption{Density-connected}
\end{subfigure}
\caption{Relazioni di raggiungibilità e connessione per densità in DBSCAN (parametri $\varepsilon$ e \texttt{minPts}).}
\label{fig:dbscan_reachability}
\end{figure}

\paragraph{Definizione di cluster in DBSCAN.}
Sulla base di queste definizione, DBSCAN definisce un cluster come un insieme massimale di punti densamente connessi. In altre parole, un cluster è formato da tutti i punti che possono essere raggiunti per densità l'uno dall'altro, partendo da un punto iniziale in una regione densa. Quindi fissati $\varepsilon$ e $\text{MinPts}$, se $D$ è il dataset iniziale, $C$ è un cluster se:
\begin{itemize}
    \item $\forall P, Q \in D$, se $P \in C$ e $Q$ è densamente connesso a $P$, allora $Q \in C$. Questo spiega la \textbf{massimalità} del cluster: se un punto $Q$ è densamente connesso a un punto $P$ già nel cluster, allora $Q$ deve essere incluso nel cluster.
    \item $\forall P, Q \in C$, $P$ è densamente connesso a $Q$. Questo spiega la \textbf{connettività} del cluster: ogni coppia di punti all'interno del cluster deve essere densamente connessa.
\end{itemize}

\paragraph{Algoritmo DBSCAN.}
Una volta definito il concetto di cluster, l'algoritmo DBSCAN procede come segue:
\begin{enumerate}
    \item Si inizia con un punto non visitato $P$ dal dataset $D$.
    \item Si calcola l'intorno $N_\varepsilon(P)$.
    \begin{enumerate}
        \item Se $|N_\varepsilon(P)| < MinPts$, allora $P$ è etichettato come rumore (outlier) e si passa al punto successivo.
        \item Altrimenti, si crea un nuovo cluster $C$ e si aggiunge $P$ a $C$.
    \end{enumerate}
    \item Si aggiunge $P$ e il nuovo intorno $N_\varepsilon(P)$ al cluster $C$. Ricorsivamente si aggiungono tutti i punti in $N_\varepsilon(P)$ che sono direttamente raggiungibili per densità da $P$.
    \item Si ripete il processo fino a quando tutti i punti del dataset sono stati visitati.
\end{enumerate}

Uno dei problemi che rimane da risolvere, però, è la \textbf{scelta dei parametri} $\varepsilon$ e $MinPts$. Generalmente se si lavora in un dataset $D$ di dimensione $m$ campioni e $n$ features si può scegliere $\text{MinPts} \ge n+1$ per garantire che un cluster abbia almeno una certa dimensione minima. Per scegliere $\varepsilon$, si può utilizzare il grafico delle distanze $k$-nearest neighbor (k-NN), tracciando la distanza del $k$-esimo vicino più prossimo per ogni punto del dataset e cercando un punto di flesso nel grafico che indichi un buon valore di $\varepsilon$.

\begin{figure}[htbp]
  \centering
  \includegraphics[width=.68\textwidth]{images/kdistance_plot.png}
  \caption{$k$-distance plot: il gomito suggerisce il valore di $\varepsilon$.}
  \label{fig:kdist}
\end{figure}

\paragraph{Complessità computazionale.}
La complessità computazionale di DBSCAN dipende principalmente dal calcolo degli intorni per ogni punto del dataset. Se si utilizza una struttura dati efficiente come un albero KD o una griglia spaziale per accelerare la ricerca dei vicini, la complessità può essere ridotta a $O(n \log n)$, dove $n$ è il numero di punti nel dataset. Tuttavia, nella sua forma più semplice, senza strutture dati avanzate, la complessità è $O(n^2)$, poiché per ogni punto si deve calcolare l'intorno confrontandolo con tutti gli altri punti.

\subsection{OPTICS}
Uno dei problemi di DBSCAN è la difficoltà di scegliere i parametri $\varepsilon$ e $MinPts$, specialmente in dataset con densità variabile. OPTICS (Ordering Points To Identify the Clustering Structure) è un algoritmo di clustering per densità che affronta questo problema creando un ordinamento dei punti basato sulla loro densità locale, senza richiedere una scelta fissa di $\varepsilon$.

\paragraph{Distanza di raggiungibilità e area localizzata.}
OPTICS cerca di capire se due punti $A, B$ sono "abbastanza" vicini in termini di densità chiamandola \textbf{area localizzata} (\emph{core distance}) e \textbf{distanza di raggiungibilità} (\emph{reachability score}):
\begin{itemize}
    \item \textbf{Area localizzata} di un punto $A$ è la distanza tra $A$ e il suo $MinPts$-esimo vicino più prossimo. Se il numero di punti nell'intorno di $A$ è minore di $MinPts$, l'area localizzata è indefinita.
    \item \textbf{Distanza di raggiungibilità} di un punto $B$ da un punto $A$ è il massimo tra l'area localizzata di $A$ e la distanza tra $A$ e $B$:
    \[
    \text{reachability-distance}(A, B) = \max(\text{core-distance}(A), D(A, B))
    \]
\end{itemize}

\noindent
Per formalizzare questi concetti, definiamo due quantità per un punto $p$ e un intero $k$:
\begin{description}
  \item[\emph{Core-distance} di $p$:] è il raggio minimo che rende $p$ un punto \emph{core}. 
  In pratica è la distanza dal $k$-esimo vicino di $p$:
  \[
  \mathrm{core\_dist}_k(p)=d_k(p).
  \]
  Se $p$ ha meno di $k$ vicini, non è core e si pone $\mathrm{core\_dist}_k(p)=+\infty$ (non definita).
  
  \item[\emph{Reachability-distance} di $o$ da $p$:] misura quanto è “raggiungibile” $o$ partendo da $p$:
  \[
  \mathrm{reach\_dist}_k(o\,|\,p)=\max\{\mathrm{core\_dist}_k(p),\, D(p,o)\}.
  \]
\end{description}

Da queste definizioni si osserva che, se $p$ è un \emph{punto core}, allora la \emph{reachability-distance} di un punto $o$ a partire da $p$ è sempre definita ed è comunque \emph{non inferiore} alla $\mathrm{core\_dist}_k(p)$. Al contrario, se $p$ non è un punto core (cioè ha meno di $k$ vicini),
la reachability-distance di $o$ da $p$ non è definita.

Per chiarire il significato di questa definizione, consideriamo un esempio. Sia $k=5$ e supponiamo che
\[
\mathrm{core\_dist}_5(p)=0.8.
\]
Un punto $o$ a distanza $D(p,o)=0.6$ da $p$ avrà
\[
\mathrm{reach\_dist}_5(o \mid p)
= \max\{0.8,\,0.6\}
= 0.8,
\]
poiché domina la core-distance di $p$. Viceversa, un punto $o$ a distanza $D(p,o)=1.2$ da $p$ avrà
\[
\mathrm{reach\_dist}_5(o \mid p)
= \max\{0.8,\,1.2\}
= 1.2,
\]
in quanto prevale la distanza euclidea tra i due punti.


\paragraph{Reachability plot.}
OPTICS crea un ordinamento dei punti basato sulla loro reachability-distance, che può essere visualizzato in un \textbf{reachability plot}. In questo grafico, l'asse x rappresenta l'ordinamento dei punti e l'asse y rappresenta la reachability-distance di ciascun punto. Le aree con bassa reachability-distance indicano regioni di alta densità, mentre le aree con alta reachability-distance indicano regioni di bassa densità o rumore. Il reachability plot consente di identificare cluster di diverse densità senza dover specificare un valore fisso di $\varepsilon$: le valli nel grafico rappresentano cluster densi, mentre le creste rappresentano separazioni tra cluster.

\begin{figure}[htbp]
  \centering
  \includegraphics[width=.8\textwidth]{images/optics_reachability.png}
  \caption{OPTICS: esempio di \emph{reachability plot}. Le zone basse (valli) indicano cluster densi; le zone alte (creste) indicano separazioni. (immagine da libro/slide)}
  \label{fig:optics-reach}
\end{figure}

\paragraph{Estrazione dei cluster.}
Una volta ottenuto il reachability plot, si possono estrarre i cluster utilizzando diverse strategie. Una delle strategie più comuni è quella di utilizzare una soglia di reachability-distance per identificare le valli nel grafico che corrispondono ai cluster. In alternativa, si possono utilizzare regole basate sulla forma del grafico, come la regola \emph{steep down/up}, che identifica i punti in cui la reachability-distance scende bruscamente (indicando l'inizio di un cluster) e risale bruscamente (indicando la fine di un cluster).

\begin{figure}[htbp]
  \centering
  \includegraphics[width=.8\textwidth]{images/cluster_extraction_optics.png}
  \caption{OPTICS. (1) A sinistra: punti nel piano e, sotto, \emph{reachability plot}; le valli (segmenti colorati) corrispondono a regioni dense/cluster, mentre la linea orizzontale tratteggiata indica una soglia 
  $\epsilon$ che produce un taglio in stile DBSCAN. Le linee tratteggiate collegano ogni gruppo nel piano al suo intervallo nel plot. (2) A destra: regola \emph{steep down/up} per l'estrazione automatica dei cluster dal reachability plot (si entra quando la reachability scende bruscamente e si esce quando risale).}  
  \label{fig:optics-cluster-extraction}
\end{figure}

\paragraph{Complessità computazionale.}
La complessità computazionale di OPTICS è simile a quella di DBSCAN, poiché entrambi gli algoritmi si basano sul calcolo degli intorni dei punti. Utilizzando strutture dati efficienti come alberi KD o griglie spaziali, la complessità può essere ridotta a $O(n \log n)$, dove $n$ è il numero di punti nel dataset. Tuttavia, nella sua forma più semplice, senza strutture dati avanzate, la complessità è $O(n^2)$.

\subsection{HDBSCAN}
Uno dei problemi di OPTICS è la complessità computazionale, che può essere elevata per grandi dataset. HDBSCAN (Hierarchical Density-Based Spatial Clustering of Applications with Noise) è un'estensione di DBSCAN e OPTICS che combina i vantaggi di entrambi gli algoritmi, creando una gerarchia di cluster basata sulla densità e utilizzando tecniche di pruning per migliorare l'efficienza. 

L'unico parametro che richiede è $MinPts$, mentre la scelta di $\varepsilon$ viene gestita automaticamente attraverso la costruzione della gerarchia dei cluster. Costruisce un dendrogramma simile a quello di OPTICS, ma utilizza una metrica di \emph{stabilità} per selezionare i cluster più significativi dalla gerarchia, migliorando così la qualità del clustering e riducendo il rumore.

Si possono definire:
\begin{itemize}
    \item Core distance di $X$: distanza dal $MinPts$-esimo vicino più prossimo di $X$.
    \item Densità di mutual reachability tra $X, Y$:
    \[
    D_{mreach}(X, Y) = \max\{\text{core\_dist}(X), \text{core\_dist}(Y), D(X, Y)\}
    \]
    \item Grafo di mutual reachability: grafo completo pesato con i punti del dataset come nodi e pesi degli archi dati dalla densità di mutual reachability.
\end{itemize}

\begin{figure}[htbp]
  \centering
  \includegraphics[width=\textwidth]{images/HDBSCAN_example.png}
  \caption{Esempio di clustering con HDBSCAN. }
  \label{fig:hdbscan-example}
\end{figure}

\paragraph{Cluster in HDBSCAN.}
Fissato un valore del raggio $\varepsilon$, si considera il grafo $G$ di \emph{mutual reachability}, ottenuto mantenendo solo gli archi con peso minore o uguale a $\varepsilon$. Un \textbf{cluster in HDBSCAN} è una \textbf{componente connessa} di $G$ che contiene almeno $MinPts$ punti.

Al variare di $\varepsilon$, la struttura del grafo cambia: riducendo $\varepsilon$, alcune componenti connesse possono \textbf{restringersi}, \textbf{dividersi in sotto-componenti} oppure \textbf{scomparire}, quando il numero di punti scende sotto la soglia $MinPts$. In questo modo si ottengono partizionamenti diversi dello spazio, corrispondenti a diversi livelli di densità.

Questa evoluzione dei cluster al variare di $\varepsilon$ induce naturalmente una \textbf{gerarchia di cluster}, che può essere rappresentata tramite un
\textbf{dendrogramma}. HDBSCAN seleziona infine i cluster più significativi valutandone la \textbf{stabilità}, ossia la loro persistenza lungo intervalli ampi di $\varepsilon$:
i cluster che sopravvivono più a lungo ai cambiamenti di densità vengono considerati i più robusti.


Da qui si può costruire il dendrogramma dei cluster:
\begin{itemize}
    \item Si costruisce il Minimum Spanning Tree (MST) del grafo di mutual reachability.
    \item Si ordina gli archi del MST in ordine decrescente di peso.
    \item Si rimuovono gli archi uno alla volta, partendo da quello con il peso più alto, e si tengono traccia delle componenti connesse risultanti.
    \item Ogni volta che si rimuove un arco, si crea un nodo nel dendrogramma che rappresenta la fusione delle componenti connesse.
\end{itemize}

A partire dal dendrogramma, HDBSCAN estrae l'insieme dei cluster sulla base della loro stabilità, ovvero della loro persistenza attraverso diversi livelli di densità. Un cluster è considerato stabile se rimane consistente su un intervallo significativo di valori di $\varepsilon$.

Sia $\lambda = 1 / \varepsilon$ una misura minimia di densità che un cluster $C$ deve avere a profondità $\varepsilon$ per esistere. Si definiscono:
\begin{itemize}
    \item $\lambda_{\text{min}}$: il valore di $\lambda$ al quale il cluster $C$ appare per la prima volta (nasce).
    \item $\lambda_{\text{max}}$: il valore di $\lambda$ al quale il cluster $C$ si dissolve (muore).
    \item $\lambda_{\text{max}}(X, C)$: il valore di $\lambda$ al quale il punto $X \in C$ lascia il cluster $C$ (se mai lo lascia).
\end{itemize}

\noindent
Da qui, definiamo la \textbf{stabilità} di un cluster $C$ come:
\[
S(C) = \sum_{X \in C} \left( \lambda_{\text{max}}(X, C) - \lambda_{\text{min}} \right)
\]
Questa misura quantifica quanto a lungo i punti rimangono nel cluster $C$ mentre la densità varia. Un cluster con alta stabilità indica che i suoi punti rimangono insieme su un ampio intervallo di densità, suggerendo che il cluster è robusto e significativo.

\paragraph{Estrazione dei cluster stabili.}
Poiché vogliamo estrarre tra i $K$ cluster del dendrogramma quelli più stabili (senza sovrapposizioni), si può formulare il problema come un problema di ottimizzazione:
\[
\begin{cases}
\displaystyle \max \;\; \sum_{i=1}^{K} \delta_i\, S(C_i) \\[6pt]
\delta_{1,\ldots,K} \in \{0,1\}^{K} \\[4pt]
\displaystyle \sum_{j \in H} \delta_j = 1 \qquad \forall\, H \in \mathrm{Leaf}(D)
\end{cases}
\]
dove $\mathrm{Leaf}(D)$ è l'insieme delle foglie del dendrogramma $D$ e ogni vincolo assicura che tra i cluster che condividono una foglia ne venga scelto uno solo e $I_h$ è l'insieme degli indici dei cluster che stanno nel cammino che porta dal nodo foglia $H$ alla radice del dendrogramma.

Questo problema di ottimizzazione può essere risolto in maniera \textbf{bottom-up} partendo dalle foglie del dendrogramma, decidendo, per ogni cluster $C$, se è meglio selezionare come soluzione finale $C$ al posto dei suoi discendenti (selezionando $C$ si ottiene un guadagno in stabilità pari a $S(C)$, mentre selezionando i discendenti si ottiene un guadagno pari alla somma delle loro stabilità):
\begin{enumerate}
    \item Si inizializza $\delta_i=1$ e si pone $S'(C_i) = S(C_i)$ per ogni cluster $C_i \in \text{Leaf}(D)$.
    \item Si procede risalendo il dendrogramma. Per ogni cluster $C$ non foglia, si calcola la somma delle stabilità dei suoi figli:
    \[
    S_{\text{children}} = \sum_{C_j \in \text{children}(C)} S'(C_j).
    \]
    \item Si confronta $S(C)$ con $S_{\text{children}}$:
    \begin{itemize}
        \item Se $S(C) > S_{\text{children}}$, si seleziona $C$ come cluster finale, ponendo $\delta_C = 1$ e $\delta_{C_j} = 0$ per ogni figlio $C_j$ di $C$. Si aggiorna inoltre $S'(C) = S(C)$.
        \item Altrimenti, si mantiene la selezione dei figli, ponendo $\delta_C = 0$ e mantenendo $\delta_{C_j}$ come sono per ogni figlio $C_j$ di $C$. Si aggiorna inoltre $S'(C) = S_{\text{children}}$.
    \end{itemize}
    \item Si ripete il passo 3 fino a raggiungere la radice del dendrogramma.
\end{enumerate}

\begin{figure}[htbp]
  \centering
  \includegraphics[width=.46\textwidth]{images/hdbscan_mst.png}\hfill
  \includegraphics[width=.46\textwidth]{images/hdbscan_condensed_tree.png}
  \caption{HDBSCAN. Sinistra: MST costruito sulla \emph{mutual reachability distance} (archi più pesanti vengono tagliati al crescere di \(\lambda\)). Destra: \emph{condensed tree}; in evidenza i cluster scelti massimizzando la stabilità.}
  \label{fig:hdbscan-figs}
\end{figure}
\chapter{Classificazione}\label{ch:classificazione}

\section{Introduzione}\label{sec:intro-class}
La \textbf{classificazione} suddivide un insieme di dati in \emph{classi} note a priori (etichette),
apprendendo da esempi etichettati come assegnare la classe a nuove tuple. È quindi
\emph{apprendimento supervisionato}. Al contrario, il \emph{clustering} non parte da etichette
(\emph{unsupervised}) e scopre gruppi per similarità.

\paragraph{Predizione (regressione).}
Quando il target è \emph{numerico continuo}, il compito è di \emph{predire} un valore reale
(apprendimento supervisionato \emph{continuo}), cercando una funzione che approssimi
il target, non un confine tra classi.

\subsection{Schema generale di un classificatore}\label{subsec:schema-class}
\begin{enumerate}
  \item \textbf{Costruzione del modello} (training): si apprende da un \emph{training set} etichettato.
  \item \textbf{Validazione/valutazione} (test): si misura la bontà su un \emph{test set} etichettato.
  \item \textbf{Uso} (deploy): si applica il modello a nuove tuple per predirne la classe.
\end{enumerate}

\paragraph{Overfitting.}
L’overfitting si verifica quando un modello “impara a memoria” il training, compreso il rumore: va molto bene sui dati visti ma generalizza male su dati nuovi. In pratica è un segnale che il modello è troppo complesso rispetto alle informazioni disponibili. Per ridurlo, si separano chiaramente i dati per la verifica e si preferiscono soluzioni più semplici quando offrono prestazioni simili.

\begin{figure}[htbp]
  \centering
  \includegraphics[width=.72\textwidth]{images/schema_classificatore.png}
  \caption{Schema a blocchi di un classificatore: addestramento, validazione e uso.}
  \label{fig:schema-class}
\end{figure}

\subsection{Requisiti desiderabili}\label{subsec:req}
\begin{itemize}
  \item \textbf{Accuratezza}: corretta predizione delle classi (o del valore, per i predittori).
  \item \textbf{Velocità}: tempi di training e di classificazione contenuti.
  \item \textbf{Robustezza}: tolleranza a rumore e dati mancanti.
  \item \textbf{Scalabilità}: efficienza su dataset di grandi dimensioni.
\end{itemize}

% ==========================================================
\section{Alberi decisionali}\label{sec:trees}
Gli \textbf{alberi decisionali} classificano applicando test su attributi lungo i nodi interni;
le \emph{foglie} portano le etichette di classe.

\subsection{Classificazione tramite albero}\label{subsec:tree-class}
La classe di una tupla $q$ si ottiene seguendo il cammino radice$\to$foglia guidato
dai test. Ogni cammino implementa una regola \texttt{IF-THEN} (le condizioni interne sono congiunte in AND). L’insieme di regole è \emph{esaustivo} e \emph{mutuamente esclusivo} (ogni tupla è coperta da una sola regola).

\begin{figure}[htbp]
  \centering
  \begin{minipage}[t]{.50\textwidth}
    \centering
    \includegraphics[width=\linewidth]{images/weather_table.png}
  \end{minipage}\hfill
  \begin{minipage}[t]{.48\textwidth}
    \centering
    \includegraphics[width=\linewidth]{images/decision_tree_weather.png}
  \end{minipage}
  \caption{Dataset \emph{weather} (a sinistra) e albero decisionale appreso (a destra). 
  La tabella contiene 14 esempi con quattro attributi descrittivi (\texttt{Outlook}, \texttt{Temperature}, 
  \texttt{Humidity}, \texttt{Windy}) e la classe binaria \texttt{P/N}. 
  L’albero (stile ID3/C4.5) sceglie come radice \texttt{Outlook}; il ramo \texttt{overcast} porta 
  direttamente alla classe \texttt{P}, mentre per \texttt{sunny} si testa \texttt{Humidity} e per \texttt{rain} 
  si testa \texttt{Windy}. L’esempio illustra il passaggio da dati tabellari a regole interpretabili.}
  \label{fig:weather-tree}
\end{figure}

\subsection{Costruzione top–down}\label{subsec:topdown}
Costruzione ricorsiva dalla radice:
\begin{enumerate}
  \item Se tutte le tuple del nodo $X$ hanno la \emph{stessa} classe $C$, crea una foglia $C$.
  \item Altrimenti scegli un attributo $A$ (non ancora usato) e \emph{ramifica} $X$ (\emph{splitting}) secondo i valori/soglia di $A$; crea i figli.
  \item Per ogni figlio $X_i$: se puro, fermati; se impuro, ripeti ricorsivamente.
\end{enumerate}

\paragraph{Pruning.}
Se le tuple nel nodo sono poche o la profondità è elevata, si può fermare prima e rendere il nodo una foglia (vedere figura \ref{fig:weather-tree} con l'attributo "overcast").

\subsection{Splitting degli attributi}\label{subsec:splitting}
\begin{itemize}
  \item \textbf{Booleani/numerici}: split \emph{binario} su soglia $t$ (``$\le t$'' a sinistra, ``$>t$'' a destra).
  \item \textbf{Categoriali}: split \emph{binario} definendo un sottoinsieme non vuoto di valori (a sinistra se il valore \emph{non} appartiene al sottoinsieme, a destra altrimenti).
\end{itemize}

\subsection{Scelta dell’attributo e strategia greedy}\label{subsec:greedy}
L’albero minimale è un problema \emph{NP-hard}; si usa una strategia \emph{greedy} che, ad ogni passo, seleziona l’attributo con massima \emph{goodness} (partizioni più pure),
costruendo l’albero “più compatto” possibile.

\section{Misure di goodness}\label{sec:goodness}
La scelta dell’attributo si basa su misure di \emph{goodness}, che variano da algoritmo ad algoritmo.

\subsection{Information Gain (ID3)}\label{subsec:ig}
\paragraph{Idea.} L'\textbf{Information gain} è un algoritmo che si basa sull'idea di selezionare l'attributo che massimizza la riduzione dell'entropia riguardo alla classe delle tuple dopo lo split. Questo perché:
\begin{itemize}
  \item \textbf{Entropia massima}: si ha quando le classi sono equamente distribuite (massima incertezza).
  \item \textbf{Entropia minima}: si ha quando tutte le tuple appartengono alla stessa classe (certezza completa). 
\end{itemize}

\noindent
Sia $S_X$ l’insieme di tuple al nodo $X$, con due classi $P$ e $N$; si indichino con $p$
e $n$ le rispettive numerosità. L’\textbf{entropia} di $S_X$ è
\[
H(S_X)\;=\; -\frac{p}{p+n}\log_2\!\frac{p}{p+n}\;-\;\frac{n}{p+n}\log_2\!\frac{n}{p+n}.
\]
Sia $A$ un attributo con $k$ valori distinti, che induce la partizione
$S_X \to S_1,\dots,S_k$. Se $S_i$ contiene $p_i$ e $n_i$ elementi, allora
\[
H(S_i)\;=\; -\frac{p_i}{|S_i|}\log_2\!\frac{p_i}{|S_i|}\;-\;\frac{n_i}{|S_i|}\log_2\!\frac{n_i}{|S_i|}
\]

Possiamo anche calcolare l'\textbf{entropia media} dopo lo split su $A$:
\[
\overline{H}_A(S_X)\;=\;\sum_{i=1}^k \frac{|S_i|}{|S_X|}\,H(S_i).
\]

L’\textbf{information gain} è definito come la riduzione di entropia ottenuta dal partizionamento $S_x$ scegliendo l'attributo $A$:
\[
\mathrm{Gain}(S_X,A)\;=\;H(S_X)\;-\;\overline{H}_A(S_X).
\]
Si sceglie l’attributo con gain massimo.

\subsubsection*{Esempio e limitazioni}\label{par:ig-example}
Sul dataset “weather” (Fig.~\ref{fig:weather-tree}) si ottengono:
\[
\mathrm{Gain}(\textit{outlook})=0.246,\quad
\mathrm{Gain}(\textit{temperature})=0.029,\quad
\mathrm{Gain}(\textit{humidity})=0.151,\quad
\mathrm{Gain}(\textit{windy})=0.048.
\]

\paragraph{Limite noto.} L’Information Gain è \emph{sbilanciato} verso attributi con molti valori:
un attributo quasi univoco (es.\ \texttt{ID}) produce molte partizioni piccole (foglie pure),
abbattendo l’entropia media e gonfiando artificialmente , pur senza reale capacità
predittiva.

\begin{figure}[htbp]
  \centering
  \includegraphics[width=.9\textwidth]{images/ig_outlook_example.png}
  \caption{Esempio di scelta della radice con Information Gain sul dataset “weather”.}
  \label{fig:ig-weather}
\end{figure}

\subsection{Gain Ratio (C4.5)}\label{subsec:gain-ratio}
Il \textbf{Gain Ratio} corregge il bias dell’Information Gain verso attributi con molti valori introducendo la \emph{split information}, che misura quanta informazione è generata dal solo atto di partizionare i dati secondo l’attributo (indipendentemente dalla classe).

Sia $S$ l’insieme di tuple nel nodo corrente e $A$ un attributo che induce la partizione $S=S_1\cup\cdots\cup S_k$. Definiamo
\[
\mathrm{SplitInfo}(A,S)
= -\sum_{i=1}^k \frac{|S_i|}{|S|}\,\log_2\!\Big(\frac{|S_i|}{|S|}\Big),
\qquad
\mathrm{Gain}(A,S)=H(S)-\sum_{i=1}^{k}\frac{|S_i|}{|S|}\,H(S_i).
\]
Il \textbf{Gain Ratio} è
\[
\mathrm{GR}(A,S)=\frac{\mathrm{Gain}(A,S)}{\mathrm{SplitInfo}(A,S)}.
\]
\paragraph{Selezione in C4.5.} Per evitare divisioni spurie quando $\mathrm{SplitInfo}$ è piccola, C4.5 sceglie l’attributo con $\mathrm{GR}$ massimo tra quelli con \emph{Gain} non inferiore (ad es.) al gain medio del nodo. In pratica:
\begin{enumerate}
  \item calcola $\mathrm{Gain}(A,S)$ e scarta attributi con gain $\le 0$;
  \item tra i rimanenti, seleziona l’attributo con $\mathrm{GR}$ più alto.
\end{enumerate} 

\paragraph{Nota pratica (attributi continui).}
Per un attributo numerico $A$ si ordinano i valori e si valutano soglie candidate $t$ nelle posizioni fra due valori consecutivi: $A\le t$ vs $A>t$. Per ogni soglia si calcolano gain e gain ratio; si sceglie la soglia che massimizza la metrica.

% --------------------------------------------------------------------

\subsection{Gini Index (CART)}\label{subsec:gini}
Sia $i$ una classe e $T$ una tupla di classe $i$ scelta a caso da $S_x$. Per ricavare il \textbf{Gini Index} si calcola la probabilità che $T$ venga classificata erroneamente, ovvero che appartenga a una classe diversa da $i$: e quindi occorre considerare:
\begin{itemize}
  \item \textbf{Probabilità che $T$ sia di classe $i$}: $P(i\mid S_X)$.
  \item \textbf{Probabilità che $T$ sia di una classe diversa da $i$}: $1 - P(i\mid S_X)$.
\end{itemize}

\noindent
Dato che il ragionamento fatto vale per ogni classe, si sommano le probabilità di errore su tutte le classi:  
\[
\mathrm{Gini}(S_X) = \sum_{i=1}^n p_i (1-p_i) = \sum_{i=1}^n (p_i-p_i^2) = \sum_{i=1}^n p_i - \sum_{i=1}^n p_i^2 = 1 - \sum_{i=1}^n p_i^2 
\]

Con la supposizione che $S_x$ contenga $k$ classi e che $p_i$ sia la probabilità che una tupla scelta a caso da $S_X$ appartenga alla classe $i$, si ha che il \textbf{Gini Index} dello split è defininito come:
\[
\mathrm{Gini_{split}}(S_X) = 1 - \sum_{i=1}^k \frac{|S_i|}{|S_x|}Gini(S_i) 
\]

CART seleziona l’attributo/soglia che \emph{minimizza} $\mathrm{GiniSplit}$. Su attributi categoriali si cercano partizioni in due sottoinsiemi di valori; su continui, soglie come in C4.5.

% --------------------------------------------------------------------

\subsection{Pruning degli alberi}\label{subsec:pruning}
Alberi molto profondi generalizzano male (generano \emph{overfitting}). Per evitare questo, si effettua un \textbf{pruning}, ovvero si riduce la dimensione dell'albero sostituendo un sottoalbero con una foglia etichettato con la classe maggioritaria delle tuple nel sottoalbero, il pruning inserisce però un tasso di errore, si fa solo se necessario. Si usano due strategie principali:
\begin{description}
  \item[Pre-pruning] in fase di costruzione dell'albero si interrompe la crescita quando la goodness dello split è al di sopra di una \emph{soglia}.
  \item[Post-pruning] si costruisce l'albero completo e poi lo si riduce valutando l'errore su validation set o tramite stima incrociata. Generalmente è più dispensioso ma più efficace.
\end{description}

\subsubsection*{Pruning pessimistico (C4.5)}
Confronta l’errore stimato del \emph{sottoalbero} $T$ radicato in $X$ con l’errore stimato della \emph{foglia} che sostituisce $T$ (classe maggioritaria in $X$).

Sia \(X\) un nodo dell’albero con insieme di esempi \(S_x\) (\(N=|S_x|\)) e classe di maggioranza \(C\).
Sia \(T\) il sottoalbero radicato in \(X\) e siano \(x_1,\dots,x_k\) i figli immediati di \(X\), con
\(S_{x_i}\) gli esempi nel figlio \(x_i\) e \(C_i\) la sua classe di maggioranza. Le due quantità
\[
\begin{aligned}
E_p(T)   &= \frac{\bigl|\{\, t \in S_x \mid \mathrm{class}(t) \neq C \,\}\bigr| + \epsilon}{\lvert S_x\rvert}\\[4pt]
E_p'(T) &= \frac{\displaystyle \sum_{i=1}^{k} \bigl|\{\, t \in S_{x_i} \mid \mathrm{class}(t) \neq C_i \,\}\bigr| + k\,\epsilon}{\lvert S_x\rvert}
\end{aligned}
\]
sono le \textbf{stime del tasso di errore} usate per decidere se fare pruning.

\paragraph{Che cosa misurano.}
\begin{itemize}
  \item \(E_p(T)\) è l’\emph{errore stimato} se \textbf{si pota} \(T\) sostituendo l’intero sottoalbero con \emph{una sola foglia} etichettata con la classe di maggioranza \(C\) del nodo \(X\). Il numeratore conta le istanze di \(S_x\) che verrebbero sbagliate da tale foglia, con una \emph{correzione} \(\epsilon\) (tipicamente \(\epsilon=\tfrac12\)) per evitare stime troppo ottimistiche su campioni piccoli.
  \item \(E_p'(T)\) è l’\emph{errore stimato} se \textbf{si mantiene lo split corrente} di \(X\) nei suoi \(k\) figli, ma \emph{troncando} ognuno di essi a foglia (ognuna etichettata con la propria maggioranza \(C_i\)). Si sommano gli errori dei \(k\) figli e si aggiunge una correzione \(\epsilon\) per \emph{ciascuna} foglia (\(k\epsilon\)).
\end{itemize}

\paragraph{Decisione di pruning.}
Confrontando \(E_p(T)\) ed \(E_p'(T)\):
\begin{itemize}
  \item per \(E_p(T) \le E_p'(T)\), \emph{potare} il nodo \(X\) è preferibile, poiché l’errore stimato come foglia è minore o uguale a quello del sottoalbero.
  \item per \(E_p(T) > E_p'(T)\), conviene \emph{mantenere} lo split, in quanto l’errore stimato del sottoalbero è inferiore a quello della singola foglia.
\end{itemize}

Il valore $\epsilon$ è una sorta di “costo fisso” per ogni foglia aggiunta all’albero. L’aggiunta di questo valore agisce da \emph{regolarizzatore}: penalizza strutture con molte foglie, evitando che piccole fluttuazioni del campione giustifichino split inutili.

\subsubsection*{Cost–complexity pruning (CART)}
Si valuta il vantaggio dello \emph{split} di un nodo \(X\) confrontando la riduzione di \emph{error rate} con l’aumento di complessità (nuove foglie).

\paragraph{Errore prima e dopo lo split.}
Sia \(S_X\) l’insieme dei campioni che arrivano al nodo \(X\) e \(C\) la classe maggioritaria in \(S_X\).
\[
E(X)=\frac{\bigl|\{\,t\in S_X:\ \mathrm{class}(t)\neq C\,\}\bigr|}{|S_X|}.
\]
Se \(X\) viene diviso in \(k\) figli \(X_1,\dots,X_k\) (con classi maggioritarie \(C_1,\dots,C_k\)), l’errore \emph{atteso dopo} lo split è
\[
E'(X)=\frac{\sum_{i=1}^{k}\bigl|\{\,t\in S_{X_i}:\ \mathrm{class}(t)\neq C_i\,\}\bigr|}{|S_X|}.
\]

\paragraph{Indice di costo–complessità per lo split.}
Definiamo il guadagno medio per foglia aggiunta:
\[
\alpha(X)=\frac{E(X)-E'(X)}{k-1}.
\]

Il valore \(\alpha\) misura \emph{quanto} diminuisce l’errore per ogni foglia extra introdotta dallo split. Se \(\alpha\) è sufficientemente piccolo, ovvero quando $\alpha$ è minore di una soglia prefissata \(\alpha_0\), lo split non è conveniente e si pota il nodo \(X\).

\begin{figure}[htbp]
  \centering
  \includegraphics[width=.75\textwidth]{images/tree_pruning_example.png}
  \caption{Pruning: confronto errore stimato del sottoalbero vs foglia (C4.5) e principio costo–complessità (CART).}
  \label{fig:pruning}
\end{figure}

\paragraph{Pro/contro degli alberi decisionali.}
\emph{Pro:} interpretabili, veloci in predizione, gestiscono mix di attributi (continui/categoriali), poca preparazione dei dati. \emph{Contro:} instabili rispetto a piccole variazioni dei dati, propensi all’overfitting, separazioni per soglie assiali (forme complesse richiedono molti nodi), accuratezza spesso inferiore a ensemble o SVM su dati ad alta dimensionalità.

% --------------------------------------------------------------------

\section{Classificatori generativi}\label{sec:generative}
I modelli generativi producono un \textbf{modello probabilistico} a partire dai dati, predicendo la \textbf{classe} di appartenenza \emph{più probabile} per un nuovo dato a partire dal modello sviluppato. Questi modelli si basano sul \textbf{teorema di Bayes}.

\subsection{Teorema di Bayes e regola di decisione}\label{subsec:bayes-rule}
Sia $\mathbf{x}$ un’osservazione e $c\in\mathcal{C}$ una classe candidata. Per decidere la
classe usiamo il \textbf{teorema di Bayes}:
\[
P(c\mid \mathbf{x})=\frac{P(\mathbf{x}\mid c)\,P(c)}{P(\mathbf{x})}.
\]
\begin{itemize}
  \item \textbf{Probabilità a priori} $P(c)$: quanto la classe $c$ è probabile \emph{prima} di vedere i dati (in pratica: frequenza della classe nel train).
  \item \textbf{Likelihood} $P(\mathbf{x}\mid c)$: quanto è plausibile osservare $\mathbf{x}$ \emph{se} la classe fosse $c$.
  \item \textbf{Evidenza} $P(\mathbf{x})$: probabilità complessiva di osservare $\mathbf{x}$ (uguale per tutte le classi).
\end{itemize}
La decisione ottima \emph{MAP} (Maximum A Posteriori) è
\[
\hat{c}(\mathbf{x})=\arg\max_{c\in\mathcal{C}} P(c\mid \mathbf{x})
=\arg\max_{c\in\mathcal{C}} P(\mathbf{x}\mid c)\,P(c),
\]
poiché $P(\mathbf{x})$ non dipende da $c$ e non influisce sull’$\arg\max$.
\subsection{Naive Bayes}\label{subsec:naive-bayes}
\paragraph{Idea.}
Assumiamo che, fissata la classe \(c\), le feature siano indipendenti (\emph{assunzione naive}). Allora la verosimiglianza fattorizza:
\[
P(\mathbf{x}\mid c)=\prod_{j=1}^{d} P(x_j\mid c).
\]

\paragraph{Regola di decisione (MAP, in scala logaritmica).} Le probabilità condizionali sono molto piccole e un prodotto di tante quantità prossime a 0 può portare problemi di underflow. Per ovviare a questi problemi si considera il \textbf{log-likelihood}:
\[
\hat{c}(\mathbf{x})=\arg\max_{c\in\mathcal{C}}
\Big[\log P(c)+\sum_{j=1}^{d}\log P(x_j\mid c)\Big].
\]

\noindent
Questo si traduce in una somma anziché in un prodotto di termini.

\paragraph{Stima essenziale delle probabilità.}
\begin{itemize}
  \item \textbf{Prior} \(P(c)\): frequenza della classe nel training.
  \item \textbf{Attributi discreti}: frequenze condizionate con \emph{Laplace smoothing} \((+\alpha)\) per evitare zeri.
  \item \textbf{Attributi continui}: modello gaussiano per \(x_j\mid c\) con media e varianza stimate sui dati della classe.
\end{itemize}

\paragraph{Vantaggi e svantaggi.} Molto veloce e facile da implementare, l'assunzione di indipendenza condizionale potrebbe non essere sempre vera e potrebbe portare ad una perdita di accuratezza (tali dipendenze non possono essere modellate da questo modello).
% --------------------------------------------------------------------

\subsection{Reti Bayesiane}\label{subsec:bayesnet}
Una \textbf{rete bayesiana} è un DAG le cui variabili $\{X_1,\dots,X_d\}$ fattorizzano come
\[
P(X_1,\dots,X_d)=\prod_{i=1}^d P(X_i\mid \mathrm{Pa}(X_i)),
\]
dove $\mathrm{Pa}(X_i)$ sono i genitori di $X_i$ nel grafo. Il DAG codifica indipendenze condizionali; i CPT (tabelle di probabilità condizionate) specificano i parametri.

\begin{figure}[htbp]
  \centering
  \includegraphics[width=.9\textwidth]{images/bayes_network_example.png}
  \caption[Bayes net Sprinkler–Rain–Grass]{Rete bayesiana \emph{Sprinkler–Rain–Grass}: il DAG orientato specifica le dipendenze e indetermina la fattorizzazione della congiunta
  \(P(\text{Rain})\,P(\text{Sprinkler}\mid \text{Rain})\,P(\text{GrassWet}\mid \text{Sprinkler},\text{Rain})\).
  Le tabelle mostrano le CPD (Conditional Probability Tables) dei nodi.
  A differenza di Naive Bayes, le feature possono essere dipendenti dato la/e causa/e (qui \textit{Rain}), e tale dipendenza è resa esplicita dagli archi.}
  \label{fig:bayes-net}
\end{figure}

\paragraph{Uso per la classificazione.}
Dato $\mathbf{x}$, si calcolano (o si approssimano) $P(y\mid \mathbf{x})$ tramite inferenza sul DAG (\emph{esatta} o \emph{approx} con sampling/variational). Le reti bayesiane generalizzano Naive Bayes (che è un caso particolare con $Y$ genitore di tutte le feature e nessun’altra dipendenza).

\section{Classificatori discriminativi}\label{sec:discriminativi}
I classificatori discriminativi stimano direttamente una funzione di decisione $f:\mathbb{R}^d\to\mathbb{R}$ (o, opzionalmente, la probabilità condizionata $P(y\mid\mathbf{x})$) senza modellare la distribuzione congiunta $P(\mathbf{x},y)$. Dato un nuovo esempio $\mathbf{x}$ si valuta $f(\mathbf{x})$ e si assegna l'etichetta corrispondente. Rispetto ai modelli generativi richiedono generalmente meno assunzioni sui dati. Tipici esempi sono il Perceptron e le Support Vector Machines (SVM).

\subsection{Classificazione lineare e non lineare}\label{subsec:lin-nonlin}

\begin{description}
  \item[Lineare:] la regola di decisione è basata su una combinazione lineare degli attributi
  \[
  f(\mathbf{x})=\mathbf{w}\cdot\mathbf{x}+b,
  \]
  Dove $\mathbf{w}\in\mathbb{R}^d$ è il vettore dei pesi e $b\in\mathbb{R}$ è il bias (termine di soglia). 
  \item[Non lineare:] Il problema di avere spazi non lineare è che non è possibile separare le classi con un iperpiano. Per risolvere questo problema si fa una trasformazioni di spazi vettoriali in spazi di dimensione superiore dove la separazione lineare è possibile. Questo si ottiene tramite il \textbf{kernel trick}, che permette di calcolare prodotti scalari in spazi trasformati senza dover esplicitamente mappare i dati.
\end{description}

\noindent
\textit{Esistono anche classificazioni lineari e non lineari binarie, dove si utilizzano approcci simili}.

\subsection{Perceptron}\label{subsec:perceptron}
\paragraph{Definizione.}
Il Perceptron è un classificatore lineare binario che produce la funzione
\[f(\mathbf{x})=\mathbf{w}\cdot\mathbf{x}+b.\]
Con etichette $y\in\{-1,+1\}$. Si parla di un algoritmo di machine learning, quindi è necessario un modo, per il perceptron, di apprendere i parametri $\mathbf{w}$ e $b$ dai dati di addestramento.

\paragraph{Regola di aggiornamento.}
Dato un esempio $(\mathbf{x}^{(i)},y^{(i)})$, se la previsione $\hat{y}^{(i)}$ è errata (ossia $\hat{y}^{(i)}\neq y^{(i)}$) si aggiorna ogni peso component-wise secondo la notazione usata in figura:
\[
\hat{\vartheta}_j \leftarrow \vartheta_j + \alpha\,\bigl(y^{(i)}-\hat{y}^{(i)}\bigr)\,x_j^{(i)},\qquad j=0,\dots,d,
\]
dove $\alpha>0$ è il learning rate, $\vartheta_j$ indica il valore corrente del peso e $\hat{\vartheta}_j$ il valore aggiornato. Equivalentemente, in forma vettoriale si ottiene
\[
\mathbf{w}\leftarrow \mathbf{w} + \alpha\,\bigl(y^{(i)}-\hat{y}^{(i)}\bigr)\,\mathbf{x}^{(i)},\qquad b\leftarrow b + \alpha\,\bigl(y^{(i)}-\hat{y}^{(i)}\bigr).
\]
Se $\hat{y}^{(i)}=y^{(i)}$ il modello non viene modificato.

\paragraph{Proprietà.}
Se i dati sono linearmente separabili, il Perceptron converge in un numero finito di aggiornamenti (teorema di Novikoff). In pratica si itera per più epoche o fino a soddisfare un criterio di stop.

\paragraph{Algoritmo.}
L'algoritmo del perceptron è definito come:
\begin{enumerate}
  \item Inizializza i pesi $\mathbf{w}$ e il bias $b$ a zero o a valori casuali.
  \item Per ogni tupla $y_j$ nel training set:
  \begin{enumerate}
    \item Calcola la previsione $\hat{y}^{(i)}$
    \item Se $\hat{y}^{(i)}\neq y^{(i)}$, aggiorna i pesi e il bias secondo la regola di aggiornamento.
    \item Incrementa $i$ e ripeti fino a completare il training set.
  \end{enumerate}
  \item Ripeti il passo 2 per un numero prefissato di epoche o fino a soddisfare un criterio di stop. Nel caso di learning offline, si ripete il passo 2 finché l'errore medio di classificazione sul training set non scende sotto una soglia prefissata.
\end{enumerate}

\paragraph{One–Vs–One (OVO).}
One–Vs–One costruisce un classificatore binario Perceptron per ogni coppia di classi $(C_p,C_q)$; per $K$ classi si addestrano $K(K-1)/2$ modelli. In fase di predizione, ogni classificatore vota per una delle due classi che confronta e si assegna la classe con il maggior numero di voti (voting).

Motivazione: OVO è utile quando le classi sono relativamente poche e si desidera che ogni modello risolva una decisione binaria semplice; ogni modello vede dati di due sole classi, spesso permettendo separazioni più semplici e modelli più piccoli. Lo svantaggio principale è il numero di classificatori e la gestione del voto/pareggio.

\noindent
Algoritmo (per ciascuna coppia $(C_p,C_q)$):
\begin{enumerate}
  \item Costruisci il training set rimuovendo istanze non appartenenti a $C_p$ o $C_q$.
  \item Inizializza i pesi $\vartheta_j$ e il bias.
  \item Per ogni epoca e per ogni esempio $(\mathbf{x}^{(i)},y^{(i)})$ nel sottoinsieme: calcola $\hat{y}^{(i)}=\operatorname{sign}(f(\mathbf{x}^{(i)}))$; se $\hat{y}^{(i)}\neq y^{(i)}$ aggiorna
  \[\hat{\vartheta}_j\leftarrow\vartheta_j+\alpha\,(y^{(i)}-\hat{y}^{(i)})\,x_j^{(i)}\quad(j=0,\dots,d).\]
\end{enumerate}

\paragraph{One–Vs–All (OVA).}
One–Vs–All costruisce un classificatore Perceptron per ogni classe $C_k$ dove il problema è $C_k$ vs "resto". Si addestrano $K$ modelli; alla predizione si calcola il punteggio $f_k(\mathbf{x})$ per ogni modello e si sceglie la classe con il punteggio più alto.

Motivazione: OVA è più parsimonioso in termini di numero di modelli rispetto a OVO (si addestrano $K$ modelli invece di $K(K-1)/2$) e può essere più efficiente quando $K$ è grande. Tuttavia ogni modello OVA affronta un problema sbilanciato (una classe vs tutte le altre), il che può richiedere tecniche di bilanciamento o regolarizzazione.

\noindent
Algoritmo (per ciascuna classe $C_k$):
\begin{enumerate}
  \item Crea etichette binarie $y^{(i)}=+1$ se l'esempio appartiene a $C_k$, altrimenti $y^{(i)}=-1$.
  \item Inizializza i pesi $\vartheta_j$ e il bias.
  \item Per ogni epoca e per ogni esempio $(\mathbf{x}^{(i)},y^{(i)})$: calcola $\hat{y}^{(i)}=\operatorname{sign}(f(\mathbf{x}^{(i)}))$; se $\hat{y}^{(i)}\neq y^{(i)}$ aggiorna
  \[\hat{\vartheta}_j\leftarrow\vartheta_j+\alpha\,(y^{(i)}-\hat{y}^{(i)})\,x_j^{(i)}\quad(j=0,\dots,d).\]
\end{enumerate}

\noindent
\textit{Breve nota comparativa: OVO tende a produrre modelli più specialistici e può funzionare meglio quando le classi sono ben separate a coppie; OVA è più semplice ed efficiente per molti problemi pratici ma richiede attenzione allo sbilanciamento delle classi}.

\subsection{Support Vector Machines (SVM)}\label{subsec:svm}
Le SVM hanno un'idea diversa dal perceptron: cercano di trovare l'iperpiano che massimizza il \textbf{margine} tra le classi, ovvero la distanza minima tra l'iperpiano e i punti dati più vicini di ciascuna classe (i \emph{support vectors}).

\begin{figure}[htbp]
  \centering
  \includegraphics[width=.9\textwidth]{images/svm_margin.png}
  \caption{Esempio di Support Vector Machine che massimizza il margine tra due classi. I punti cerchiati sono i support vectors che definiscono l'iperpiano ottimale.}
  \label{fig:svm-margin}
\end{figure}

Per dati linearmente separabili, esistono infinite iperpiani che separano le classi; le SVM scelgono quello con il margine massimo, che tende a generalizzare meglio su dati non visti.

\paragraph{Formulazione del problema.}
Sia $D$ un dataset di addestramento formato da punti \((\mathbf{x}^{(i)},y^{(i)})\), con etichette \(y^{(i)} \in \{-1,+1\}\), pesi \(\mathbf{w}\) e bias \(b\). L'iperpiano di decisione è definito da:
\[
f(\mathbf{x}) = \mathbf{w} \cdot \mathbf{x} + b.
\]
Se indichiamo con $\gamma$ il margine, l'obiettivo delle SVM è trovare i parametri $w, b$ che massimizzano $\gamma$. Indicando con
\begin{align*}
  &H_1: \mathbf{w} \cdot \mathbf{x} + b = \gamma \\
  &H_2: \mathbf{w} \cdot \mathbf{x} + b = -\gamma
\end{align*}
le equazioni dei due iperpiani paralleli che definiscono il margine, passanti per i vettori di supporto delle due classi. Da questo si può dedurre che i punti di classe $+1$ soddisfano la disuguaglianza \(\mathbf{w} \cdot \mathbf{x}^{(i)} + b \geq \gamma\), mentre i punti di classe $-1$ soddisfano \(\mathbf{w} \cdot \mathbf{x}^{(i)} + b \leq -\gamma\). 

Combinando queste due condizioni, si ottiene la seguente disuguaglianza per tutti i punti del dataset:
\[
y^{(i)}(\mathbf{w} \cdot \mathbf{x}^{(i)} + b) \geq \gamma, \quad \forall i.
\]
Poiché $y^{(i)}$ può essere $+1$ o $-1$, questa condizione assicura che ogni punto sia correttamente classificato e si trovi al di fuori del margine. Tuttavia si può riscrivere la formulazione in modo più semplice, fissando \(\gamma = 1\), il che porta alla condizione:
\[
y^{(i)}(\mathbf{w} \cdot \mathbf{x}^{(i)} + b) \geq 1, \quad \forall i.
\]

Purtroppo questa formulazione "libera" in termini di $\gamma$ non è direttamente utilizzabile: se $(\mathbf{w},b)$ soddisfa i vincoli, allora anche $(\lambda\mathbf{w},\lambda b)$ soddisfa i vincoli per ogni scalare $\lambda>1$ e fornisce un margine $\lambda\gamma$ più grande. Di conseguenza non esiste un massimo finito per $\gamma$ senza imporre una normalizzazione aggiuntiva. Per evitare questa degenerazione si fissa implicitamente la scala del vettore dei pesi fissando il valore del margine (ad es. $\gamma=1$) e si minimizza la norma di $\mathbf{w}$, che è equivalente a massimizzare il margine in modo ben definito. La soluzione del problema è quella di normalizzare il vettore dei pesi in modo che il margine sia fissato a 1, quindi trasformare il vettore $\mathbf{w}$ in un vettore:
\[
\hat{\mathbf{w}} = \frac{\mathbf{w}}{\|\mathbf{w}\|}.
\]

Il vettore $\hat{\mathbf{w}}$ è il vettore unità (modulo pari a 1) dei pesi normalizzato con stessa direzione di $\mathbf{w}$. Inoltre, poiché ha la stessa direzione è perpendicolare all'iperpiano separatore:

\begin{figure}[htbp]
  \centering
  \includegraphics[width=.8\textwidth]{images/normalized_hyperplane_svm.png}
  \caption{Iperpiani con pesi $\mathbf{w}$ normalizzati per margine unitario.}
  \label{fig:svm-margin-normalized}
\end{figure}

Indicando con $x_2$ un vettore di supporto che giace su $H_2$ e $x_1$ la sua proiezione in $H_1$ (figura \ref{fig:svm-margin-normalized}), la distanza tra i due iperpiani è data da:
\[
x_1 = x_2 + 2 \gamma = x_2 + 2 \gamma \frac{\mathbf{w}}{\|\mathbf{w}\|}
\]

\noindent
Possiamo ricalcolare l'equazione dell'iperpiano come:
\begin{align*}
&\mathbf{w}\cdot\bigl(x_2 + 2\gamma \frac{\mathbf{w}}{\|\mathbf{w}\|}\bigr) + b = 0 \\
\Rightarrow\quad &\mathbf{w}\cdot x_2 + b + 2\gamma \|\mathbf{w}\| = 0
\end{align*}

Osservando che $x_2$ giace su $H_2$ abbiamo $\mathbf{w}\cdot x_2 + b = -\gamma$. Sostituendo nell'ultima equazione:
\[
-\gamma + 2\gamma\,\|\mathbf{w}\| = 0 \quad\Longrightarrow\quad 2\gamma\,\|\mathbf{w}\| = \gamma.
\]
Se $\gamma\neq 0$ segue
\[
\|\mathbf{w}\| = \tfrac{1}{2},
\]
ma ricordando che nella definizione del margine la distanza tra le due hyperplane considerate è $2\gamma$, fissando la normalizzazione standard si ricava l'identità
\[
\gamma = \frac{1}{\|\mathbf{w}\|},
\]
da cui la conclusione voluta: massimizzare $\gamma$ equivale a minimizzare $\|\mathbf{w}\|$ (e quindi, per praticità numerica, si minimizza spesso $\tfrac{1}{2}\|\mathbf{w}\|^2$).

Il problema di ottimizzazione da risolvere, quindi, è:
\[
\begin{cases}
\displaystyle\min\;\|\mathbf{w}\|\\[6pt]
	y^{(i)}(\mathbf{w}\cdot\mathbf{x}^{(i)}+b)\ge 1\quad\forall\;1\le i\le n
\end{cases}
\]
che si può risolvere applicando il metodo di discesa del gradiente.

Nella pratica, per rendere più semplice il calcolo dei gradienti si preferisce considerare il quadrato della norma:
\[
\begin{cases}
\displaystyle\min\;\tfrac{1}{2}\,\|\mathbf{w}\|^2\\[6pt]
	y^{(i)}(\mathbf{w}\cdot\mathbf{x}^{(i)}+b)\ge 1\quad\forall\;1\le i\le n
\end{cases}
\]

\paragraph{SVM Soft margin.} Esiste un problema nella classificazione di SVM chiamato \textbf{Hard Margin}: ovvero una classificazione dei dati troppo precisa, in quanto non può ricadere all'interno dell'intervallo dei due iperpiani. Una variante si chiama \textbf{Soft Margin}, che permette ad alcuni punti di cadere all'interno del margine o addirittura di essere classificati in modo errato, introducendo delle variabili di slack $\xi_i\ge 0$ per ogni punto dati. 

Per riformulare il problema, si introducono $n$ variabili $\epsilon_1, \epsilon_2, ..., \epsilon_n$ tale che, quando $\epsilon_i = 0$, il punto ricade al di fuori del margine e soddisfa la relazione:
\[
y^{(i)}(\mathbf{w}\cdot\mathbf{x}^{(i)}+b)\ge 1,
\]

\noindent
Altrimenti se $\epsilon_i > 0$, il punto ricade all'interno del margine o è classificato in modo errato, e la relazione diventa:
\[
y^{(i)}(\mathbf{w}\cdot\mathbf{x}^{(i)}+b)\ge 1 - \epsilon_i.
\]

Da qui si introduce un parametro $\lambda$ di \textbf{trade-off}\footnote{Il parametro di trade-off $\lambda$ bilancia l'importanza tra la massimizzazione del margine e la minimizzazione dell'errore di classificazione. Un valore più alto di $\lambda$ dà più peso alla riduzione degli errori, mentre un valore più basso enfatizza la massimizzazione del margine.} tra la minimizzazione della norma di $\mathbf{w}$ e la penalizzazione degli errori di classificazione, portando alla seguente formulazione del problema di ottimizzazione:
\[
\begin{cases}
\displaystyle\min\;\biggl(\frac{1}{n}\sum_{i=1}^n\epsilon_i + \lambda\,\|\mathbf{w}\|^2\biggr)\\[6pt]
\qquad y^{(i)}\bigl(\mathbf{w}\cdot\mathbf{x}^{(i)}+b\bigr)\ge 1-\epsilon_i, & \forall\; i=1,\dots,n\\[4pt]
\qquad \epsilon_i\ge 0, & \forall\; i=1,\dots,n
\end{cases}
\]

\noindent
Più piccolo è $\lambda$, più trascurabile è $\|\mathbf{w}\|$, meno importante è la dimensione del margine.

Il problema può essere risolto anche considerando il suo duale tramite la Lagrangiana. Introducendo i moltiplicatori di Lagrange $\alpha_i$ per i vincoli di classificazione (e $\mu_i$ per i vincoli $\epsilon_i\ge0$) si ottiene una funzione quadratica in $\alpha_i$, quindi il duale è una QP risolvibile con algoritmi dedicati. Ricavati i moltiplicatori $\alpha_i$, il vettore dei pesi si esprime come
\[
\mathbf{w}=\sum_{i=1}^n \alpha_i\,y^{(i)}\,\mathbf{x}^{(i)}.
\]
Il bias $b$ si ricava:
\[
y^{(i)}(\mathbf{w}\cdot\mathbf{x}^{(i)}+b)=1 \quad\Longrightarrow\quad b=\frac{1}{y^{(i)}} - \mathbf{w}\cdot\mathbf{x}^{(i)}.
\]

Nel caso di dati non linearmente separabili si effettua un \emph{mapping} in uno spazio di dimensione superiore tramite una funzione $\phi:\mathbb{R}^d\to\mathbb{R}^D$ (con $D>d$) e si applica SVM in tale spazio. Quindi, creiamo un nuovo problema di ottimizzazione dove al posto di $\mathbf{x}^{(i)}$ usiamo $\phi(\mathbf{x}^{(i)})$:
\[
\begin{cases}
\displaystyle\max\;\Biggl(\sum_{i=1}^n c_i - \tfrac{1}{2}\sum_{i=1}^n\sum_{j=1}^n y^{(i)} c_i\,\bigl(\phi(\mathbf{x}^{(i)})\!\cdot\!\phi(\mathbf{x}^{(j)})\bigr)\,y^{(j)} c_j\Biggr)\\[6pt]
\qquad \displaystyle\sum_{i=1}^n c_i\,y^{(i)} = 0\\[6pt]
\qquad 0 \le c_i \le \frac{1}{2n\lambda},\quad i=1,\dots,n
\end{cases}
\]

\paragraph{Kernel Trick.} Il prodotto scalare \(\phi(\mathbf{x}^{(i)})\cdot\phi(\mathbf{x}^{(j)})\) è \textbf{dispendioso} da calcolare. Per questo motivo, si usa una \textbf{funzione kernel $K$} tale che soddisfi tale condizione:
\[
K(\mathbf{x}^{(i)},\mathbf{x}^{(j)}) = \phi(\mathbf{x}^{(i)})\cdot\phi(\mathbf{x}^{(j)}).
\]

Grazie a questa funzione, si definisce il mapping del nuovo spazio permettendo di sostituire il prodotto scalare con la funzione kernel, risparmiando tempo di calcolo (esempi di kernel sono il kernel polinomiale, gaussiano e sigmoide).

\paragraph{SVM Multi-classe.} SVM, come il perceptron, può essere esteso a problemi di classificazione multi-classe usando strategie come One–Vs–One (OVO) o One–Vs–All (OVA), addestrando più modelli binari e combinando le loro predizioni per ottenere la classificazione finale.

\section{Apprendimento Lazy}
Fino a qui abbiamo visto tipi di apprendimento chiamati \textbf{eager}, in cui il modello viene costruito durante la fase di training.  Esiste un secondo tipo di apprendimento chiamato \textbf{lazy}, dove il modello memorizza un training set di dati e calcola la funzione di classificazione localmente per ogni nuovo dato.

La funzione di predizione è quindi approssimata \emph{localmente}, inoltre questi metodi funzionano su grandi dataset con pochi attributi e che si aggiornano continuamente.

\subsection{K-Nearest Neighbor}\label{subsec:knn}
K-Nearest Neighbor (KNN) è un algoritmo di classificazione lazy che assegna la classe di un nuovo esempio basandosi sulle classi dei suoi \(k\) vicini più prossimi nel training set.

\begin{figure}[htbp]
  \centering
  \includegraphics[width=.8\textwidth]{images/knn_example.png}
  \caption{Esempio di classificazione con K-NN (k=5). Il punto da classificare (in verde) è circondato dal cerchio che indica i $k$ vicini più prossimi: tra i 5 vicini ci sono tre triangoli rossi e due quadrati blu, quindi per maggioranza il punto viene assegnato alla classe rossa.}
  \label{fig:knn-example}
\end{figure}

\paragraph{Algoritmo.}
\begin{enumerate}
  \item Memorizza il training set \(D=\{(\mathbf{x}^{(i)},y^{(i)})\}_{i=1}^{n}\).
  \item Per ogni nuovo esempio \(\mathbf{x}\):
  \begin{enumerate}
    \item Calcola la distanza tra \(\mathbf{x}\) e ogni esempio in \(D\) (es. distanza euclidea).
    \item Identifica i \(k\) esempi più vicini.
    \item Assegna a \(\mathbf{x}\) la classe più frequente tra i \(k\) vicini.
  \end{enumerate}
\end{enumerate}

\paragraph{Varianti.}
Esistono alcune varianti di KNN:
\begin{itemize}
  \item \textbf{Distanza pesata:} i punti a distanza più vicina hanno un peso maggiore nel voto.
  \item \textbf{K-NN per valori continui:} Assegna agli attributi ignoti della tupla da classificare la media dei valori degli attributi delle tuple più vicine (rispetto agli attributi noti).
\end{itemize}

\section{Ensemble Learning}\label{sec:ensemble}
L'ensemble learning ha lo scopo di combinare più modelli, solitamente omogenei, per migliorare le prestazioni di classificazione rispetto a un singolo modello. L'idea è che combinando le predizioni di più modelli si possa ridurre l'errore complessivo, sfruttando la diversità tra i modelli. Poiché richiede molta potenza di calcolo, si combinano modelli semplici e veloci da addestrare.

\subsection{Bagging}
Il paradigma \emph{Bagging} consiste nell'addestrare diversi classificatori in parallelo, combinando le predizioni finali.

\paragraph{Algoritmo.}
Siano $M_1, M_2, ..., M_k$ i $k$ modelli da combinare:
\begin{enumerate}
  \item Addestra ciascun modello $M_i$ su un sottoinsieme casuale del training set o su un campione di dati estratto tramite bootstrapping\footnote{Il bootstrapping è una tecnica di campionamento con reinserimento che consente di creare più sottoinsiemi di dati a partire da un dataset originale. Ogni sottoinsieme può contenere duplicati e viene utilizzato per addestrare modelli diversi nell'ensemble learning.}.
  \item Combina opportunamente le predizioni dei modelli:
  \begin{itemize}
    \item \textbf{Classificazione:} voto di maggioranza tra le classi predette.
    \item \textbf{Regressione:} media delle predizioni.
  \end{itemize}
\end{enumerate}

\paragraph{Random Forest.} Una Random Forest combina molti alberi decisionali addestrati su sottoinsiemi casuali del training set e su sottoinsiemi casuali di feature. Ogni albero viene addestrato su un campione bootstrap del dataset e, ad ogni split, si considera solo un sottoinsieme casuale delle feature per determinare la migliore divisione. La predizione finale viene effettuata tramite voto di maggioranza tra gli alberi.

\begin{figure}[htbp]
  \centering
  \includegraphics[width=.75\textwidth]{images/random_forest.png}
  \caption{Schema di Random Forest: dai dati di training $(X)$ si costruiscono $k$ alberi decisionali $T_1(X,\alpha_1)$, $T_2(X,\alpha_2)$, $\dots$, $T_k(X,\alpha_k)$, ciascuno con parametri $\alpha_i$ diversi. La predizione finale è ottenuta combinando i risultati attraverso voting o averaging ($\sum T_i(X,\alpha_i)$).}
  \label{fig:random-forest}
\end{figure}

Il vantaggio è quello di usarlo su dataset di dati molto predittivi, in quanto riduce la correlazione tra gli alberi e migliora la generalizzazione.

\subsection{Boosting}
Il \emph{Boosting} è un paradigma di ensemble learning che costruisce sequenzialmente una serie di modelli deboli, dove ogni modello successivo si concentra sugli errori commessi dai modelli precedenti. 

\paragraph{Algoritmo.}
Siano $M_1, M_2, ..., M_k$ i $k$ modelli da combinare:
\begin{enumerate}
  \item Inizializza i pesi delle istanze del training set in modo uniforme.
  \item Per ogni modello $M_i$:
  \begin{enumerate}
    \item Addestra $M_i$ sul training set ponderato.
    \item Calcola l'errore di $M_i$ e aggiorna i pesi delle istanze: aumenta i pesi delle istanze mal classificate e diminuisce quelli delle istanze correttamente classificate.
  \end{enumerate}
  \item Combina le predizioni dei modelli, pesando ciascuna predizione in base alla sua accuratezza (anche qui, nel caso di classificazione si usa il voto ponderato, mentre nella regressione si usa una media ponderata).
\end{enumerate}

\begin{figure}[htbp]
  \centering
  \includegraphics[width=.9\textwidth]{images/boosting_example.png}
  \caption{Schema di Boosting: ogni classificatore viene addestrato su dati pesati in base agli errori dei modelli precedenti. I dati mal classificati ricevono peso maggiore, così i modelli successivi si concentrano sulle istanze più difficili. Le predizioni finali sono combinate in un ensemble ponderato.}
  \label{fig:boosting-example}
\end{figure}

\subsection{Adaboost} 
In questo algoritmo si usano solitamente \textbf{alberi decisionali} con due foglie chiamati \emph{stumps}. Combinando opprtunamente tali stumps, che sono semplici e veloci, si ottiene un classificatore accurato.

\begin{enumerate}
  \item Inizializza i pesi delle istanze del training set in modo uniforme.
  \item Per $i=1$ a $k$:
  \begin{enumerate}
    \item Crea lo stump $M_i$ che minimizza l'errore ponderato sul training set.
    \item Per lo stump $M_i$, calcola un peso $P_i$ basato sulla sua accuratezza:
    \item Aumenta i pesi delle tuple classificate in modo errato e diminuisci quelli delle tuple classificate correttamente, dopodiché normalizza tra 0 e 1 i nuovi pesi.
  \end{enumerate}
  \item Combina linearmente le predizioni dei modelli usando i pesi $P_i$.
\end{enumerate}

\paragraph{Gini Index Pesato.}
Sia $D$ un \emph{dataset} con $N$ tuple e $k$ classi $C_1, C_2, \dots, C_k$ e sia $w(\mathbf{x})$ il peso associato alla tupla $\mathbf{X}$. Indico con $T_{C_i}$ l'insieme delle \textbf{tuple} di $D$ aventi \textbf{classe} $C_i$. La probabilità di osservare una tupla di classe $C_i$ è:
\[
p_i = \frac{\sum_{\mathbf{X} \in T_{C_i}} w(\mathbf{X})}{\sum_{\mathbf{Y} \in D} w(\mathbf{Y})}
\]
Il \textbf{Gini index} dello splitting è dato da:
\[
gini_{split}(S_X) = \sum_{i=1}^k \frac{|S_i|}{|S_X|} \; gini(S_i)
\qquad\qquad
  gini(S_i) = 1 - \sum_{i=1}^n p_i^2
\]

\paragraph{Peso dello stump.} 
Considerando TotalError la somma dei pesi delle tuple classificate erroneamente dallo stump $M_i$, il peso $P_i$ dello stump si calcola come:
\[
P_i = \frac{1}{2} \log\biggl(\frac{1 - \text{TotalError}}{\text{TotalError}}\biggr).
\]

\paragraph{Aggiornamento dei pesi.}
Sia $w^{(t)}(\mathbf{X})$ il peso della tupla $\mathbf{X}$ all'iterazione $t$. Dopo aver calcolato lo stump $M_i$ e il suo peso $P_i$, si aggiorna il peso della tupla come:

\noindent
Sia $w^{(i)}(\mathbf{X})$ il peso della tupla $\mathbf{X}$ al passo $i$-esimo.

\begin{itemize}
  \item Se $\mathbf{X}$ \textbf{non} è classificata correttamente dallo stump $M_i$, il peso \textbf{aumenta}:
  \[
    w^{i+1}(\mathbf{X}) = w^{i}(\mathbf{X}) \cdot e^{w(D_i)}
  \]
  \item Se $\mathbf{X}$ è classificata \textbf{correttamente} dallo stump $M_i$, il peso \textbf{diminuisce}:
  \[
    w^{i+1}(\mathbf{X}) = w^{i}(\mathbf{X}) \cdot e^{-w(D_i)}
  \]
\end{itemize}

\noindent
Esempi che mostrano l'andamento di $e^{w(D_i)}$ e $e^{-w(D_i)}$ in funzione di $w(D_i)$:
\begin{figure}[htbp]
  \centering
  \begin{minipage}[b]{0.48\textwidth}
    \centering
    \includegraphics[width=\linewidth]{images/aumento_pesi_tuple.png}
    \\[4pt]
    {\small Incremento del peso: $e^{w(D_i)}$ in funzione di $w(D_i)$}
  \end{minipage}\hfill
  \begin{minipage}[b]{0.48\textwidth}
    \centering
    \includegraphics[width=\linewidth]{images/diminuzione_pesI_tuple.png}
    \\[4pt]
    {\small Decremento del peso: $e^{-w(D_i)}$ in funzione di $w(D_i)$}
  \end{minipage}
  \caption{Andamento dell'incremento e del decremento dei pesi delle tuple per Adaboost.}
  \label{fig:adaboost-weight-changes}
\end{figure}

\subsection*{Bagging vs Boosting}
Il bagging è utilizzato maggiormente per ridurre \emph{overfitting}, mentre il boosting è più utilizzato quando si vogliono combinare classificatori molto semplice che presi singolarmente hanno bassa accuracy.

\section{Validazione di un classificatore}

\subsection{Matrice di confusione}
La matrice di confusione viene usata per rappresentare l'accuracy dove ogni riga contiene valori reali mentre ogni colonna i valori predetti. Un modello che funziona bene si vede dalle colonne, poiché se hanno valori diversi da 0 allora il modello predice bene la classe.

\begin{figure}[htbp]
  \centering
  \includegraphics[width=0.7\linewidth]{images/confusion_matrix.png}
  \caption{Esempio di matrice di confusione per un classificatore a tre classi (Gatto, Cane, Coniglio): le righe indicano le classi reali, le colonne le classi predette. I numeri nelle celle sono le frequenze assolute; le somme marginali mostrano il totale per riga/colonna. Un buon classificatore presenta valori elevati sulla diagonale principale (corretta assegnazione).}
  \label{fig:confusion-matrix}
\end{figure}

\paragraph{Misure di accuratezza con due classi.}
Si possono considerare due classi "Positiva" $P$ e "Negativa" $N$. Da questo possono nascere:
\begin{itemize}
  \item \textbf{True positive}: tuple di classe $P$ che vengono classificate come $P$.
  \item \textbf{False positive}: tuple di classe $P$ che vengono classificate come $N$.
  \item \textbf{True negative}: tuple di classe $N$ che vengono classificate come $N$.
  \item \textbf{False negative}: tuple di classe $N$ che vengono classificate come $P$.
\end{itemize}


\paragraph{Misure di accuratezza (due classi).}
Per il caso binario, indicando con $\lvert\mathrm{Tpos}\rvert$ il numero di true positive, con
 $\lvert\mathrm{Tneg}\rvert$ il numero di true negative, con $\lvert\mathrm{Fpos}\rvert$ il numero di false
 positive, con $\lvert\mathrm{Fneg}\rvert$ il numero di false negative, e con $\lvert\mathrm{Pos}\rvert,\lvert\mathrm{Neg}\rvert$ i totali veri delle classi positive e negative, si definiscono le misure
pi\`u comuni:
\begin{align*}
&\text{Recall (Rec) / Sensitivity / True Positive Rate (TPR)}:\quad &&\mathrm{Rec}=\frac{\lvert\mathrm{Tpos}\rvert}{\lvert\mathrm{Pos}\rvert},\\
&\text{Specificity (Spec) / True Negative Rate (TNR)}:\quad &&\mathrm{Spec}=\frac{\lvert\mathrm{Tneg}\rvert}{\lvert\mathrm{Neg}\rvert},\\
&\text{False Positive Rate (FPR)}:\quad &&\mathrm{FPR}=\frac{\lvert\mathrm{Fpos}\rvert}{\lvert\mathrm{Neg}\rvert},\\
&\text{False Discovery Rate (FDR)}:\quad &&\mathrm{FDR}=\frac{\lvert\mathrm{Fpos}\rvert}{\lvert\mathrm{Tpos}\rvert + \lvert\mathrm{Fpos}\rvert},\\
&\text{Precision (Prec)}:\quad &&\mathrm{Prec}=\frac{\lvert\mathrm{Tpos}\rvert}{\lvert\mathrm{Tpos}\rvert + \lvert\mathrm{Fpos}\rvert},\\
&\text{Accuracy (Acc)}:\quad &&\mathrm{Acc}=\frac{\lvert\mathrm{Tpos}\rvert + \lvert\mathrm{Tneg}\rvert}{\lvert\mathrm{Pos}\rvert + \lvert\mathrm{Neg}\rvert},\\
&\text{F1 (armonica di Precision e Recall)}:\quad &&F1 = 2\cdot\frac{\mathrm{Prec}\cdot\mathrm{Rec}}{\mathrm{Prec}+\mathrm{Rec}}.
\end{align*}

Queste misure sono utili per valutare diversi aspetti del classificatore: la recall misura la capacit\`a di trovare
le istanze positive, la precision misura la correttezza delle predizioni positive, mentre l'accuracy fornisce una
visione globale. In presenza di sbilanciamento di classe conviene preferire precision/recall o F1 all'accuracy.

\subsection{Soglia discriminativa in un classificatore binario}
In alcuni casi, la distinzione tra due classi (caso binario) si basa su un valore soglia $\sigma$. Ad esempio, predico se una mail è spam oppure no sulla base di tale soglia. Le performance del classificatore sono valutate al variare di $\sigma$ tramite \textbf{curve ROC}. Esse rappresentano il \emph{True Positive Rate (TPR)} in funzione del \emph{False Positive Rate (FPR)}, al variare di $\sigma$.

Se la soglia è molto alta, tutte le tuple sono classificate come \emph{negative}: nessuna tupla negativa è classificata come positiva (\( \mathrm{FPR}=0 \)) e nessuna tupla positiva è classificata come positiva (\( \mathrm{TPR}=0 \)). 

\subsection{Curva ROC}
Se la soglia è molto bassa, tutte le tuple sono classificate come \emph{positive}: tutte le tuple negative sono classificate come positive (\( \mathrm{FPR}=1 \)) e tutte le tuple positive sono classificate come positive (\( \mathrm{TPR}=1 \)). Per valori intermedi di soglia si ottengono valori di TPR e FPR compresi tra 0 e 1.

Situazione ideale: \( \mathrm{TPR} \) aumenta fino a 1 mentre \( \mathrm{FPR} \) si mantiene pari a 0 (\emph{miglior classificatore}). Un \emph{classificatore random} ha sempre valori di \( \mathrm{TPR} \) e \( \mathrm{FPR} \) uguali al variare di $\sigma$.

\begin{figure}[htbp]
  \centering
  \includegraphics[width=.6\linewidth]{images/curve-roc-example.png}
  \caption{Esempio di curva ROC al variare della soglia $\sigma$.}
\end{figure}

\subsection{Curva di Precision-Recall}
A differenza della curva ROC, la \textbf{curva di Precision-Recall} rappresenta la \textbf{Precision} in funzione della \textbf{Recall} al variare di $\sigma$.

Le curve \textbf{ROC} si utilizzano nel caso di dataset \emph{bilanciati} (frequenza simile delle due classi), mentre la \textbf{curva di Precision-Recall} è preferibile nel caso di dataset \emph{sbilanciati}.

Come indicatore dell’accuratezza del classificatore si usa l’area al di sotto delle curve ROC e Precision-Recall, detta \textbf{Area Under the Curve (AUC)}. \(\mathrm{AUC}\in[0,1]\); \(\mathrm{AUC}=1\) indica un classificatore perfetto.

\subsection{Validazione di un classificatore}
Per validare un classificatore si considerano solitamente diversi \emph{partizionamenti} del dataset in \emph{training} e \emph{test set} e si addestra il classificatore sul training set così ottenuto.
Generalmente si utilizzano due metodi:
\begin{description}
  \item[Holdout:] Si fissa una percentuale $X$, dopo si partiziona il dataset in due set indipendenti, training e test sets, in base a $X$, si addestra il modello sul training set e si applica il classificatore al test set per misurare l'accuratezza. Una variante prevede di ripetere l'holdout $k$ volte e di calcolare la media delle accuratezze ottenute.
  \item[K-Fold Cross Validation:] Si partiziona il dataset in $k$ sottoinsiemi (fold) di dimensioni approssimativamente uguali. Si eseguono $k$ iterazioni, in ciascuna si usa un fold come test set e gli altri $k-1$ fold come training set. Si calcola l'accuratezza per ogni iterazione e si fa la media delle $k$ accuratezze ottenute per avere una stima complessiva delle prestazioni del classificatore.
\end{description}
\chapter{Cenni di Regressione}
La regressione è una tecnica di apprendimento supervisionato utilizzata per prevedere valori continui basandosi su un insieme di dati di input. A differenza della classificazione, che assegna etichette discrete, la regressione mira a modellare la relazione tra una variabile dipendente (target) e una o più variabili indipendenti (predittori).

Da notare che la regressione può essere analizzata su diversi aspetti:
\begin{itemize}
    \item Un aspetto statistico: la regressione viene utilizzata per stimare le relazioni tra variabili e per fare inferenze sui dati.
    \item Un aspetto di machine learning: la regressione viene utilizzata per costruire modelli predittivi che possono generalizzare su nuovi dati.
\end{itemize}

Questo capitolo darà una panoramica delle principali tecniche di regressione utilizzate nel data mining e nel machine learning.

\section{Regressione lineare semplice}
La regressione lineare semplice è il caso più elementare di regressione, in cui si cerca di modellare la relazione tra due variabili: una variabile indipendente \( x \) e una variabile dipendente \( y \). 

\subsection{Formulazione del modello}
La relazione viene rappresentata da una retta di regressione, espressa dall'equazione:
\[
y = \beta_0 + \beta_1 x + \epsilon
\]
dove:
\begin{itemize}
    \item \( \beta_0 \) è l'intercetta (valore di \( y \) quando \( x = 0 \)). Questo parametro viene anche chiamato \textbf{bias} e rappresenta il punto in cui la retta interseca l'asse delle ordinate,
    \item \( \beta_1 \) è il coefficiente di regressione (pendenza della retta) che indica quanto varia \( y \) al variare di \( x \).
    \item \( \epsilon \) è l'errore casuale. Questo errore viene introdotto per tenere conto delle variazioni non spiegate dal modello lineare.
\end{itemize}

\subsection{Stima dei parametri}
Nel contesto del machine learning, l'obiettivo è stimare i parametri \( \beta_0 \) e \( \beta_1 \). In particolare si può utilizzare quella che viene chiamata la \textbf{metodologia dei minimi quadrati} (Ordinary Least Squares, OLS) che minimizza la somma dei quadrati degli errori tra i valori osservati e quelli predetti dal modello. La funzione di costo da minimizzare è:
\[
J(\beta_0, \beta_1) = \sum_{i=1}^{n} (y_i - (\beta_0 + \beta_1 x_i))^2
\]
dove \( n \) è il numero di osservazioni nel dataset.

\subsection{Interpretazione geoemetrica}
Geometricamente, la regressione lineare semplice cerca di trovare la retta che meglio si adatta ai punti dati in uno spazio bidimensionale. La pendenza \( \beta_1 \) indica l'inclinazione della retta, mentre l'intercetta \( \beta_0 \) indica il punto in cui la retta interseca l'asse delle ordinate.

\section{Regressione lineare multipla}
Un altro caso di regressione lineare è quella multipla, in cui si considerano più variabili indipendenti \( x_1, x_2, \ldots, x_p \) per prevedere la variabile dipendente \( y \). 

\subsection{Formulazione del modello}
La relazione viene rappresentata dall'equazione:
\[
y = \beta_0 + \beta_1 x_1 + \beta_2 x_2 + \ldots + \beta_p x_p + \epsilon
\]
o, in forma matriciale:
\[
\mathbf{y} = \mathbf{X} \boldsymbol{\beta} + \boldsymbol{\epsilon}
\]
dove:
\begin{itemize}
    \item \( \mathbf{y} \) è il vettore delle osservazioni della variabile dipendente,
    \item \( \mathbf{X} \) è la matrice delle variabili indipendenti (inclusa una colonna di 1 per l'intercetta),
    \item \( \boldsymbol{\beta} \) è il vettore dei coefficienti di regressione,
    \item \( \boldsymbol{\epsilon} \) è il vettore degli errori casuali.
\end{itemize}

\subsection{Stima dei parametri}
Anche in questo caso, si può utilizzare la metodologia dei minimi quadrati per stimare i parametri \( \boldsymbol{\beta} \). La soluzione analitica è data dalla formula:
\[
\boldsymbol{\hat{\beta}} = (\mathbf{X}^T \mathbf{X})^{-1} \mathbf{X}^T \mathbf{y}
\]
dove \( \boldsymbol{\hat{\beta}} \) rappresenta i coefficienti stimati.

\subsection{Interpretazione geometrica}
Geometricamente, la regressione lineare multipla cerca di trovare un iperpiano in uno spazio multidimensionale che meglio si adatta ai punti dati. Ogni coefficiente \( \beta_j \) rappresenta l'effetto marginale della variabile \( x_j \) sulla variabile dipendente \( y \), mantenendo costanti tutte le altre variabili indipendenti.

\section{Regressione non lineare}
In molti casi, la relazione tra le variabili indipendenti e la variabile dipendente non è lineare. In questi casi, si possono utilizzare modelli di regressione non lineare, che possono assumere varie forme, come polinomiali, esponenziali o logaritmiche. Questi modelli possono essere stimati utilizzando tecniche di ottimizzazione numerica, poiché spesso non esiste una soluzione analitica semplice come nel caso della regressione lineare.

Un esempio di regressione polinomiale è dato dall'equazione:
\[
y = \beta_0 + \beta_1 x + \beta_2 x^2 + \beta_3 x^3 + \epsilon
\]
dove i termini \( x^2 \) e \( x^3 \) permettono di modellare relazioni più complesse tra \( x \) e \( y \). Tuttavia questa formulazione si può convertire in una funzione lineare, considerando $x_2 = x^2$ e $x_3 = x^3$ come nuove variabili indipendenti:
\[
y = \beta_0 + \beta_1 x_1 + \beta_2 x_2 + \beta_3 x_3 + \epsilon
\]

\section{Regressione logistica}
La regressione logistica è una tecnica utilizzata per problemi di classificazione binaria, ma può essere vista come un'estensione della regressione lineare. In questo caso, l'obiettivo è prevedere la probabilità che un'osservazione appartenga a una delle due classi. La relazione viene modellata utilizzando la funzione logistica (sigmoide):
\[
P(y=1|x) = \frac{1}{1 + e^{-(\beta_0 + \beta_1 x_1 + \ldots + \beta_p x_p)}}
\]
dove \( P(y=1|x) \) rappresenta la probabilità che la variabile dipendente \( y \) sia uguale a 1 dato il vettore delle variabili indipendenti \( x \).

\subsection{Regressione logistica binaria semplice}
Nel caso più semplice di regressione logistica binaria con una sola variabile indipendente \( x \), la probabilità viene espressa come:
\[
P(y=1|x) = \frac{1}{1 + e^{-(\beta_0 + \beta_1 x)}}
\]
In questo caso, il modello stima la probabilità che l'evento \( y=1 \) si verifichi in funzione della variabile \( x \).

\paragraph{Stima dei parametri.}
Per stimare i parametri ottimali \( \beta_0 \) e \( \beta_1 \) si può andare a minimizzare la somma delle log-loss delle singole tuple. Data una tupla $T$ avente variabile predittiva $x$ e variabile target $y \in \{0,1\}$, la log-loss è definita come:
\[
\text{log-loss}(x) = 
\begin{cases}
-\log p(x) & \text{se } y = 1 \\
-\log (1 - p(x)) & \text{se } y = 0
\end{cases}
\]
dove \( p(x) = P(y=1|x) \) è la probabilità stimata dal modello. La funzione di costo complessiva da minimizzare è quindi:
\[
J(\beta_0, \beta_1) = -\sum_{i=1}^{n} \left[ y_i \log p(x_i) + (1 - y_i) \log (1 - p(x_i)) \right]
\]
\chapter{Subgraph Matching}
L'operazione di \textit{subgraph matching} consiste nel trovare tutte le occorrenze di un grafo più piccolo (detto \textit{query graph}) all'interno di un grafo più grande (detto \textit{data graph}). Questa operazione è fondamentale in molte applicazioni, come il rilevamento di pattern in reti sociali, l'analisi di reti biologiche e la ricerca di strutture specifiche in database grafici.

\section{Isomorfismo di Grafi}
L'\textbf{isomorfismo} tra due grafi \(G_1\) e \(G_2\) è una corrispondenza biunivoca tra i loro insiemi di nodi che preserva le relazioni di adiacenza. In altre parole, due grafi sono isomorfi se esiste una mappatura tra i nodi di \(G_1\) e \(G_2\) tale che due nodi sono connessi da un arco in \(G_1\) se e solo se i loro corrispondenti nodi sono connessi da un arco in \(G_2\).

\begin{figure}[htbp]
    \centering
    \includegraphics[width=0.8\textwidth]{images/subgraph_mapping_example.png}
    \caption{Esempio di subgraph mapping: il grafo di query (a sinistra) viene trovato all'interno del grafo di dati (a destra).}
    \label{fig:subgraph_mapping_example}
\end{figure}

\subsection{Automorfismo}
Un caso particolare di \emph{isomorfismo} è l'\textbf{automorfismo}: si verifica quando un grafo è isomorfo a se stesso. In altre parole, un automorfismo è una mappatura dei nodi di un grafo su se stessi che preserva le relazioni di adiacenza. Gli automorfismi sono importanti nello studio delle simmetrie nei grafi e possono essere utilizzati per semplificare la rappresentazione di un grafo.

\begin{figure}[htbp]
    \centering
    \includegraphics[width=0.8\textwidth]{images/automorphism_example.png}
    \caption{Esempi di \emph{automorfismi} del grafo \(G\): a sinistra il grafo originale; al centro un automorfismo che scambia i vertici \(2\) e \(3\) mantenendo \(1\) fisso; a destra un secondo automorfismo che permuta i vertici come indicato dalle frecce tratteggiate. In entrambi i casi la struttura di adiacenza è preservata.}
    \label{fig:automorphism_example}
\end{figure}

\section{Operazione di subgraph matching}
L'operazione di \textit{subgraph matching} consiste, definita in modo più formale, nel verificare se il \emph{grafo query} è contenuto nel \emph{grafo dati} attraverso una mappatura che preserva le relazioni di adiacenza.

La differenza rispetto al concetto di isomorfismo risiede nel fatto che, nel subgraph matching, il grafo query può essere più piccolo del grafo dati e non è necessario che tutti i nodi del grafo dati siano coinvolti nella mappatura. Quindi , il subgraph matching cerca una corrispondenza parziale, \textbf{iniettiva}, tra i due grafi, mentre l'isomorfismo richiede una corrispondenza completa, \textbf{biunivoca}, tra tutti i nodi e gli archi dei due grafi.

Definiamo $G_1 = (V_1, E_1)$ come il grafo query e $G_2 = (V_2, E_2)$ come il grafo dati. L'operazione di subgraph matching cerca una funzione iniettiva \(f: V_1 \rightarrow V_2\) chiamata \textbf{mapping} tale che per ogni arco \((u, v) \in E_1\), l'arco \((f(u), f(v))\) appartiene a \(E_2\).

\begin{figure}[htbp]
    \centering
    \includegraphics[width=0.8\textwidth]{images/subgraph_matching_example.png}
    \caption{Esempio di \emph{subgraph matching}: il grafo \emph{Query} (in alto a sinistra) viene ricercato all'interno del grafo \emph{Target} (in alto a destra). In basso sono mostrati i possibili match del sottografo query all'interno del grafo target, evidenziati dai riquadri rossi.}
    \label{fig:subgraph_matching_example}
\end{figure}

Come si vede dalla figura \ref{fig:subgraph_matching_example}, il processo di subgraph matching ritorna più occorrenze del grafo query all'interno del grafo dati, evidenziando le diverse mappature possibili che soddisfano le condizioni di adiacenza.

\section{Complessità computazionale}
L'operazione di graph matching è nota per essere computazionalmente complessa. Infatti, il problema del graph matching è NP-Hard, ovvero non esiste un algoritmo noto che possa risolverlo in tempo polinomiale per tutti i casi. Questo rende il graph matching un algoritmo particolarmente esoso.

Per quanto riguarda il subgraph matching, la complessità dipende dalla dimensione del grafo query e del grafo dati, nonché dalla struttura dei grafi stessi. In generale è un problema NP-Completo, il che significa che non esiste un algoritmo noto che possa risolverlo in tempo polinomiale per tutti i casi, tuttavia esistono algoritmi euristici e approssimativi che possono essere utilizzati per affrontare il problema in modo più efficiente in casi specifici o con vincoli particolari.

\section{Algoritmi di subgraph matching}
\subsection{Soluzione Bruteforce}
La soluzione più semplice per il problema del subgraph matching è l'approccio \textbf{bruteforce}, che consiste nel generare tutte le possibili mappature dei nodi del grafo query sui nodi del grafo dati e verificare se ciascuna mappatura preserva le relazioni di adiacenza.

\begin{figure}[htbp]
    \centering
    \includegraphics[width=0.78\textwidth]{images/bruteforce_subgraph_matching.png}
    \caption{Ricerca \emph{brute force} per \emph{subgraph matching}: in alto i grafi \(G_a\) (query) e \(G_b\) (target); sotto l'albero di ricerca che esplora tutte le corrispondenze possibili tra vertici (es.\ \((1,4)\), \((1,5)\), \((1,6)\), \((1,7)\)). Le foglie evidenziate indicano i mapping completi che preservano le adiacenze (match isomorfi).}
    \label{fig:bruteforce_subgraph_matching}
\end{figure}

Questo è chiaramente una soluzione NP-Hard, inefficiente. Si possono adottare strategie migliori per velocizzare la computazione:
\begin{itemize}
    \item \textbf{Look-ahead}: prima di verificare una mappatura completa, si può controllare se i nodi parzialmente mappati soddisfano le condizioni di adiacenza. Se non lo fanno, si può scartare immediatamente quella mappatura. 
    \item \textbf{Backtracking}: si può utilizzare una strategia di backtracking per esplorare lo spazio delle mappature in modo più efficiente. Se si scopre che una mappatura parziale non può essere estesa a una mappatura completa valida, si torna indietro e si prova una diversa mappatura.
\end{itemize}

\paragraph{Esempio di backtracking.}
Partendo dalla radice dell'albero di ricerca (figura \ref{fig:backtracking_subgraph_matching}), assegnamo progressivamente i vertici del grafo \emph{query} ai vertici del \emph{target}. Dopo le prime scelte \((1\!\to\!a)\) e \((2\!\to\!b)\), proviamo \((3\!\to\!c)\). A questo punto un controllo locale sui vincoli struttura-preservanti rivela un'incoerenza: il grado di \(3\) nel query è 3, mentre quello di \(c\) nel target è 2; quindi nessun proseguimento potrà produrre un mapping valido. L'algoritmo esegue allora \emph{pruning} del ramo e \emph{backtracking} allo stato precedente per provare alternative (ad es.\ \((3\!\to\!d)\)). Questo comportamento evita esplorazioni inutili e riduce drasticamente lo spazio di ricerca.

\begin{figure}[htbp]
    \centering
    \includegraphics[width=0.8\textwidth]{images/backtracking_subgraph_matching.png}
    \caption{Esempio di \emph{backtracking} con pruning nel \emph{subgraph matching}. A sinistra il grafo \emph{query} e il grafo \emph{target}; a destra l'albero di ricerca. Il ramo che mappa \((3\!\to\!c)\) viene potato perché viola un vincolo di grado (\(\deg(3)=3>\deg(c)=2\)), quindi non può condurre a una soluzione.}
    \label{fig:backtracking_subgraph_matching}
\end{figure}

\subsection{Algoritmo di Ullmann}
L'algoritmo di Ullman utilizza il grado del nodo come criterio di selezione per ridurre lo spazio di ricerca. In particolare, prima di tentare una mappatura, l'algoritmo verifica se il grado del nodo nel grafo query è minore o uguale al grado del nodo corrispondente nel grafo dati. Se questa condizione non è soddisfatta, la mappatura viene scartata immediatamente (\emph{look-ahead}).

Dopo aver costruito una mappatura iniziale, l'algoritmo applica una procedura di \textbf{refinement} per eliminare ulteriori mappature non valide. Questa procedura verifica che per ogni arco nel grafo query, l'arco corrispondente esista nel grafo dati. Se un arco manca, la mappatura viene scartata (\emph{backtracking}).

\begin{figure}[htbp]
    \centering
    \includegraphics[width=0.9\textwidth]{images/ullmann_algorithm_example.png}
    \caption{Algoritmo di Ullmann: a sinistra il grafo \emph{query} \(Q\) e il grafo \emph{target} \(T\); a destra l'albero di ricerca con le corrispondenze provate. I nodi cerchiati in rosso indicano scelte scartate tramite \emph{pruning}; il ramo con \((3,7)\) completa un mapping valido.}
    \label{fig:ullmann_algorithm_example}
\end{figure}

\paragraph{Esempio della figura \ref{fig:ullmann_algorithm_example}.}
Per prima cosa si calcola l'ordine di ogni vertice, nel grafo $Q$ abbiamo \( deg(1) = deg(2) = deg(5) \), nel grafo $T$ abbiamo \( deg(6) = deg(7) = \deg(5) = 2, deg(4) = 1 \).

L'obiettivo è quello di trovare un \emph{isomorfismo} tra i due grafi, iniziando dalla radice dell'albero di ricerca. Si prova l'assegnazione \((1, 4)\) ma viene scartata perché il grado di \(1\) è maggiore del grado di \(4\). Si prova quindi \((1, 5)\) e si procede con \((2, 4\) (non funziona per lo stesso motivo \( (1, 4)\)) , quindi si prova \((2, 6)\) e così via. Alla fine si trova il mapping valido \((1, 5), (2, 6), (5, 7)\).

\subsection{Algoritmo VF}
L'algoritmo VF (Vento-Foggia) è un altro approccio per il problema del subgraph matching che utilizza una strategia di backtracking con pruning basata su vincoli locali. L'algoritmo costruisce una mappatura incrementale dei nodi del grafo query sui nodi del grafo dati, verificando a ogni passo se la mappatura parziale soddisfa i vincoli di adiacenza.

\noindent
L'algoritmo costruisce 3 insiemi per ognuno dei grafi $Q, T$:
\begin{itemize}
    \item $M_Q, M_T$: nodi già mappati
    \item $T_Q, T_T$: nodi adiacenti a quelli mappati
    \item $U_Q, U_T$: nodi non ancora considerati
\end{itemize}

\noindent
Dove il pedice $Q$ o $T$ indica se l'insieme appartiene al grafo query o target.

\textbf{N.B.} È importante notare che l'algoritmo VF è ricorsivo ma lavora a \textbf{stati}: ad ogni passo ricorsivo considera lo stato attuale in cui, definito dagli insiemi di nodi mappati, adiacenti e non ancora considerati.


\paragraph{Algoritmo.}
Ad ogni passo dell'algoritmo, per lo stato corrente $s$, si eseguono i seguenti passi:
\begin{enumerate}
    \item Si seleziona un nodo $q \in T_Q(s)$ e si prova a mappare $\forall t \in T_T(s)$.
    \item Si verifica la regola di \textbf{fattibilità} per un certa coppia $(q, t) \in T_Q(s) \times T_T(s)$. Se la coppia è fattibile, si crea un nuovo stato $s'$ aggiornando gli insiemi $M_Q, M_T, T_Q, T_T, U_Q, U_T$ di entrambi i grafi. Da notare che si preferisce sempre scegliere una coppia di nodi che il grado più alto possibile, in modo da massimizzare le possibilità di pruning e ridurre la computazione nel caso medio.
    \item Si ripete il processo fino a quando tutti i nodi del grafo query sono stati mappati (trovando così un match) o fino a quando non ci sono più nodi da mappare (in tal caso si esegue il backtracking).
\end{enumerate}

\paragraph{Regola di fattibilità per grafi indiretti.}
Una coppia di nodi $(q, t) \in T_Q(s) \times T_T(s)$ è considerata fattibile se soddisfa le seguenti condizioni:
\begin{itemize}
    \item Per ogni nodo query $q' \in M_Q(s)$ connesso a $q$, esiste un nodo target $t' \in M_T(s)$ connesso a $t$.
    \item Il numero di nodi query in $T_Q(s)$ deve essere minore o uguale\footnote{Minore o uguale nel caso di subgraph matching, nel caso di graph matching invece deve essere uguale.} al numero di nodi target in $T_T(s)$ (regola look-ahead a un livello).
    \item Il numero di nodi query in $U_Q(s)$ deve essere minore o uguale\footnote{Anche qui,minore uguale solo nel caso di subgraph matching. Nel caso di Graph deve essere uguale.} al numero di nodi target in $U_T(s)$ (regola look-ahead a due livelli).
\end{itemize}

\begin{figure}[htbp]
    \centering
    \includegraphics[width=\textwidth]{images/vf_algorithm_example.png}
    \caption{Rappresentazione grafica dei sei insiemi utilizzati dall'algoritmo VF per verificare la regola di fattibilità durante la costruzione incrementale della mappatura tra nodi del grafo query $Q$ e nodi del grafo target $T$.  
    La figura mostra, per uno stato $s$ dell'algoritmo, la suddivisione di entrambi i grafi nei tre insiemi fondamentali: i nodi già mappati ($\text{Match}_Q(s)$ e $\text{Match}_T(s)$), i nodi adiacenti a quelli mappati ($\text{Term}_Q(s)$ e $\text{Term}_T(s)$) e i nodi non ancora considerati ($\text{Rem}_Q(s)$ e $\text{Rem}_T(s)$).  
    Le frecce tratteggiate indicano le possibili connessioni tra nodi dei diversi insiemi, mentre le frecce orizzontali centrali rappresentano la mappatura parziale $M(s)$ costruita dallo stato corrente.  
    Questa struttura consente di verificare la regola di fattibilità: ogni nuova coppia $(q,t)$ può essere aggiunta alla mappatura se preserva le adiacenze verso i nodi già mappati, se esistono sufficienti nodi adiacenti disponibili (look-ahead a un livello) e se esistono sufficienti nodi non ancora considerati per supportare eventuali mappature future (look-ahead a due livelli).}
    \label{fig:vf_algorithm_example}
\end{figure}

\paragraph{Regola di fattibilità per grafi diretti.}
Nel caso di grafi diretti, la regola di fattibilità deve considerare anche la direzione degli archi. Quindi, ogni grafo non ha più 3 insiemi di nodi ma 6:
\begin{itemize}
    \item $M_Q^{in}, M_T^{in}$: nodi già mappati considerando gli archi entranti
    \item $M_Q^{out}, M_T^{out}$: nodi già mappati considerando gli archi uscenti
    \item $T_Q^{in}, T_T^{in}$: nodi adiacenti a quelli mappati considerando gli archi entranti
    \item $T_Q^{out}, T_T^{out}$: nodi adiacenti a quelli mappati considerando gli archi uscenti
    \item $U_Q, U_T$: nodi non ancora considerati
\end{itemize}

\noindent
La regola di fattibilità in questo caso, deve essere modificata per tenere conto delle direzioni degli archi:
\begin{itemize}
    \item Per ogni nodo query $q' \in M_Q^{in}(s)$, predecessore di $q$, esiste un nodo target $t' \in M_T^{in}(s)$ predecessore di $t$.
    \item Per ogni nodo query $q' \in M_Q^{out}(s)$, successore di $q$, esiste un nodo target $t' \in M_T^{out}(s)$ successore di $t$.
    \item Il numero di nodi query in $T_Q^{in}(s)$ deve essere minore o uguale al numero di nodi target in $T_T^{in}(s)$.
    \item Il numero di nodi query in $T_Q^{out}(s)$ deve essere minore o uguale al numero di nodi target in $T_T^{out}(s)$.
    \item Il numero di nodi query in $U_Q(s)$ deve essere minore o uguale al numero di nodi target in $U_T(s)$.
\end{itemize}

\paragraph{Complessità e considerazioni}
La complessità, nell'ipotesi che i due grafi abbiano $N$ nodi, si può calcolare come:
\begin{enumerate}
    \item Sapendo che ogni nodo ha in media $O(N)$ nodi adiacenti, quindi per ogni nodo query si devono considerare $O(N)$ possibili mappature nel grafo target.
    \item Per ogni mappatura, si devono verificare le condizioni di fattibilità, che richiedono di controllare gli archi adiacenti. Questo richiede $O(N)$ operazioni.
\end{enumerate}

Quindi nel caso migliore, la complessità dell'algoritmo dipende dalla visita di un solo nodo candidato per ogni nodo query, portando a una complessità di \(O(N^2)\). Tuttavia, nel caso peggiore, l'algoritmo potrebbe dover esplorare tutte le possibili mappature, portando a una complessità esponenziale di \(O(N!)\).

La complessità spaziale, al più $k$ stati (dove $k$ è il numero di nodi del grafo query) porta a una complessità spaziale di \(O(k \cdot N)\), considerando che per ogni stato si devono memorizzare gli insiemi di nodi mappati, adiacenti e non ancora considerati per entrambi i grafi.

\subsection{Algoritmo VF2}
L'algoritmo VF2 è un miglioramento dell'algoritmo VF, progettato per essere più efficiente nel risolvere il problema del subgraph matching. VF2 ottimizza lo spazio utilizzato, fino ad ottenere una complessità spaziale pari a $O(N)$, utilizzando strutture dati globali e condivise tra i vari stati. Si introducono sei strutture dati:
\begin{itemize}
    \item $\text{core}_1$ e $\text{core}_2$: array che memorizzano la mappatura corrente dei nodi del grafo query e del grafo target, rispettivamente.
    \item $\text{in}_1$ e $\text{in}_2$: array booleani che indicano se un nodo è adiacente a un nodo già mappato (insiemi $\text{Term}_Q$ e $\text{Term}_T$).
    \item $\text{out}_1$ e $\text{out}_2$: array booleani che indicano se un nodo non è ancora considerato (insiemi $\text{Rem}_Q$ e $\text{Rem}_T$).
\end{itemize}

Grazie a queste strutture, si può \emph{tracciare} contemporaneamente l'appartenenza dei nodi agli insiemi necessari per la verifica della regola di fattibilità, senza dover mantenere insiemi separati per ogni stato dell'algoritmo. Questo riduce significativamente l'overhead di memoria, permettendo all'algoritmo di gestire grafi più grandi in modo più efficiente.

\subsection{Algoritmo RI}
L'algoritmo RI è un approccio alternativo per il problema del subgraph matching. Gli algoritmi VF si basano su una strategia di backtracking con pruning, mentre l'algoritmo RI si concentra invece \textbf{sull'ordine} in cui i nodi della query vengono processati nell'albero di ricerca. Un ordinamento efficace, infatti, \emph{velocizza} molto il matching anche in presenza di regole di pruning meno restrittive rispetto a VF.

In particolare, l'ordinamento in RI è calcolato indipendentemente dal grafo target (static ordering) e si può riassumere nei seguenti passi:
\begin{enumerate}
    \item Si ordinano i nodi del grafo query in modo da \textbf{massimizzare} la probabilità che un cammino parziale nell'albero di ricerca venga tagliato il prima possibile.
    \item Seguendo l'ordinamento calcolato al passo 1, si mappano i nodi della query a nodi del target verificando per ogni coppia candidata $(q, t)$ che:
    \begin{enumerate}
        \item $q$ e $t$ non siano già stati mappati ad altri nodi.
        \item Il grado di $q$ sia minore o uguale al grado di $t$.
    \end{enumerate}
    \item Si ripete il passo 2 finché l'intero spazio di ricerca non viene esplorato.
\end{enumerate}

\paragraph{Ordinamento dei nodi.}
Il punto principale di questo algoritmo è capire come ordinare i nodi della query per massimizzare l'efficacia del pruning. L'obiettivo dell'ordinamento è quello di costruire una sequenza ordinata di nodi \( [q_1, q_2, \ldots, q_n] \) tale che, durante la ricerca, i nodi connessi ai nodi già mappati vengano processati il prima possibile. Questo aumenta la probabilità che le condizioni di fattibilità vengano violate precocemente, permettendo di tagliare rami dell'albero di ricerca in modo più efficiente. Per farlo, si può definire una \textbf{regola generale}: al passo $i$, si sceglie il nodo $q_i$ con grado più alto e con un elevato numero di connessioni con i nodi già presenti nell'ordinamento $\bigcup^{i-1}$.

Sia $\bigcup^{i-1} = \{q_1, q_2, \ldots, q_{i-1}\}$ la sequenza parziale già ordinata al passo $i-1$. Allora, il punteggio di un nodo candidato $v$ da inserire nell'ordinamento è definito sulla base di 3 insiemi:
\begin{enumerate}
    \item $V_{adj}(i)$: insieme dei nodi adiacenti a $v$ che sono già presenti in $\bigcup^{i-1}$.
    \item $V_{conn}(i)$: insieme dei nodi in $\bigcup^{i-1}$ adiacenti ad almeno un nodo che non appartiene a $\bigcup^{i-1}$ ed è connesso a $v$.
    \item $V_{rem}(i)$: insieme dei nodi non ancora considerati (cioè non in $\bigcup^{i-1}$) e non connessi a nessun nodo in $\bigcup^{i-1}$, ma connessi a $v$.
\end{enumerate}

Da questo si crea un punteggio per il successivo nodo da inserire:
\begin{enumerate}
    \item Massimo valore di $|V_{adj}(i)|$.
    \item In caso di parità, massimo valore di $|V_{conn}(i)|$.
    \item In caso di ulteriore parità, massimo valore di $|V_{rem}(i)|$.
\end{enumerate}

\noindent
Se ancora dovesse esserci parità, si sceglie il nodo con il grado più alto.


\subsubsection*{Esempio.}
Consideriamo il grafo di query \(Q\) nella figura \ref{fig:ri_algorithm_example} 

\begin{figure}[htbp]
    \centering
    \includegraphics[width=0.4\textwidth]{images/ri_algorithm_example.png}
    \caption{Grafo non diretto di esempio (11 nodi, 0-10), usato per illustrare l'ordinamento dei nodi della query nell'algoritmo RI.}
    \label{fig:ri_algorithm_example}
\end{figure}

\paragraph{Passo $i=1$.}
Per prima cosa, dobbiamo ordinare i nodi. Lo stato iniziale dell'ordinamento è $U^0 = \emptyset$, per ogni nodo $v$ calcoliamo i 3 insiemi:
\[
|V_{\text{adiac}}(1,v)| = 0, \quad |V_{\text{conn}}(1,v)| = 0, \quad |V_{\text{rem}}(1,v)| = \deg(v)
\]
Quindi vince il nodo con grado massimo, ovvero il nodo 4. Ora lo stato dell'ordinamento è $U^1 = (4)$.

\paragraph{Passo $i=2$.}
Possiamo calcolare una tabella per tutti i candidati al passo 2, in quanto ora $U^1 = (4)$.

\begin{table}[htbp]
\centering
\footnotesize
\begin{tabular}{c|ccc}
\textbf{Nodo $v$} & $V_{\text{adiac}}(2,v)$ & $V_{\text{conn}}(2,v)$ & $V_{\text{rem}}(2,v)$ \\
\hline
0  & \{4\}       & \{4\}       & \{9\}       \\
1  & \{4\}       & \{4\}       & \{2\}       \\
2  & $\varnothing$ & \{4\}     & $\varnothing$ \\
3  & \{4\}       & \{4\}       & $\varnothing$ \\
5  & \{4\}       & \{4\}       & \{2,8\}     \\
6  & \{4\}       & \{4\}       & \{9\}       \\
7  & \{4\}       & \{4\}       & \{8\}       \\
8  & $\varnothing$ & \{4\}     & $\varnothing$ \\
9  & $\varnothing$ & \{4\}     & \{10\}      \\
10 & $\varnothing$ & $\varnothing$ & \{9\}   \\
\end{tabular}
\end{table}

In questo caso, si nota che i nodi candidati hanno $|V_{\text{adiac}}| = 1$. e $|V_{\text{conn}}| = 1$. Quindi si passa a guardare $|V_{\text{rem}}|$, che è massimo per il nodo 5 (2 nodi). Quindi il secondo nodo dell'ordinamento è $u_2$ = 5, e lo stato diventa $U^2 = (4, 5)$.

\paragraph{Tabella finale}
Dopo aver ripetuto questo processo per tutti i nodi, otteniamo l'ordinamento finale:
\[
U = (4, 5, 1, 3, 7, 0, 6, 9, 2, 8, 10)
\]


\subsection{RI-DS}
Una versione ottimizzata dell'algoritmo RI, chiamata RI-DS (RI with Domain Size ordering), introduce un'ulteriore strategia di ordinamento basata sulle dimensioni del \textbf{dominio} di ciascun nodo della query. In particolare si cerca di applicare la regola del grado di Ullman in fase di matching una sola volta per una stessa coppia $(q, t)$ e ridurre a monte le possibili coppie candidate. Questo perché se un nodo $q$ della query ha un grado maggiore di un nodo $t$ del target, allora non può esistere una mappatura valida che includa la coppia $(q, t)$.

Si può definire, per un certo nodo $q$ della query, il suo \textbf{dominio} $D(q)$ come l'insieme dei nodi del grafo target che possono essere mappati a $q$ rispettando la condizione del grado:
\[
D(q) = \{t \in V_T: deg(q) \leq deg(t)\}
\]
dove $deg(x)$ indica il grado del nodo $x$. Da questo, l'algoritmo esegue i seguenti passi:
\begin{enumerate}
    \item Ordina i nodi del grafo query.
    \item Calcola i domini per ogni nodo query.
    \item SEguendo l'ordinamento calcolato al passo 1, si mappano i nodi della query a nodi del target verificando per ogni coppia candidata $(q, t)$ che:
    \begin{enumerate}
        \item $q$ e $t$ non siano già stati mappati ad altri nodi.
        \item $t \in D(q)$, ovvero il nodo target $t$ appartiene al dominio del nodo query $q$.
    \end{enumerate}
    \item Si ripete il passo 3 finché l'intero spazio di ricerca non viene esplorato.
\end{enumerate}

\section{Subgraph Matching in Database di Grafi}
Il problema del subgraph matching è di fondamentale importanza nei database di grafi, dove si desidera trovare sottografi specifici all'interno di uno o più grafi target. Il problema diventa particolarmente rilevante in applicazioni come la bioinformatica, l'analisi delle reti sociali e la ricerca di pattern in grandi dataset.

La vera sfida in questi scenari è l'efficienza: i database di grafi possono contenere milioni o addirittura miliardi di nodi e archi, rendendo il subgraph matching un compito computazionalmente intensivo. 

\subsection{Indicizzazione}
Una soluzione \emph{naif} consisterebbe nell'eseguire il subgraph matching su ogni grafo del database; tuttavia, ciò sarebbe proibitivo in termini di tempo.

Per ottenere tempi ragionevoli, si indicizzano i grafi del database così da \textbf{filtrare} rapidamente i candidati rilevanti per una data query. L'idea è il classico paradigma \emph{filter-and-verify}: prima si usa l'indice per selezionare soltanto i grafi che condividono con la query determinate \emph{caratteristiche discriminanti} (ad es.\ frequenti sottografi, cammini/percorsi $q$-gram, firme di vicinato, conteggi di etichette e gradi), poi si applica il subgraph matching \emph{solo} su questo sottoinsieme ristretto. In questo modo si riduce drasticamente il numero di confronti, concentrando l'attenzione sui grafi con una reale possibilità di contenere il sottografo cercato, e migliorando sensibilmente i tempi di risposta senza sacrificare la correttezza del risultato.

\noindent
Esistono due tipi di indicizzazione:
\begin{description}
    \item[Indicizzazione basata su feature] - il grafo viene rappresentato tramite un insieme di \emph{feature} (sottografi frequenti, cammini, alberi, ecc.) che catturano le sue proprietà strutturali. Queste feature vengono poi memorizzate in un indice che consente di recuperare rapidamente i grafi che contengono determinate feature.
    \item[Indicizzazione non basata su feature] - i grafi vengoono mappati in uno spazio metrico e memorizzato in un \textbf{albero} (come un B-tree) dove fare ricerca basata su distanze. In questo modo, si possono recuperare rapidamente i grafi che sono "vicini" alla query in termini di struttura.  
\end{description}

\section{Features dei grafi}
Le \textbf{feature} sono delle proprietà strutturali dei grafi che possono essere utilizzate per l'indicizzazione e il filtraggio nei database di grafi. Si possono estrarre dal grafo diverse feature:
\begin{itemize}
    \item \textbf{Piccoli sottografi}: si possono identificare e memorizzare piccoli sottografi frequenti all'interno del grafo. Questi sottografi possono essere utilizzati come feature per rappresentare la struttura del grafo.
    \item \textbf{Cammini e percorsi}: si possono estrarre cammini di lunghezza fissa (chiamati $q$-gram) o percorsi specifici all'interno del grafo. Questi cammini possono essere utilizzati per confrontare la struttura dei grafi.
    \item \textbf{Alberi}: si possono identificare alberi radicati o alberi di profondità limitata all'interno del grafo. Questi alberi possono essere utilizzati come feature per rappresentare la struttura gerarchica del grafo.
\end{itemize}

\paragraph{Esempio di filtraggio tramite profili di feature.}
\begin{figure}[htbp]
  \centering
  \includegraphics[width=\textwidth]{images/feature_indexing_example.png}
  \caption{Indicizzazione \emph{feature-based} per il filtraggio dei candidati.
  A sinistra: dalla query \(Q\) si estraggono piccole caratteristiche (feature) e
  si costruisce il profilo di frequenza \(F_Q\).
  A destra: per ogni grafo del database (es.\ \(G\)) è precomputato il profilo \(F_G\).}
  \label{fig:feature_indexing_example}
\end{figure}

L'idea è rappresentare sia la query \(Q\) sia ciascun grafo \(G\) con un vettore/multinsieme di \emph{feature} e delle loro occorrenze (cammini, piccoli pattern etichettati, ecc.).  
Nel riquadro di sinistra, \(F_Q\) riporta per ogni feature il numero richiesto dalla query: una di esse richiede \(3\) occorrenze, altre due compaiono almeno una volta.  
Nel riquadro di destra, \(F_G\) mostra le occorrenze del medesimo insieme di feature nel grafo candidato \(G\).

Regola di filtro: se esiste una feature \(f\) tale che \(F_Q(f) > F_G(f)\), allora \(G\) \emph{non} può contenere \(Q\) e viene scartato.  
Nell'immagine, per una feature \(Q\) richiede \(3\) occorrenze mentre \(G\) ne ha \(1\) (\(\rightarrow\) \emph{2 occorrenze mancanti}); inoltre due feature presenti in \(Q\) non compaiono in \(G\) (\emph{2 feature mancanti}). Il grafo \(G\) viene dunque escluso già in fase di filtro, riducendo il numero di candidati da passare al subgraph matching.

\subsection{Schema di subgraph matching in database di grafi}
Nel caso di database di grafi, il subgraph matching segue lo schema \emph{filter-and-verify} (filtra e verifica):
\begin{enumerate}
    \item \textbf{Preprocessing}: per ogni grafo del database, si estraggono tutte le features rilevanti che contiene.
    \item \textbf{Filtering}: dalla query si estraggono tutte le feature contenute e si confrontano con quelle dei grafi del database, scartando quelli che non le contengono tutte.
    \item \textbf{Matching}: per ogni grafo rimanente, si esegue l'algoritmo di subgraph matching (ad es.\ VF2, RI-DS) per verificare se la query è effettivamente un sottografo del grafo target.
\end{enumerate}

Grazie a questo schema, si riduce drasticamente il numero di grafi su cui eseguire il subgraph matching, migliorando l'efficienza complessiva del processo.

\subsection{Indicizzazione inversa}
Un'altra tecnica di indicizzazione nei database di grafi è l'uso di un indice inverso\footnote{per indice inverso si intende una struttura dati che mappa feature a grafi contenenti tali feature} basato sulle feature. Ad ogni feature (intesa come chiave) si associa la lista dei grafi che la contengono, con il relativo numero di occorrenze della feature.

L'insieme dei candidati per il matching si può ottenere per intersezione delle liste dei grafi associate a ciascuna feature della query. In questo modo, si ottiene rapidamente l'insieme ristretto di grafi che contengono tutte le feature richieste dalla query, riducendo il numero di confronti necessari per il subgraph matching.

\subsection{Algoritmo SING}
\begin{figure}[htbp]
    \centering
    \includegraphics[width=0.9\textwidth]{images/sing_algorithm_example.png}
    \caption{Query \(Q\) (angolo con nodo \(A\) adiacente a \(B\) e \(C\)) e due grafi candidati. 
    Nel grafo \(g_1\) il pattern è presente come \emph{subgrafo non indotto}, ma non come \emph{indotto} a causa dell'arco diagonale aggiuntivo; 
    nel grafo \(g_2\) il pattern è presente anche come \emph{subgrafo indotto}.}
    \label{fig:sing_algorithm_example}
\end{figure}

\noindent
L'algoritmo SING (Subgraph search In Non-homogeneous Graphs) è un algoritmo di subgraph matching in un database di grafi basato su indexing tramite cammini di nodi. Per migliorare l'efficacia dell'indicizzazione si associa ad ogni feature la frequenza di occorrenza in ogni grafo del database ed il nodo da cui parte. 

L'algoritmo utilizza due tipi di \textbf{indicizzazione} (come si vede in figura \ref{fig:sing_indexing_example}):
\begin{description}
    \item[Indice inverso globale] - ad ogni feature si associa la lista dei grafi del database che la contengo e il relativo conteggio di occorrenze. Questo aiuta a filtrare rapidamente i grafi che non contengono tutte le feature della query.
    \item[Indice inverso locale per ciascun grafo $\mathbf{g}$] - per ogni feature presente in $g$ è associato un vettore binario dove l'$i$-esimo bit indica se la feature parte dal nodo $i$ di $g$. Questo aiuta a velocizzare il processo di matching all'interno di ciascun grafo candidato.
\end{description}

\begin{figure}[htbp]
    \centering
    \includegraphics[width=\textwidth]{images/sing_indexing_example.png}
    \caption{Indicizzazione inversa globale e locale per l'algoritmo SING.
    A sinistra: indice inverso globale che associa ad ogni feature a lista dei grafi che la contengono e il conteggio delle occorrenze.
    A destra: indice inverso locale per il grafo \(g_1\), dove per ogni feature è indicato un vettore binario che mostra i nodi di partenza della feature in \(g_1\).}
    \label{fig:sing_indexing_example}
\end{figure}

\paragraph{Processamento della query.}
L'algoritmo procede a eseguire una query di subgraph matching seguendo questi passi:
\begin{enumerate}
    \item Si esegue il \textbf{primo filtraggio}, ovvero per ogni feature $F$ della query si recuperano i grafi che la contengono, per un maggiore numero di volte del grafy query, dall'indice inverso globale. Si ottiene così un insieme di grafi candidati che contengono tutte le feature della query e poi insieme si calcola l'intersezione $R$.
    \item Dopo si esegue un \textbf{secondo filtraggio}, per ogni grafo $G \in R$ si usa l'indica locale per calolcare gli insiemi di nodi compatibili con i nodi della query. Da quello, si scartano i grafi che non hanno almeno un nodo compatibile per ogni nodo della query. Questo, dato un vertice $v$ della query, calcola l'insieme dei vertici di un grafo $G$ del database che sono compatibili con $v$. Si dice compatibile un nodo $u$ di $G$ se esiste almeno una feature che parte da $v$ nella query e da $u$ in $G$. Dato $S$ insieme di queste features che partono da $v$, è sufficiente calcolare nell'indice locale del grafo $G$ un AND logico tra i vettori binari associati alle feature di $S$ e se restituisce un vettore non nullo, allora esiste almeno un nodo compatibile con $v$ in $G$.
    \item Infine, si esegue il \textbf{matching} vero e proprio sui grafi rimanenti, usando un algoritmo di subgraph matching (ad es.\ VF2, RI-DS) per verificare se la query è effettivamente un sottografo del grafo target.
\end{enumerate}

\section*{Riferimenti}
I riferimenti per questo capitolo sono:
\begin{itemize}
    \item Articolo originale dell'algoritmo Vento-Foggia\cite{Cordella1999VF}
    \item Articolo originale dell'algoritmo VF2\cite{Cordella2004VF2}
    \item Articolo dell'algoritmo RI e RI-DS \cite{Bonnici2013RI}
\end{itemize}


\chapter{Subgraph Matching di Grafi Frequenti}
Come nel caso del capitolo \ref{ch:frequent-itemsets}, l'estrazione di sotto-grafi frequenti consiste nell'\textbf{identificare} tutti i sotto-grafi che appaiono frequentemente in un database di grafi.

Anche in questo caso si conta il numero di grafi del dataset che contengono il sottografo, chiamato \textbf{supporto} e se il supporto del sotto-grafo è maggiore o uguale ad una soglia minima di supporto, allora il sotto-grafo è considerato \textbf{frequente}. Per fare una relazione con il mining di insiemi frequenti, possiamo considerare una mappatura del tipo:
\begin{itemize}
    \item Transazioni $\rightarrow$ Grafi
    \item Itemsets $\rightarrow$ Sotto-grafi
    \item Item $\rightarrow$ Nodi
    \item Relazioni tra gli item $\rightarrow$ Archi tra i nodi
\end{itemize}

\begin{figure}[htbp]
    \centering
    \includegraphics[width=\textwidth]{images/frequent-subgraph-matching.png}
    \caption{Esempio di estrazione di sotto-grafi frequenti da un database di grafi.}
    \label{fig:frequent-subgraph-matching}
\end{figure}

Spesso non si calcola il supporto di un sotto-grafo, ma bensì alla seua frequenza: il conteggio relativo delle occorrenze del sotto-grafo all'interno di un singolo grafo. In questo caso, il sotto-grafo è considerato frequente se la somma delle frequenze nei grafi del database è maggiore o uguale alla soglia minima di supporto, che indichiamo con $\sigma$ ed è una percentuale.

\begin{figure}[htbp]
    \centering
    \includegraphics[width=\textwidth]{images/frequent-subgraph-frequency.png}
    \caption{Esempio di estrazione di sotto-grafi frequenti basata sulla frequenza nei grafi del database.}
    \label{fig:frequent-subgraph-frequency}
\end{figure}

Non si fa riferimento al supporto perché così abbiamo un valore \emph{normalizzato}\footnote{La frequenza di un elemento è definita come il numero di elementi (il supporto in questo caso) diviso il numero totale (il numero di grafi nell'intero database in questo caso).}.

Il problema dell'estrazione di sotto-grafi frequenti pone sfide molto più complesse rispetto all'estrazione di insiemi frequenti, in quanto il problema dell' \textbf{isomorfismo tra sotto-grafi} è NP-completo. Questo significa che non esistono algoritmi efficienti noti per risolvere questo problema in tutti i casi.

\noindent
Per risolvere questo problema ci si pone delle domande:
\begin{itemize}
    \item Quale strategia adottare?
    \item Come generare in maniera efficiente i sottografi candidati?
    \item Come evitare o gestire le ridondanze nella generazione dei candidati?
\end{itemize}
Esistono due tipologie di algoritmi principali per l'estrazione di sotto-grafi frequenti:
\begin{itemize}
    \item \textbf{Algoritmi Apriori}: algoritmi che sono basati sulla regola \emph{apriori} e la generazione di candidati tramite join di sottografi frequenti.
    \item \textbf{Algoritmi pattern-growth}: algoritmi basati sulla generazione dei candidati tramite \emph{aggiunta di nodi/archi} ai sottografi frequenti esistenti.
\end{itemize}

\section{Algoritmo FSG}
Uno dei primi algoritmi proposti per l'estrazione di sotto-grafi frequenti è l'algoritmo \textbf{FSG} (Frequent Subgraph Discovery). FSG è un algoritmo basato sulla strategia \emph{apriori} che utilizza un approccio di generazione di candidati tramite join di sottografi frequenti.

\subsection{Regola Apriori per sotto-grafi}
Nel contesto dell'estrazione di sotto-grafi frequenti, la regola \emph{apriori} afferma che se un sotto-grafo è frequente, allora tutti i suoi sotto-grafi devono essere anch'essi frequenti. Questo implica che se un sotto-grafo non è frequente, allora nessuno dei suoi super-grafi può essere frequente (esattamente come nel caso degli itemsets\ref{ch:frequent-itemsets}).

\subsection{Join tra sotto-grafi}
Per capire come funzionano gli algoritmi basati sulla regola apriori, è importante comprendere come avviene la generazione dei candidati tramite join tra sotto-grafi. Due sotto-grafi $g_1$ e $g_2$ possono essere uniti se condividono un sottografo con archi in comune, chiamato \textbf{grafo core}. 

Il problema di questo tipo di Join è che, a differenza degli itemset, si possono generare più candidati distinti a partire dalla stessa coppia di sotto-grafi. Esistono 3 scenari possibili.

\paragraph{Scenario 1: i due sottografi differiscono per un nodo.}
\begin{figure}[htbp]
    \centering
    \includegraphics[width=\textwidth]{images/fsg-join-1.png}
    \caption{Join tra due sottografi che differiscono per un nodo.}
    \label{fig:fsg-join-1}
\end{figure}

Come si vede dall'esempio in figura \ref{fig:fsg-join-1}, i due sottografi condividono un sottografo comune, composto dalle etichette $a, b, c$, ma il secondo nodo $a$ differisce nel modo in cui è collegato (per $G_1^4$ è collegato a $c$, mentre per $G_2^4$ è collegato ad $a$). In questo caso esistono due possibili risultati:
\begin{enumerate}
    \item Tenere separate le due occorrenze del nodo $a$, ottenendo il grafo $G_1^5$.
    \item Unire i due nodi $a$ in un unico nodo, ottenendo il grafo $G_2^5$.
\end{enumerate}

\noindent
In questo caso i candidati vengono scelti in base al \emph{contesto}:
\begin{description}
    \item[Esempio 1]: Se i nodi $a, b, c$ rappresentano rispettivamente \emph{atomi di carbonio, idrogeno e ossigeno}, allora le due occorrenze del nodo $a$ rappresentano due atomi di carbonio distinti e quindi si sceglie il grafo $G_1^5$.
    \item[Esempio 2]: Se i nodi $a, b, c$ rappresentano rispettivamente \emph{persone, pubblicazioni e conferenze}, allora le due occorrenze del nodo $a$ rappresentano la stessa persona che ha pubblicato due articoli distinti e quindi si sceglie il grafo $G_2^5$.
\end{description}

\paragraph{Scenario 2: il grafo \emph{core} ha più automorfismi.}
\begin{figure}[htbp]
    \centering
    \includegraphics[width=\textwidth]{images/fsg-join-2.png}
    \caption{Join tra due sottografi con grafo core avente più automorfismi.}
    \label{fig:fsg-join-2}
\end{figure}

Nel caso di figura~\ref{fig:fsg-join-2} il \emph{core} comune dei due sottografi $G_1^5$ e $G_2^5$ è un ciclo di lunghezza~4
(etichettato tutto con $a$). Poiché tutti i vertici del core hanno la stessa etichetta, il core è altamente simmetrico: il ciclo $C_4$ ammette $8$ automorfismi (le $4$ rotazioni e le $4$ riflessioni del core).

Quando si esegue la \emph{join}, bisogna quindi decidere \emph{come} mappare il core di $G_2^5$ sul core di $G_1^5$. Automorfismi diversi producono posizionamenti diversi del nodo extra (in $G_1^5$ il nodo $b$, in $G_2^5$ il nodo $c$) rispetto al core. Dopo l'eliminazione dei duplicati isomorfi, restano tre candidati non isomorfi (raffigurati a destra nella figura):
\begin{enumerate}
    \item \textbf{$G_1^6$ (spigoli opposti).} Il nodo $b$ si collega a un estremo di un lato del quadrato e il nodo $c$ all'estremo del lato opposto.
    \item \textbf{$G_2^6$ (stesso lato).} I nodi $b$ e $c$ si collegano sullo \emph{stesso} lato del quadrato, ciascuno collegato ai due estremi di quel lato.
    \item \textbf{$G_3^6$ (condivisione di un vertice).} I nodi $b$ e $c$ condividono un estremo: entrambi sono collegati allo stesso vertice del core (e all'altro estremo del rispettivo lato adiacente).
\end{enumerate}

\noindent
Come nello Scenario~1, la scelta tra i candidati dipende dal \emph{contesto}:

\begin{description}
    \item[Esempio 1]: Se $a$ rappresenta atomi di \emph{carbonio} in un ciclo (ad es.\ un anello C$_4$) e $b,c$ sono due \emph{sostituenti}, i tre candidati corrispondono a tre posizionamenti non equivalenti: \emph{opposti} ($G_1^6$), \emph{adiacenti sullo stesso lato} ($G_2^6$) e \emph{geminali} sullo stesso carbonio ($G_3^6$). Il dominio (vincoli chimici o esempi osservati) decide quale configurazione mantenere.
    \item[Esempio 2]: Se $a$ sono \emph{autori} connessi da collaborazioni, $b$ e $c$ sono due \emph{pubblicazioni} che coinvolgono coppie di autori adiacenti, allora:
    \begin{itemize}
        \item due pubblicazioni scritte da \emph{coppie disgiunte} di autori $\Rightarrow G_1^6$;
        \item due pubblicazioni scritte dalla \emph{stessa coppia} $\Rightarrow G_2^6$;
        \item due pubblicazioni che \emph{condividono un autore} ma non l'altro $\Rightarrow G_3^6$.
    \end{itemize}
\end{description}

\paragraph{Scenario 3: i sottografi candidati hanno più grafi \emph{core} in comune.}
\begin{figure}[htbp]
    \centering
    \includegraphics[width=\textwidth]{images/fsg-join-3.png}
    \caption{Join tra due sottografi con più grafi core in comune.}
    \label{fig:fsg-join-3}
\end{figure}

Nell'esempio di figura~\ref{fig:fsg-join-3} i due pattern di partenza $G_1^4$ e $G_2^4$ condividono \emph{due} massimi sottografi comuni
(raffigurati in basso a sinistra), che chiamiamo $H_1^3$ e $H_2^3$.
La join va quindi eseguita due volte, una per ciascun core, perché
allineamenti diversi del nucleo comune portano a sovrapposizioni differenti dei nodi esterni.

\medskip
\noindent\textbf{Join rispetto al primo core $\mathbf{H_1^3}$.}
Prendendo $H_1^3$ come nucleo condiviso, i nodi rimanenti nei due pattern sono due vertici etichettati $a$ collegati a posizioni diverse del core. Come nello Scenario 1, abbiamo due possibilità:
\begin{enumerate}
    \item mantenere \emph{separate} le due occorrenze del nodo $a$ $\Rightarrow$ candidato $G_1^5$ (in alto a destra, primo grafo);
    \item \emph{fondere} le due occorrenze del nodo $a$ in un unico vertice $\Rightarrow$ candidato $G_2^5$ (in alto a destra, secondo grafo).
\end{enumerate}

\noindent\textbf{Join rispetto al secondo core $\mathbf{H_2^3}$.}
Ripetendo lo stesso procedimento usando il core $H_2^3$, otteniamo altri due candidati non isomorfi:
\begin{enumerate}
    \item mantenere \emph{separate} le due occorrenze del nodo $a$ $\Rightarrow$ candidato $G_3^5$ (in basso a destra, primo grafo);
    \item \emph{fondere} le due occorrenze del nodo $a$ in un unico vertice $\Rightarrow$ candidato $G_4^5$ (in basso a destra, secondo grafo).
\end{enumerate}

\medskip
\noindent
Come per gli scenari precedenti, la scelta del candidato dipende dal \emph{contesto}:
\begin{description}
    \item[Esempio 1]: Sia $a$ un \emph{carbonio} e $b$ un \emph{eteroatomo}\footnote{Un eteroatomo è un atomo diverso dal carbonio in un composto organico.} (ad es.\ ossigeno) su una catena. Se i due frammenti osservati provengono da \emph{due sostituenti distinti} allora si mantengono i nodi separati (si seleziona $G_1^5$ o $G_3^5$ a seconda del core coerente con i dati); se invece i dati indicano che si tratta dello \emph{stesso} sostituente visto in due prospettive compatibili, si effettua la fusione ($G_2^5$ o $G_4^5$).
    \item[Esempio 2]: Sia $a$ un \emph{autore} e $b$ un \emph{progetto}. I due core alternativi rappresentano due gruppi di co-autori sovrapposti in modo diverso. Se le osservazioni mostrano che le due occorrenze di $a$ sono \emph{persone diverse}, si sceglie un candidato con nodi separati ($G_1^5$ o $G_3^5$); se sono la \emph{stessa persona} che compare in entrambi i pattern, si sceglie il candidato con merge ($G_2^5$ o $G_4^5$).
\end{description}

\paragraph{Caso generale.}
Il caso generale della Join tra due sottografi $g_1$ e $g_2$ prevede i seguenti passi:
\begin{enumerate}
    \item Sia $F^k$ l'insieme dei sottografi frequenti con $k$ archi e sia $C^{k+1}$ l'insieme dei candidati di sotto-grafi di dimensione $k+1$ (inizialmente vuoto).
    \item Per ogni coppia di sottografi frequenti con $k$ archi ($G_1^k, G_2^k$) in $F^k$:
    \begin{enumerate}
        \item Calcola l'insieme dei core condivisi da $G_1^k$ e $G_2^k$.
        \item Per ogni core condiviso e per ogni automorfismo del core, effettua la join e aggiungi i sottografi candidati trovati a $C^{k+1}$.
    \end{enumerate}
    \item Restituisci l'insieme $C^{k+1}$ dei sottografi candidati di dimensione $k+1$.
\end{enumerate}

\subsection{Procedura dell'algoritmo}
L'algoritmo FSG segue i seguenti passi:
\begin{enumerate}
    \item Per prima cosa calcola nodi e archi frequenti (sotto-grafi di dimensione 1 e 2) e li aggiunge all'insieme finale $O$ dei risultati.
    \item Per un certo $k \ge 3$ ripete:
    \begin{enumerate}
        \item Genera i candidati $C_k$ di sotto-grafi di dimensione $k$ tramite join dei sotto-grafi frequenti di dimensione $k-1$.
        \item Per ogni grafo $G$ nel database verifica la regola Apriori, scartando tutti i candidati che contengono almeno un sottografo con archi che non è frequente.
        \item Calcola il supporto di ogni grafo candidato e l'insieme dei sottografi frequenti con $k$ archi. Poi aggiunge  questi sottografi all'insieme finale $O$ dei risultati.
    \end{enumerate}
    \item Restituisci $O$
\end{enumerate}

\subsection{Generazione dei candidati}
Il passo più complesso dell'algoritmo FSG è la generazione dei candidati. Per generare i candidati di dimensione $k$, l'algoritmo esegue una join tra tutti i sottografi frequenti di dimensione $k-1$. Una volta eseguita la join si potrebbe presentare il problema dei \textbf{candidati ridondanti}, ovvero sottografi isomorfi generati più volte. Per evitare questo problema, FSG utilizza una tecnica di \textbf{canonical labeling} (forma canonica), che assegna ad ogni sottografo un'etichetta unica basata sulla sua struttura e sulle etichette dei nodi e degli archi. In questo modo,se due sottografi isomorfi hanno la stessa etichetta canonica, uno dei due può essere scartato come ridondante.

\subsection{Stringa di adiacenza}
La stringa di adiacenza di un grafo è una stringa ottenuta concatenando le righe della matrice di adiacenza (nel caso di grafi indiretti si considera la metà superiore della matrice) come si vede nell'esempio in figura \ref{fig:adjacency-string-example}

\begin{figure}[htbp]
    \centering
    \includegraphics[width=0.8\textwidth]{images/adjacency_string_example.png}
    \caption{Esempio di stringa di adiacenza di un grafo.}
    \label{fig:adjacency-string-example}
\end{figure}

\subsection{Forma canonica}
Per un grafo $G$ con $n$ nodi, esistono $n!$ possibili modi di etichettare i nodi (ovvero $n!$ stringhe di adiacenza) considerando tutte le possibili permutazioni dei nodi di $G$ (diverse permutazioni possono produrre la stessa stringa di adiacenza). La \textbf{forma canonica} di un grafo $G$ è definita come la stringa di adiacenza minima (o massima), in ordine lessicografico, tra tutte le possibili stringhe di adiacenza generate dalle permutazioni dei nodi di $G$. Il problema di questo calcolo è che \emph{computazionalmente oneroso} e FSG risolve questo problema utilizzando \textbf{gradi dei nodi} e \textbf{etichette} per ridurre il numero di permutazioni da considerare.

\noindent
\paragraph{Algoritmo.}
Il calcolo della forma canonica segue questi passi:
\begin{enumerate}
    \item Partiziona i nodi in gruppi sulla base del grado.
    \item Partiziona ogni gruppo ottenuto al passo precedente in sottogrupi sulla base delle etichette dei nodi.
    \item Considera le $k$ stringhe di adiacenza che è possibile ottenere permutando in tutti i modi possibili i nodi in ciascun sottogruppo.
    \item Tra le $k$ stringhe scegli quella lessicograficamente più piccola.
\end{enumerate}

\noindent
\paragraph{Esempio.}
Considerando l'immagine in figura \ref{fig:canonical-form-example} il grafo ha etichette $\{a,b\}$ e gradi $\deg(v_1)=3$, $\deg(v_2)=\deg(v_3)=\deg(v_4)=1$. Il partizionamento per grado produce i blocchi $D_3=\{v_1\}$ e $D_1=\{v_2,v_3,v_4\}$. 
\begin{figure}[htbp]
    \centering
    \includegraphics[width=0.9\textwidth]{images/canonical-form-example.png}
    \caption{Esempio di calcolo della forma canonica di un grafo.}
    \label{fig:canonical-form-example}
\end{figure}

\noindent
Raffinando\footnote{In questo contesto, raggruppare i nodi per etichetta} per etichetta otteniamo:
\[
D_1^a=\{v_4,v_2\},\qquad D_1^b=\{v_3\},\qquad D_3^a=\{v_1\}.
\]
Gli unici nodi permutabili sono quelli nel blocco $D_1^a$, quindi le
permutazioni ammesse (mostrate in figura) generano due ordini dei vertici e,
di conseguenza, due stringhe di adiacenza (si leggono le
$6$ voci sopra la diagonale nell'ordine $A_{12}A_{13}A_{14}A_{23}A_{24}A_{34}$):
\[
\pi_1=(v_4,v_1,v_3,v_2)\;\Rightarrow\; s_1=\texttt{00y0xx},\qquad
\pi_2=(v_1,v_4,v_3,v_2)\;\Rightarrow\; s_2=\texttt{00x0yx}.
\]
Assumendo l'ordinamento lessicografico $0<x<y$, risulta
$\texttt{00x0yx}<\texttt{00y0xx}$. 
Pertanto, la \textbf{forma canonica} del grafo è la stringa
\[
\boxed{\texttt{00x0yx}}
\]
corrispondente all'ordine dei vertici $\pi_2=(v_1,v_4,v_3,v_2)$ mostrato nella figura.

\subsection{Verifica della regola Apriori}
Per ogni sottografo candidato $G$ con $k$ archi, l'algoritmo FSG verifica la regola Apriori controllando che tutti i sottografi di $G$ con $k-1$ archi siano frequenti. Se almeno uno di questi sottografi non è frequente, allora $G$ viene scartato come candidato.

Per contare la frequenza degli $m$ $k$-sottografi candidati al passo $k$ si dovrebbero risolvere $n \cdot k$ problemi di subgraph matching, con $n$ numero di grafi nel database. Grazie all'indicizzazione inversa, si può ridurre il numero di problemi di subgraph matching da risolvere. L'idea è di costruire un indice che mappa ogni sottografo frequente di dimensione $k-1$ ai sottografi candidati di dimensione $k$ che lo contengono. 

\noindent
In particolare
\begin{itemize}
    \item Ad ogni sottografo $S$ frequente di dimensione $k-1$ viene associata una lista di sottografi candidati (TID List: Transaction IDentifier List) di dimensione $k$ che lo contengono.
    \item Per ogni $(k+1)$-sottografo candidato $C$, si recuperano le TID List di tutti i suoi $k$-sottografi e si calcola l'intersezione di queste liste per ottenere la lista dei grafi che contengono $C$.
    \item Se la dimensione della lista ottenuta è minore della soglia di supporto, allora $C$ viene scartato come candidato.
\end{itemize}

\section{Algoritmo gSpan}
Un altro algoritmo popolare per l'estrazione di sotto-grafi frequenti è l'algoritmo \textbf{gSpan} (graph-based Substructure pattern mining). gSpan utilizza un approccio di \textbf{pattern-growth} per generare i candidati, evitando la necessità di eseguire join tra sottografi frequenti. In modo analogo si potrebbe pensare l'approccio \emph{pattern-growth} come una \textbf{depth-first search} (DFS) nello spazio dei sottografi, dove si parte da un sottografo frequente e si aggiungono nodi o archi per generare nuovi sottografi candidati.

\subsection{Visita DFS}
La visita DFS è un algoritmo di ricerca che esplora il grafo dando priorità ai nodi più profondi prima di tornare indietro. Gli archi vengono esplorati infatti a partire dall'\textbf{ultimo nodo} scoperto $v$ che presenta archi non ancora esplorati Terminata l'esplorazione di tutti gli archi uscenti da $v$, si torna indietro al nodo precedente nella pila e si continua l'esplorazione da lì. Un esempio può essere visto in figura \ref{fig:dfs_visit_example}.

\begin{figure}[htbp]
    \centering
    \includegraphics[width=0.8\textwidth]{images/dfs_visit_example.png}
    \caption{Esempio di visita in profondità (DFS) su un grafo diretto. I numeri all'interno dei nodi indicano i tempi di scoperta e di completamento (d/f) di ciascun vertice. I nodi colorati in grigio o nero rappresentano rispettivamente i vertici scoperti e completati, mentre le frecce tratteggiate indicano archi di ritorno o di attraversamento classificati durante l'esecuzione.}
    \label{fig:dfs_visit_example}
\end{figure}

\paragraph{Albero DFS.}
Durante la visita DFS di un grafo $G$ viene prodotto un albero $T$, chiamato \emph{albero DFS}, i cui nodi sono i nodi di $G$, mentre gli archi sono gli archi di $G$ esplorati durante la visita e che hanno portato alla scoperta di nuovi nodi.

\noindent
Gli archi durante la costruzione dell'albero vengono divisi in due tipi:
\begin{itemize}
    \item \textbf{Archi forward}: archi di $G$ presenti in $T$ che collegano un nodo a uno dei suoi discendenti nell'albero DFS.
    \item \textbf{Archi back}: archi di $G$ non presenti in $T$ che collegano un nodo a uno dei suoi antenati nell'albero DFS.
\end{itemize}

\noindent
Un esempio di albero DFS può essere visto in figura \ref{fig:dfs_tree_example}.

\begin{figure}[htbp]
    \centering
    \includegraphics[width=\textwidth]{images/dfs_tree_example.png}
    \caption{A sinistra il grafo di partenza; a destra alcuni alberi di visita in profondità (DFS) ottenuti con diversi ordini di esplorazione. Gli archi forward (dell'albero) sono in nero continuo, mentre gli archi backward sono tratteggiati, come indicato in legenda.}
    \label{fig:dfs_tree_example}
\end{figure}

\subsection{Codifica DFS}
L'algoritmo gSpan utilizza una codifica DFS per rappresentare i sottografi. La codifica DFS è una sequenza di tuple che rappresentano gli archi del grafo in base all'ordine di visita DFS. Considerando un arco $(u, v)$ con etichette $l_u$ e $l_v$ sui nodi $u$ e $v$ e $l_{(u,v)}$ sull'arco, la tupla che rappresenta l'arco è definita come:
\[
\langle i, j, l_u, l_{(u,v)}, l_v \rangle
\]
dove $i$ e $j$ indicano \textbf{il tempo di visita} del nodo $u$ e del nodo $v$ durante la visita DFS.

\paragraph{Costruzione della codifica DFS.}
Per costruire la codifica DFS di un grafo $G$ si seguono i seugenti passi:
\begin{enumerate}
    \item Dato un nodo $v$, tutti i suoi \emph{archi uscenti backward} devono essere elencati per primi, ordinati in base alle etichette dei nodi di destinazione e delle etichette degli archi.
    \item Tra gli \emph{archi forward} che partono dallo stesso nodo $v$, si seleziona l'arco con destinazione un nodo già visitato prima: $(u, v) < (u, v') \Leftrightarrow t(v) < t(v')$.
    \item Tra gli \emph{archi forward} che partono da nodi diversi, si seleziona quello con sorgente con tempo di visita minore: $(u, v) < (u', v') \Leftrightarrow t(u) < t(u')$.
    \item Tra gli \emph{archi backward} che partono dallo stesso nodo, si seleziona quello con destinazione uun nodo visitato prima, ovvero con tempo di visita minore: $(u, v) < (u, v') \Leftrightarrow t(v) < t(v')$.
    \item Tra gli \emph{archi backward} che partono da nodi diversi, si seleziona quello con sorgente con tempo di visita minore: $(u, v) < (u', v') \Leftrightarrow t(u) < t(u')$.
\end{enumerate}

\noindent
Si trova la codifica DFS minima del grafo in figura \ref{fig:dfs_tree_example} nella tabella \ref{tab:dfs_code_example}.

\begin{table}[htbp]
\centering
\begin{tabular}{lccc}
\toprule
\textbf{edge no.} & \textbf{(b) $\alpha$} & \textbf{(c) $\beta$} & \textbf{(d) $\gamma$} \\
\midrule
0 & $(0,1,X,a,Y)$ & $(0,1,Y,a,X)$ & $(0,1,X,a,X)$ \\
1 & $(1,2,Y,b,X)$ & $(1,2,X,a,X)$ & $(1,2,X,a,Y)$ \\
2 & $(2,0,X,a,X)$ & $(2,0,X,b,Y)$ & $(2,0,Y,b,X)$ \\
3 & $(2,3,X,c,Z)$ & $(2,3,X,c,Z)$ & $(2,3,Y,b,Z)$ \\
4 & $(3,1,Z,b,Y)$ & $(3,0,Z,b,Y)$ & $(3,0,Z,c,X)$ \\
5 & $(1,4,Y,d,Z)$ & $(0,4,Y,d,Z)$ & $(2,4,Y,d,Z)$ \\
\bottomrule
\end{tabular}
\caption{Codifica degli archi per i tre schemi (b)~$\alpha$, (c)~$\beta$ e (d)~$\gamma$.}
\label{tab:dfs_code_example}
\end{table}

\paragraph{Codice DFS minimo.}
Si definisce il particolare, tra tutti i possibili codici DFS di un grafo $G$, il \textbf{codice DFS minimo} come il codice che è lessicograficamente più piccolo tra tutti i codici DFS di $G$. Questo codice viene utilizzato come rappresentazione unica del grafo per evitare ridondanze durante la generazione dei candidati. Nella figura \ref{fig:dfs_tree_example} il codice DFS minimo, rappresentato in tabella \ref{tab:dfs_code_example}, corrisponde allo schema: (d) $\gamma$.

\paragraph{DFS Code Tree.}
Sempre dalla visita DFS e dall'albero DFS si può generare un albero chiamato \textbf{DFS Code Tree}, in cui ogni nodo rappresenta un codice DFS minimo di un sottografo. La radice dell'albero rappresenta il codice vuoto, il primo livello rappresenta i sottografi con un arco, il secondo livello rappresenta i sottografi con due archi, il livello $k$ rappresenta i sottografi con $k+1$ archi, e così via. Ogni figlio di un nodo rappresenta un'estensione del sottografo rappresentato dal nodo padre, ottenuta aggiungendo un arco.

\subsection{Generazione dei candidati}
In gSpan un sottografo candidato con $k+1$ archi viene generato per aggiunta di un singolo arco a partire da un sottografo frequente con $k$ archi. Per \emph{non} generare ridondanze, gSpan permette di aggiungere archi solo in due modi:
\begin{itemize}
    \item Un arco da un nodo che sta nel cammino \textbf{più a destra} dalla radice ad un nodo foglia nell'albero DFS associato al codice DFS minimo.
    \item Un arco dal nodo foglia del cammino \emph{più a destra} dell'albero DFS associato al codice DFS minimo ad un nuovo nodo (qualsiasi).
\end{itemize}

\noindent
Un esempio può essere visto in figura \ref{fig:gspan_candidate_generation}.

\begin{figure}[htbp]
    \centering
    \includegraphics[width=\textwidth]{images/gspan_candidate_generation.png}
    \caption{Esempio di generazione dei candidati in \textsc{gSpan}. 
    (a) Sottografo frequente di partenza. 
    I candidati con $k\!+\!1$ archi si ottengono aggiungendo un solo arco secondo la regola del \emph{cammino più a destra} del codice DFS minimo: 
    (b-d) aggiunte \emph{backward} da un nodo del cammino più a destra verso un antenato; 
    (e-f) aggiunte \emph{forward} dal nodo foglia del cammino più a destra verso un nuovo nodo. 
    Gli archi tratteggiati indicano l'arco aggiunto. 
    La riga inferiore (b.0-b.3, e.0-e.2) illustra le varianti non ridondanti prodotte per ciascuna estensione rispettando l'ordine del codice DFS minimo.}
    \label{fig:gspan_candidate_generation}
\end{figure}

\paragraph{Pruning dello spazio di ricerca.}
I grafi candidati con $k+1$ archi generati a partire da un grafo frequente con $k$ archi vengono processati seguendo l'ordine lessicografico dei loro codici DFS minimi. Se generiamo un grafo candidato $G_1$ con lo stesso codice di un grafo $G_0$ già esaminato, possiamo fare un pruning dell'intero sottoalbero radicato in $G_1$, evitando di esaminare ulteriori grafi candidati generati a partire da $G_1$. Questo perché tutti i grafi in questo sottoalbero avranno codici DFS minimi maggiori o uguali a quello di $G_1$, e quindi saranno già stati esaminati quando abbiamo esaminato $G_0$.

\section*{Riferimenti}
I riferimenti di questo capitolo includono:
\begin{itemize}
    \item Materiale visto a lezione.
    \item Approfondimenti nel libro \cite{CookHolder2006}.
\end{itemize}
\chapter{Elementi di Reti Neurali}
\chapter{Introduzione a Transformer e LLM}
I transformer e i modelli di linguaggio di grandi dimensioni (LLM) hanno rivoluzionato il campo dell'elaborazione del linguaggio naturale (NLP) e dell'intelligenza artificiale. In particolare, i transformer hanno introdotto un nuovo paradigma per la modellazione delle sequenze, superando molte delle limitazioni dei modelli precedenti come le reti neurali ricorrenti (RNN) e le Long Short-Term Memory (LSTM).

\section{Transformer}
Le reti neurali ricorrenti (RNN) sono particolarmente \emph{efficaci} nella gestione di \textbf{sequenza di dati}, come il testo, grazie alla loro capacità di mantenere uno stato interno che può catturare informazioni a lungo termine.

\subsection{Seq2seq}
Una tipologia di RN, nota come \textbf{seq2seq}, è stata utilizzata per \textbf{trasformare} una sequenza di elementi (come delle parole in una frase) in un'altra sequenza di elementi (come la traduzione di quella frase in un'altra lingua). Questa architettura è stata ampiamente adottata in compiti di traduzione automatica, generazione di testo e altre applicazioni di NLP, in particolare grazie a un modello specifico di seq2seq di tipo \textbf{LSTM}, che memorizza le dipendenze tra termini importanti in sequenze molto lunghe.

\begin{figure}
    \centering
    \includegraphics[width=0.8\textwidth]{images/seq2seq.png}
    \caption{Schema di un modello \emph{encoder--decoder} basato su reti neurali ricorrenti.
    L'\textbf{encoder} riceve in ingresso una sequenza $x^{(1)}, x^{(2)}, \ldots, x^{(T)}$ e la comprime in una rappresentazione latente finale.
    Il \textbf{decoder} utilizza tale rappresentazione per generare iterativamente la sequenza di output
    $y^{(1)}, y^{(2)}, \ldots, y^{(T')}$, producendo a ogni passo un simbolo di uscita fino al token di fine sequenza (\texttt{<End>}).
    Le frecce ricorsive indicano la dipendenza temporale tra gli stati consecutivi.}
\end{figure}

\paragraph{Limitazioni delle RNN.}
Una delle limitazioni di questi modelli di reti neurali è

\part{Approfondimenti}

\backmatter
\printbibliography  

\end{document}
